\documentclass[12pt]{report}
%
\usepackage[english]{babel}
\usepackage[font=small,labelfont=bf,format=hang]{caption}
\usepackage[utf8]{inputenc}
\usepackage{graphicx}
\usepackage{color}
\usepackage{geometry}
\usepackage[comma,numbers,sort&compress]{natbib}
%
\usepackage{amsmath}
\usepackage{rotating}
\usepackage{listings}
%
%\setcounter{secnumdepth}{2}
\textwidth 150 mm
\textheight 200 mm
\oddsidemargin 0pt
\evensidemargin 0pt
\topmargin 0pt
\headheight 0pt
\headsep 1.25cm
\renewcommand{\baselinestretch}{1.15}
%
\newcommand{\pluseq}{\mathrel{+}=}
\newcommand{\und}{\underline}
%
%------------------------------------------------------------------
% define new verbatim macro
\def\beginverbatim{\par\bigskip\begingroup\setuplisting\doverbatim}
 {\catcode`\|=0 \catcode`\\=12 
   |obeylines|gdef|doverbatim#1\endverbatim{#1|bigskip|endgroup}}
% 
% define listing macro
\def\listing#1 {\par\bigskip\begingroup\setuplisting\input#1 
 \bigskip\endgroup}

\newcount \lineno  % the number of lines listed
\def\setuplisting{ \small \tt \hsize=8in 
    \lineno=0
    \parindent=0pt
    \baselineskip=0.27cm
    \lineskip=0.0cm
    \lineskiplimit=0.0cm
    \def\par{\leavevmode\endgraf} \catcode`\`=\active 
    \obeylines 
    \uncatcodespecials
    \obeyspaces
    \everypar{\advance\lineno by1 \llap{\sixrm\the\lineno\ \ }}}
   {\obeyspaces\global\let =\ }  % let active space = control space

\def\uncatcodespecials {\def\do##1{\catcode`##1=12}\dospecials}
% end listing macro
%
\hfuzz 20pt
%------------------------------------------------------------------
%
\begin{document}
%
\bibliographystyle{/home/simonson/tex/misc/acmunsrt}

\hfill \today

\vspace*{3cm}

\Large
\begin{center}
{\bf Implicit solvent models in X-PLOR}
\end{center}

\vskip 1cm

\large

\centerline{Thomas Simonson, Savvas Polydorides, Francesco Villa, David Mignon}
  \vskip 0.7cm

\begin{center}
Laboratoire de Biochimie, Ecole Polytechnique \\
\normalsize
thomas.simonson@polytechnique.fr
\end{center}

\vskip 2cm
\normalsize

\pagebreak
\pagestyle{headings}
\chapter{Generalized Born models}
\section{Introduction}
The Generalized Born (GB) model \cite{Still90,Hawkins95,Schaefer96,Qiu97} is an
efficient and accurate implicit solvent model for biomolecular simulations and
protein design. It describes the solvent around the biomolecule as a dielectric
continuum. But the numerical complexities of an inhomogeneous solute/solvent dielectric
system are swept away and replaced by approximate, efficient, analytical formulas.
The model can be used either to determine the energy of a single structure or
to generate multiple structures by molecular dynamics or simulated annealing. Several
review articles describe the theoretical background, the performance, and the ongoing
progress of the GB model; see eg \cite{Bashford00,Roux99,Simonson03}. Two GB variants have
been implemented in X-PLOR \cite{Xplor} and CNS \cite{CNS}. The first is termed GB/ACE
(Schaefer \& Karplus, {\it J. Phys.\ Chem.}, 1996, 100:1578), for ``Analytical Continuum
Electrostatics''; the second is termed GB/HCT, for ``Hawkins, Cramer \& Truhlar'' (HCT,
{\it Chem.\ Phys.\ Lett.}, 1995, 246:122). We emphasize that the GB solvation model decribes
the solvent response to the charges and Coulomb potential of the solute. Therefore, it
is meaningless to use GB in a simulation or structure refinement where the ordinary
electrostatics energy term is turned off.  

The Theory section below reviews the GB/ACE and GB/HCT models. Expressions of the
solvation energies and forces are given. This section can be skipped by those already
familiar with the model. The following section, Syntax, gives the necessary syntax and
the default options for using GB in X-PLOR. 

\section{Theory}
\subsection{GB energy}
In the world of continuum electrostatics, a biomolecular solute is viewed as
a set of (fractional) atomic charges in a cavity delimited by the solute surface,
embedded in a high dielectric solvent medium \cite{Kirkwood38}. The electrostatic
energy $E^{\rm elec}$ is the sum of the Coulomb interaction energies between all
solute charges and a solvation term $\Delta E^{\rm solv}$; the latter includes the
interaction energies of each solute charge with solvent (its ``self-energy''), and
a solvent-screening contribution to the interaction energies between solute charges:
\begin{eqnarray}
E^{\rm elec} &=& \sum_{i < j} \frac{q_i q_j}{r_{ij}} + \Delta E^{\rm solv} \\
\Delta E^{\rm solv} &=& 
   \sum_i \Delta E_i^{\rm self} + \sum_{i < j} \Delta E_{ij}^{\rm int}  \label{eq:gbenergy}
\end{eqnarray}

In the GB model, the solvent contribution $\Delta E^{int}_{ij}$ to the interaction
energy between the charges $q_i$ and $q_j$ is approximated by \cite{Still90}:
\begin{equation} \label{eq:gbinte}
\Delta E^{\rm int}_{ij} =
  -\frac{\tau  q_i q_j }{(r_{ij}^2 + b_i b_j \exp[-r_{ij}^2/4 b_i b_j])^{1/2}}
\end{equation}
where $r_{ij}$ is the distance between the charges, $\tau$ is given by
\begin{equation}
\tau = 1/\epsilon_p - 1/\epsilon_w
\end{equation}
$\epsilon_p$, $\epsilon_w$ are the protein and solvent dielectric constants, and $b_i$ is the `solvation
radius' of charge $i$. By analogy to the case of a single charge in a spherical cavity, $b_i$ is defined by
\begin{equation}  \label{eq:bi}
\Delta E_i^{\rm self} = - \frac{\tau q_i^2}{2 b_i}
\end{equation}
where $\Delta E_i^{\rm self}$ is the self-energy of charge $i$. By partitioning the
solute into atomic volumes (following Lee \& Richards, for example \cite{Lee71}),
one can express the self-energy $\Delta E_i^{\rm self}$ as a sum over all the solute
atoms \cite{Hawkins95,Schaefer96}:
\begin{equation} \label{eq:self}
\Delta E_i^{\rm self} = -\frac{\tau q_i^2}{2 R_i} + \tau q_i^2 \sum_{k \neq i} E_{ik}^{\rm self}
\end{equation}
where $R_i$ is a constant atomic radius to be determined (close to the
van der Waals radius) and $E_{ik}^{\rm self}$ is related to the integral of
the electrostatic energy over the volume of atom $k$. Notice that the
charges of the other atoms, $q_k$, do not appear here. The effect of
these atoms is merely to exclude solvent from the vicinity of atom $i$
\cite{Schaefer90}.

The volume integral $E_{ik}$ is approximated in two steps. The first
step is to approximate the electric field by the `Coulombic field'
of charge $i$ \cite{Schaefer90}. This is simply the unscreened field that
would exist if $q_i$ were in a vacuum; it radiates uniformly in all directions
and falls off as $1/r^2$ with distance; the corresponding energy density
is $1/r^4$. The next step is to calculate the integral of $1/r^4$ over the
volume of atom $k$. The different GB variants do this in different ways.
In GB/ACE, for example, Schaefer \& Karplus assume the density of each solute
atom is a gaussian centered at the atom's position. The integral $E_{ik}$
then has a tractable form, which can be approximated by interpolating
between a Gaussian form at short ranges and a $1/r^4$ form at long range, 
leading to the Ansatz \cite{Schaefer96}:
\begin{equation} \label{eq:ace}
E_{ik}^{\rm self} = \frac{1}{\omega_{ik}} \exp(-r_{ik}^2/\sigma^2_{ik}) + 
      \frac{V_k}{8 \pi} \left( \frac{r_{ik}^3}{r_{ik}^4 + \mu^4_{ik}} \right)^4
\end{equation}
Here, $\omega_{ik}$ and $\mu_{ik}$ are simple functions of the atomic
volume $V_k$, the atomic radii $R_i$, $R_k$ (= $[3V_k/4\pi]^{1/3}$),
and an adjustable ``smoothing'' parameter $\alpha$ that determines the
width of the atomic gaussian distributions (see below).
The atomic charges are taken directly from the existing force
field. The adjustable parameters of the model are then the volumes $V_k$
and the smoothing parameter $\alpha$. Ionic strength is not included,
although methods to do so have been proposed \cite{Onufriev00,Srinivasan99}.
Volumes $V_k$ can be either calculated using Voronoi polyhedra (using an
external program \cite{Lee71} and reading them into X-PLOR), or assigned
values from existing libraries \cite{Schaefer96,Onufriev00,Schaefer01,Lopes07}. 
Note that the $V_k$ are considered to be constants, independent of the
solute conformation. This is important to obtain tractable expressions for
the GB forces (see below).

With the above self-energy approximations, $\Delta E^{\rm self}_i$ can sometimes
become positive, so that the (necessarily positive) solvation radius can no longer
be defined by Eq.\ (\ref{eq:bi}). Therefore, we use a definition proposed by
Schaefer et al.\ \cite{Schaefer98a}:
\begin{eqnarray}
b_i &=& - \frac{\tau q_i^2}{2 \Delta E^{\rm self}_i}
     \hspace*{25mm} {\rm if } \, 
   \Delta E^{\rm self}_i \leq E_{min} = - \frac{\tau q_i^2}{2 b_{max}}
                   \nonumber \\
&=& b_{max} \left( 2 - \frac{\Delta E^{\rm self}_i}{E_{min}} \right)
     \hspace*{8mm} {\rm if } \, 
   \Delta E^{\rm self}_i \geq E_{min}  \label{eq:bi2}
\end{eqnarray}
Here, $b_{max}$ is an upper limit for the solvation radius, which can be set to
the largest linear dimension of the solute, for example. This definition leads
to continuous energies and forces.

\subsection{Calculation of forces}
\subsubsection{Interaction energy term}
We first consider the GB `interaction' term, on the far right of
Eq.\ (\ref{eq:gbenergy}), and its gradient $\nabla_n$ with respect
to the position of solute particle $n$. Noting that the solvation
radii $b_i$, $b_j$ depend on all the atomic positions and using the
chain rule for differentiation, we have:
\begin{equation} \label{eq:togroup}
\nabla_n \sum_{i<j} \Delta E^{\rm int}_{ij} =
  \sum_{i<j} \frac{\partial \Delta E^{\rm int}_{ij}}{\partial r_{ij}} \nabla_n r_{ij} +
  \sum_{i<j} \frac{\partial \Delta E^{\rm int}_{ij}}{\partial b_i} \nabla_n b_i +
  \sum_{i<j} \frac{\partial \Delta E^{\rm int}_{ij}}{\partial b_j} \nabla_n b_j
\end{equation}
Only terms with $i =n$ or $j=n$ contribute to the first sum on the right.
The second sum can be written
\begin{equation} 
\sum_{i<j} \frac{\partial \Delta E^{\rm int}_{ij}}{\partial b_i} \nabla_n b_i =
\frac{1}{2} \sum_i \left( 
   \sum_{j \neq i} \frac{\partial \Delta E^{\rm int}_{ij}}{\partial b_i} 
                   \right) 
   \frac{\partial b_i}{\partial \Delta E^{\rm self}_i} \nabla_n \Delta E^{\rm self}_i
\end{equation}
The quantity in parentheses will be denoted $dE^{\rm int,b}_i$,
since, for a given conformation, it depends only on $i$. The last quantity
on the right can be written:
\begin{eqnarray}
\nabla_n \Delta E^{\rm self}_i = \sum_{k \neq i} \nabla_n E^{\rm self}_{ik}
&=&  \nabla_n E^{\rm self}_{in} \hspace*{11mm} {\rm if } \, i \neq n \nonumber \\
&=& \sum_{k \neq n} \nabla_n E^{\rm self}_{nk} \hspace*{5mm} {\rm if } \, i = n.
\end{eqnarray}
Grouping the second and third terms on the right of (\ref{eq:togroup}) and
rearranging the first, we obtain:
\begin{equation} \label{eq:force}
\nabla_n \sum_{i<j} \Delta E^{\rm int}_{ij} =
 \sum_{i \neq n} \left( \frac{\partial \Delta E^{\rm int}_{in}}{\partial r_{in}} +
  dE^{\rm int,b}_n \frac{\partial b_n}{\partial \Delta E^{\rm self}_n} 
    \frac{\partial E^{\rm self}_{ni}}{\partial r_{in}} +
   dE^{\rm int,b}_i \frac{\partial b_i}{\partial \Delta E^{\rm self}_i}
       \frac{\partial E^{\rm self}_{in}}{\partial r_{in}} \right) \frac{\und{r}_n - \und{r}_i}{r_{in}}
\end{equation}
with
\begin{eqnarray}
dE^{\rm int,b}_i &=& 
   \sum_{j \neq i} \frac{\partial \Delta E^{\rm int}_{ij}}{\partial b_i} \label{eq:deintb} \\
\frac{\partial b_n}{\partial \Delta E^{\rm self}_n} &=&
     - \frac{b_n}{\Delta E^{\rm self}_n}       \hspace*{20mm} {\rm if } \, 
   \Delta E^{\rm self}_n \leq E_{min} =  - \frac{\tau q_n^2}{2 b_{max}} \nonumber \\
&=& - \frac{b_{max}}{E_{min}}  \hspace*{23mm} {\rm if } \, 
                                     \Delta E^{\rm self}_n \geq E_{min} \label{eq:dbdE}
\end{eqnarray}
The quantities $b_i$ and $dE^{\rm int,b}_i$ can be `precalculated', so that obtaining
the force on atom $n$ requires only a loop over all solute atoms. In (\ref{eq:force}),
the derivatives of $\Delta E^{\rm int}_{in}$ are the same for GB/ACE and GB/HCT:
\begin{eqnarray}
\frac{1}{r_{in}} \frac{\partial \Delta E^{\rm int}_{in}}{\partial r_{in}} &=&
  \frac{ \tau q_i q_j }{\left[ r_{ij}^2 + b_i b_j \exp(-\frac{r_{ij}^2}{4b_i b_j})\right]^{3/2}}
\left( 1 - \frac{1}{4} \exp(-\frac{r_{ij}^2}{4b_i b_j}) \right) \\
dE^{\rm int,b}_i &=& \sum_{j \neq i} 
  \frac{ \frac{1}{2} \tau q_i q_j b_j \exp(-\frac{r_{ij}^2}{4b_i b_j})}
       {\left[r_{ij}^2 + b_i b_j \exp(-\frac{r_{ij}^2}{4b_i b_j})\right]^{3/2}}
   \left(1 +  \frac{r_{ij}^2}{4b_i b_j} \right) 
\end{eqnarray}

\subsubsection{GB/ACE self-energy term}
The self-energy and the associated forces depend on the GB variant. With GB/ACE,
\begin{equation} \label{eq:acef}
\frac{1}{r_{ij}} \frac{\partial E_{ij}^{\rm self}}{\partial r_{ij}} 
   = - \frac{2}{\omega_{ij}\sigma_{ij}^2} \exp(-\frac{r_{ik}^2}{\sigma^2_{ik}}) + 
       \frac{V_j}{2 \pi} \left( \frac{r_{ij}^{10}}{r_{ij}^4 + \mu^4_{ij}} \right)^5
                         \left( 3 (r_{ij}^4 + \mu^4_{ij}) - 4 r_{ij}^4 \right).
\end{equation}
The parameters $\omega_{ij}$, $\sigma_{ij}$, $\mu_{ij}$ are defined by:
\begin{eqnarray}
\frac{1}{\omega_{ik}} &=& \frac{4}{3 \pi \alpha_{ik}^3}
        (Q_{ik} - {\rm arctan} Q_{ik}) \frac{1}{\alpha_{ik} R_k} \\
\sigma_{ik}^2 &=& \frac{3 (Q_{ik} - {\rm arctan} Q_{ik})}
                       {(3+f_{ik})Q_{ik} - 4 {\rm arctan} Q_{ik}} \alpha_{ik}^2 R_{ik}^2 \\
Q_{ik} &=& \frac{q_{ik}^2}{(2q_{ik}^2 +1)^{1/2}}  \\
f_{ik} &=& \frac{2}{q_{ik}^2 +1} - \frac{1}{2q_{ik}^2 +1} \\
q_{ik}^2 &=& \frac{\pi}{2} \left( \frac{\alpha_{ik} R_k}{R_i}  \right)^2 \\
\alpha_{ik} &=& {\rm Max} ( \alpha, R_i/R_k ) \label{eq:alpha} \\
\mu_{ik} &=& \frac{ 77 \pi \sqrt{2} R_i }
              {512 ( 1 - 2 \pi^{3/2} \sigma_{ik}^3) \frac{R_i}{\omega_{ik} V_k} } \\
V_k &=& \frac{4}{3}\pi R_k^3 \label{eq:radius}
\end{eqnarray}

\subsubsection{GB/HCT self-energy}
With GB/HCT, the self-energy contribution $E_{ik}^{\rm self}$ is given by
\cite{Hawkins95}
\begin{equation}
4 E_{ik}^{\rm self} = \frac{1}{L_{ik}} - \frac{1}{U_{ik}} 
         + \frac{r_{ik}}{4} \left( \frac{1}{U^2_{ik}} - \frac{1}{L^2_{ik}} \right) 
         + \frac{1}{2r_{ik}} \ln \frac{L_{ik}}{U_{ik}} 
         + \frac{R_k^2}{4r_{ik}} \left( \frac{1}{L^2_{ik}} - \frac{1}{U^2_{ik}} \right),
\end{equation}
where
\begin{eqnarray}
L_{ik} &=& 1         
\hspace*{2.2cm} {\rm if } \hspace*{.5cm} r_{ik} + R_k \leq R_i, \nonumber \\
L_{ik} &=& R_i       
\hspace*{2cm} {\rm if } \hspace*{.5cm} r_{ik} - R_k \leq R_k < r_{ik} + R_k, \nonumber \\
L_{ik} &=& r_{ik} - R_k  
\hspace*{.9cm} {\rm if } \hspace*{.5cm} R_i \leq R_k < r_{ik} - R_k,  \\
U_{ik} &=& 1         
\hspace*{2.2cm} {\rm if } \hspace*{.5cm} r_{ik} + R_k \leq R_i, \nonumber \\
U_{ik} &=& r_{ik} - R_k      
\hspace*{.9cm} {\rm if }   \hspace*{.5cm} R_i < r_{ik} + R_k.
\end{eqnarray}
The corresponding gradient is given by:
\begin{eqnarray}
\frac{4}{r_{ik}} \frac{\partial E_{ik}^{\rm self}}{\partial r_{ik}} &=&
  - \frac{1}{r_{ik}} \left( \frac{L'_{ik}}{L_{ik}^2} - \frac{U'_{ik}}{U_{ik}^2} \right)
  + \frac{1}{4r_{ik}} \left( \frac{1}{U^2_{ik}} - \frac{1}{L^2_{ik}} \right)
  - \frac{1}{2} \left( \frac{U'_{ik}}{U^3_{ik}} - \frac{L'_{ik}}{L^3_{ik}} \right) 
\label{eq:bsolvhct}  \\
  &-& \frac{1}{2r^3_{ik}} \ln \frac{L_{ik}}{U_{ik}} 
  + \frac{1}{2r_{ik}^2} \left( \frac{L'_{ik}}{L_{ik}} - \frac{U'_{ik}}{U_{ik}} \right)
  - \frac{R_k^2}{4r^3_{ik}} \left( \frac{1}{L^2_{ik}} - \frac{1}{U^2_{ik}} \right) 
  - \frac{R_k^2}{2r^2_{ik}} \left( \frac{L'_{ik}}{L^3_{ik}} - \frac{U'_{ik}}{U^3_{ik}} \right)
\nonumber
\end{eqnarray}
with $L'_{ik} = \partial L_{ik} / \partial r_{ik}$, $U'_{ik} = \partial U_{ik} / \partial r_{ik}$.
The radii $R_k$ are calculated from the atomic volumes as in Eq.\ (\ref{eq:radius}),
then reduced by a scaling factor $S_k \leq 1$ which depends only on the chemical type of
atom $k$. Reasonable values are given in Table 1 of \cite{Hawkins95}.

This basic model was modified by Onufriev et al \cite{Onufriev00} to improve performance
for proteins. The self-energy in Eq.\ (\ref{eq:self}) is replaced by:
\begin{eqnarray} 
\Delta E_i^{\rm self} &=& -\frac{\tau q_i^2}{2 b_i} \\
b_i &=& \left[ (R_i - \rho_0)^{-1} 
               - \lambda \sum_{k \neq i} E_{ik}^{self} \right]^{-1} - \delta
 \label{eq:selfnew}
\end{eqnarray}
In other words, the atomic radius $R_i$ is reduced by a constant offset $\rho_0$,
the self-energy contribution $E_{ik}^{\rm self}$ is scaled by a constant factor $\lambda$,
and the solvation radius $b_i$ is reduced by a constant offset $\delta$. The values
$\lambda = 1.4$, $\rho_0 = 0.09$ {\AA} and $\delta$ = 0.15 {\AA} were used in
\cite{Onufriev00}.

\subsection{Pairs of interacting groups}
In structure refinement, it is often necessary to use a model in which
different parts of the macromolecule are artificially duplicated, for example
a protein side chain that is disordered and occupies multiple positions in a
crystal structure. To allow for these situations, both X-PLOR \cite{Xplor} and
CNS \cite{CNS} view the system formally as a set of ``pairs of interacting
groups''. Usually, there is only one such pair: the macromolecule
interacting with itself:
\begin{center} $M$ $\leftrightarrow$ $M$, \end{center}
where $M$ is the macromolecule and $\leftrightarrow$ indicates an interaction. In the
case of a single disordered protein side chain thought to have two main conformations,
one would normally consider a protein $P$ with two copies of the side chain: $S_1$ and $S_2$,
leading to the following pairs of interacting groups:
\begin{center}
$P \, \backslash \{S_1, S_2\}$ $\leftrightarrow$ $P \, \backslash \{S_1, S_2\}$ \\
$P \, \backslash \{S_1, S_2\}$ $\leftrightarrow$ $S_1$; weight of 1/2 \\
$P \, \backslash \{S_1, S_2\}$ $\leftrightarrow$ $S_2$; weight of 1/2,
\end{center}
where $P \, \backslash \{S_1, S_2\}$ represents the protein without the disordered side
chain and the protein--$S$ interactions are weighted by 1/2 because there
are two copies of $S$. The two copies of $S$ do not interact with each other.
This formalism is implemented in X-PLOR through the {\bf constraints interaction}
statement (for an example, see the gbtests/testfirst.inp test case).

The same formalism applies to the GB energy terms. If the interacting groups are denoted
$A_p$, $B_p$ with $p=1,N$, their pairs take the form $P_p = A_p \times B_p = 
\{(i,j); i \in A_p, j \in B_p\}$. There are $N$ pairs of groups $P_p$ and each has a
weight $w_p$. The GB interaction and self energies take the form:
\begin{equation} \label{eq:pig}
\Delta E^{\rm int} = \frac{1}{2} \sum_{p=1}^N w_p \left( \sum_{i \in A_p, j \in B_p}
 \Delta E^{\rm int}_{ij} \right)
\end{equation}
\begin{equation} 
\Delta E^{\rm self} = \sum_{p=1}^N w_p \sum_{i \in A_p, j \in B_p}
\left( -\frac{\tau q_i^2}{2 R_i}\delta_{ij} + \tau q_i^2 E_{ij}^{\rm self} \right)
\end{equation}
These equations generalize Eqs.\ (\ref{eq:gbinte}), (\ref{eq:self}), which correspond
to a single pair $P_1 = M \times M$, $M$ being the whole macromolecule. The factor
$\frac{1}{2}$ in Eq.\ (\ref{eq:pig}) corrects for double counting of $i,j$ and $j,i$
terms; $\delta_{ij}$ is the Kronecker symbol.

\subsection{Crystal symmetry}
The implementation of crystal symmetry is described below and in a published article \cite{Moulinier03}.
The system is now assumed to have $n_G$ symmetry elements, which are isometries of the form
\begin{equation}
S: \und{r} \rightarrow \und{\und{R}} \, \und{r} + \und{\rho}
\end{equation}
$\und{\und{R}}$ is a rotation or an inversion with respect to a plane or a point, and $\und{\rho}$
is a translation vector. The total solvation energy now involves a sum over symmetry images; the solvation
energy $E$ per asymetric unit is
\begin{equation} 
E = \frac{1}{2n_G} \sum_{iS} \sum_{jS'} g(S\und{r}_i, S'\und{r}_j)
  = \frac{1}{2} \sum_{ijS} g(\und{r}_i,S\und{r}_j)
\end{equation}
where $n_G$ is the order of the symmetry group (which is infinite for an infinite crystal). 
In practice, the infinite summation over all crystal translations can be truncated with
a minimum image convention \cite{AllenBK}, since the total electrostatic interaction energy
(Coulomb plus solvation) is rather short-ranged, in contrast to the Coulomb energy alone.

To obtain the solvation forces, we use the relations
\begin{eqnarray}
\nabla_n g(\und{r}_n,S\und{r}_j) &=& 
     g'(\und{r}_n,S\und{r}_j) \frac{\und{r}_n - S \und{r}_j}{| \und{r}_n - S \und{r}_j |} 
\nonumber \\
\nabla_n g(\und{r}_i,S\und{r}_n) &=& 
     R^{-1} g'(\und{r}_i,S\und{r}_n) \frac{\und{r}_i - S \und{r}_n}{| \und{r}_i - S \und{r}_n |} 
\label{eq:jimage} \nonumber \\
\nabla_n g(\und{r}_n,S\und{r}_n) &=& 
     2 g'(\und{r}_n,S\und{r}_n) \frac{\und{r}_n - S \und{r}_n}{| \und{r}_n - S \und{r}_n |}
\nonumber
\label{eq:iimage}
\end{eqnarray}
Here, $g'(\und{r}_i,\und{r}_j)$ represents differentiation of $g_{ij}=g(\und{r}_i,\und{r}_j)$ considered as a
function of the scalar variable $r_{ij} = |\und{r}_i - \und{r}_j|$. The gradient of the solvation energy takes
the form
\begin{eqnarray}
\nabla_i \, E &=& \sum_{i \leq j, S} \lambda_{ij} g'(r_{iJ}) \frac{\und{r}_i - \und{r}_J}{r_{iJ}}
\nonumber \\
 &+& \sum_{j \leq i, S} R^{-1} \lambda_{ij} g'(r_{Ij}) \frac{\und{r}_I - \und{r}_j}{r_{Ij}} 
\end{eqnarray}
The indices $I$, $J$ correspond to the images of the particles $i$, $j$ under $S$.

The energy and forces can be accumulated by summing over the interacting pairs ($i$,$j$) where $i \leq j$
\cite{Verlet67}. While processing the ($i,j$) term, we do two things: 1) we accumulate the contribution of $j$ to
the force on $i$ (`direct' contribution); 2) we calculate and set aside $G_{ij} = \lambda_{ij} R^{-1} g'(r_{iJ})
\frac{\und{r}_i - \und{r}_J}{r_{iJ}}$, which represents the contribution of $i$ to the force on $j$ (`scatter'
contribution). In the vectorized code of CNS or X-PLOR, once the loop over all $j$ is finished, the $G_{ij}$
are `scattered' \cite{AllenBK}, or added to the appropriate atomic forces, $F_j$.

\section{Syntax}
\subsection{GB energy terms}
The GB solvation energy is divided into four terms: direct self-energy and interaction energy terms, and
direct and self-energy terms corresponding to interactions with symmetry images:
\[ E_{GBSOLV} = E_{GBSELF} + E_{GBINT} + E_{PGBS} + E_{PGBI} \]
They are available to the user through the variables \$GBSE, \$GBIN, \$PGBS, and \$PGBI. They are activated
by the {\bf flags} statement in the usual way:
\begin{verbatim}
flags include gbse gbin pgbs pgbi end
\end{verbatim}
They are inactive by default.

\subsection{Setting the GB options}
All the parameters of the GB solvent model are under user control, with
sensible defaults. The setup of the atomic volumes is described further on.
The other GB parameters are set up with the {\bf nbonds} subcommand:
\begin{description}
\item {\bf NBONDs $<$nbonds-statement$>$ $|$ $<$gborn-nbonds-statement$>$ END} \\
applies to electrostatic, van der Waals, and GB energy terms.
\item[$<$gborn-nbonds-statement$>$ :==] \mbox{}
\begin{description}
\item [GBACE $|$ GBHCT] Excusive flags activating the GB/ACE or the GB/HCT model.
Default: inactive.
\item [WEPS=$<$real$>$] Solvent dielectric constant. Default: 1 if GB is inactive,
80 if GB is active. 
\item [SMOOTh=$<$real$>$] Determines the atomic widths in GB/ACE; denoted $\alpha$
in Eq.\ (\ref{eq:alpha}). Default: 1.
\item [LAMBda=$<$real$>$] Scaling factor for solvation radii in GB/HCT; denoted $\lambda$
in Eq.\ (\ref{eq:selfnew}). Default: 1.
\item [OFFSet=$<$real$>$] Offset for atomic radii in GB/HCT; denoted $\rho_0$
in Eq.\ (\ref{eq:selfnew}). Default: 0.
\end{description}
\end{description}

\subsection{Setting up atomic volumes for GB}
Two approaches can be used:

\subsubsection{Volume libraries}
Two sets of `standard' atomic volumes are available for proteins, in two force
field parameter files: {\bf param19.gb.pro} and {\bf paramber.gb.inp}, located
in \$GBXPLOR/gbtoppar (see File Orgaization, below). These volumes are automatically
read along with the other force field parameters. The first set was developed by
Schaefer and coworkers \cite{Schaefer01} and modified and tested for protein simulations
by Calimet et al \cite{Calimet01}, and is meant to be used with the Charmm19 topology
(toph19.pro) and parameter set. The second was developed and tested by our group
\cite{Lopes07} and is meant to be used with the Amber all-atom force field
\cite{Cornell95}. Other volume libraries are available in the literature and can
be formatted for X-PLOR, for example nucleic acid libraries \cite{Tsui00}. 

The syntax of the {\bf NONBonded} subcommand is modified accordingly: \\
{\bf NONB $<$type$>$ $<$real$>$ $<$real$>$ $<$real$>$ $<$real$>$ [$<$real$>$ $<$real$>$]} \\
reads the Lennard-Jones parameters for a specified chemical type, as before;
the first pair of reals is $\epsilon$, $\sigma$; the second pair is $\epsilon$,
$\sigma$ for 1--4 non-bonded interactions. The last two reals are $V$, the atomic
volume (Eqs.\ \ref{eq:self}, \ref{eq:radius}), and $S$, the scaling parameter
used for the HCT solvation radius (see text following Eq.\ \ref{eq:bsolvhct}). If the
last two reals are omitted, $V$ and $S$ will both be set to 9999. Thus, for applications
not using GB, there is backward compatibility with X-PLOR parameter files not set up
for GB. But for applications using GB, $V$ must be included in the parameter file for
both GB/ACE and GB/HCT, and $S$ must be included for GB/HCT. 

\subsubsection{Volumes calculated with an external program}
In some cases, it may be desirable to calculate the atomic volumes corresponding
to a particular family of conformations and/or proteins, instead of relying on
`standard' values \cite{Wagner99}. The standard GB/ACE volumes were obtained from
atomic Voronoi volumes calculated for a large set of protein structures, then averaged
over each chemical
type \cite{Schaefer01}, then reduced by a factor of 0.9 to account for systematic
errors in the GB/ACE self-energy approximation \cite{Calimet01}. Several programs
have the capability to calculate Voronoi volumes for each individual atom of a
particular protein (eg the VORONOI package of Fred Richards). If these are then
stored in a particular field of a PDB coordinate file (for example the field normally
used for the temperature factors, WMAIN), this information can be read into X-PLOR using
the {\bf coordinate} statement, then made available to the GB routines internally.
To do this, the volumes must be copied into the RMSD array, then averaged over
each chemical type using the {\bf parameter reduce} statement: \\
\begin{verbatim}
coor @volumes.pdb                      {read coordinate file with       }
                                       {atomic volumes in wmain field   }

vector do (rmsd = wmain) (all)         {copy into rmsd field            }

flags exclude * include gbse gbin end  {activate GB energy terms, so    }
                                       {GB parameters will be reduced   }

parameter reduce selection=(all)       {average volumes over            }
  overwrite=true mode=average end      {each chemical type              }
end
flags include bonds angl dihe impr vdw elec  {reactivate the other terms}
\end{verbatim}
The atomic volumes, suitably averaged, are then available for GB calculations.

\subsection{Examples}
\subsubsection{Minimization with GB/ACE}
\begin{verbatim}
coordinates  @protein.pdb
parameter
nbonds 
tolerance=0.25 atom cdie trunc 
nbxmod=5 vswitch e14fac=1. cutnb=15. ctonnb=13. ctofnb=14.
eps=1. weps=80. smooth=1.3 gbace   {GB options}
end
end
flags include gbse gbin end
minimize powell nstep=50 end
\end{verbatim}

\subsubsection{Molecular dynamics with GB/HCT}
\begin{verbatim}
remarks   Asparagine MD with GB/HCT
remarks   this file: dyna.inp 

topology  
  @GBXPLOR:gbtoppar/amber/topamber.inp    {Amber topology file       }
  @GBXPLOR:gbtoppar/amber/patches.pro     {N- and C-terminal patches }
end                                       {for Amber force field     }
parameter 
  @GBXPLOR:gbtoppar/amber/paramber.gb.inp {Amber parameter file      }
end                                       {including GB parameters   }
segment
name="ASN1"
molecule name=ASN number=1 end
end
patch NASN refe=nil=(resid 1) end
patch CASN refe=nil=(resid 1) end
parameter 
nbonds 
    atom cdie trunc 
    e14fac=0.8333333                     ! use this to reproduce amber elec
    cutnb 500. ctonnb 480. ctofnb 490.   ! essentially no cutoff
    tolerance=100.                       ! only build the nonbonded list once
    nbxmod 5 vswitch
    wmin=1.0
end 
end
parameters nbonds
    EPS=1. WEPS=80. GBHCT                ! GB parameters
    offset=0.09 lambda=1.33              ! GB parameters
end end
coor @volumes.pdb                        ! PDB with volumes in wmain
vector do (RMSD = wmain) (all)           ! copy into rmsd
vector do (rmsd = rmsd * 0.9) (all)      ! reduce volumes by 10%
flags include gbse gbin end
parameter reduce selection=(all) overwrite=true mode=average end end
coor @asn.pdb

! Now run constant energy dynamics; random initial velocities
vector do (vx = maxwell(250)) (all)
vector do (vy = maxwell(250)) (all)
vector do (vz = maxwell(250)) (all)
dynamics verlet
     nstep=500000 timest=0.001 {ps}     ! 500 ps dynamics
     iasvel=current                     ! current velocities 
     nprint=250 iprfrq=250              ! statistics output
end 
stop
\end{verbatim}

\section{Installation and testing}
\subsection{File organization}
Currently, the GB source files, test files, and documentation are stored
separately from the rest of the X-PLOR distribution, although this is likely
to change in the future. The top of the GB directory tree is named ``gbxplor''.
We assume it is in the topmost directory of an existing X-PLOR distribution,
and  assigned to the environment variable \$GBXPLOR. It contains a Readme file
and the four following subdirectories: 
\begin{itemize}
\item
{\bf gbsource} contains the additional or modified source code for the GB model.
The new or modified source files are:
\bigskip

\begin{tabular}{l l}
File &  Description \\ \hline
gborn.s & The main source module for GB. \\
gborn.fcm & Common blocks for GB. \\
energy.s & Main energy routines; modified to set up and call GB routines. \\
ener.fcm & Common blocks for energy routines; modified to include GB terms. \\
nbonds.s & Set up of nonbonded options; modified to parse and set up GB options. \\ 
nbonds.fcm & Common blocks containing nonbonded and GB options. \\
parmio.s & Force field parameter reader; modified to read and set up GB parameters. \\
param.fcm & Common blocks for force field parameters; modified to include GB parameters. \\
\hline
\end{tabular}
\medskip
\item
{\bf gbtest} contains test files with the necessary data and shell scripts.
\item
{\bf gbtoppar} contains protein parameter files for use with GB/ACE and GB/HCT.
\item
{\bf gbdoc} contains this documentation.
\end{itemize}

\subsection{Installing}
If the files described here were obtained in the form of an archive {\bf gbxplor.tar.gz},
simply unpack it by ``gunzip -c gbxplor,tar.gz $|$ tar xvf -''. {\bf Make a backup copy
of the original X-PLOR source directory}; copy the new or modified GB source code modules
from \$GBXPLOR/gbsource into the main source directory \$SOURCE, and follow the usual
X-PLOR installation procedure. Typically, this means precompiling, then compiling the
new modules, as well as any other modules that make use of ener.fcm, nbonds.fcm, or
param.fcm. 

\subsection{Testing}
Test files are in the directory \$GBXPLOR/gbtest. The tests consist of energy minimization
and dynamics of an asparagine molecule and of the protein thioredoxin. The shell script
\$GBXPLOR/gbtest/runtests.com can be used to execute all the tests (requiring about five
minutes of CPU time on an 800MHz Intel PIII). \$GBXPLOR/gbtest/difftests.com compares the
output to the output files provided with the code.

\chapter{Generalized Born in a protein design context}

\section{Residue GB and the protein design context}
In protein design, we compute interaction energies between residue pairs, to be stored in an energy matrix \cite{Simonson13}.
If the energy function is pairwise additive, each $I,J$ element of the energy matrix will not depend on residues other
than $I$ and $J$. However, in continuum electrostatics, the energy is a many-body function. Thus, the GB interaction between
two atoms $i$, $j$ has the form:
\begin{equation}
E_{ij}^{\rm int} = E_{ij}^{\rm int}({\bf r}_i, {\bf r}_j)
              =\frac{\tau q_i q_j}{(r_{ij}^2+b_ib_j\exp[-r_{ij}^2/4b_ib_j])^{1/2}}
\end{equation}
It depends on the whole set of solute atomic coordinates through the atomic solvation radii,
\begin{equation}
b_i=b_i(\textbf{r}_{1},...,\textbf{r}_{N})
\end{equation}
Thus, an atomic-pairwise decomposition of screening energies is not possible. Nevertheless, Archontis \& Simonson showed how
to modify the GB formulation to employ {\it residue-pairwise} solvation energies \cite{Archontis05b}. 

We define the self-energy contribution of a pair of residues $I$ and $J$:
\begin{equation}
E_{IJ}^{\rm self}=\sum_{i\in I, j\in J} E_{ij}^{\rm self}
\end{equation}
The total self-energy can be written 
\begin{equation}
E^{\rm self}=\sum_I E_I^{\rm self} = \sum_I \sum_J E_{IJ}^{\rm self}
\end{equation}
where $E_I^{\rm self}$ is the self-energy of residue $I$ and each sum runs over all residues.

We now define {\it residue} solvation radii $B_I$, by analogy to the atomic solvation radii above \cite{Archontis05b}:
\begin{equation}
E_I^{\rm self} = \tau \sum_{i \in I}\frac{q_i^2}{2b_i} \stackrel{\rm def}{=} \tau \sum_{i \in I}\frac{q_i^2}{2B_I}
\end{equation}
Thus, $B_I$ is a harmonic average of the atomic solvation radii $b_i$ weighted by the squared charges:
\begin{equation}
\sum_{i \in I} \frac{1}{B_I} = \sum_{i \in I} \frac{q_i^2}{b_i}
\end{equation}

Finally, we consider the GB interaction energy between residues $I, J$. With the usual GB formulation, we would have
\begin{equation}
\Delta E_{IJ}^{\rm int} = -\tau \sum_{i \in I, j \in J}\frac{q_i q_j}{(r_{ij}^2+b_ib_j\exp[-r_{ij}^2/4b_ib_j])^{1/2}}
\end{equation}
Here, however, we introduce a new approximation for $\Delta E_{IJ}^{\rm int}$, where the residue solvation radii replace the
atomic ones:
\begin{equation}
\Delta E_{IJ}^{\rm int} = -\tau \sum_{i \in I, j \in J, i\neq j}\frac{q_i q_j}{(r_{ij}^2+B_IB_J\exp[-r_{ij}^2/4B_IB_J])^{1/2}}
\end{equation}
This new approximation for the GB interaction is referred to as {\bf ``residue GB''} \cite{Archontis05b}. For now, we assume
we are using this method.

For a given structure and set of interatomic distances, the interaction energy $\Delta E_{IJ}^{\rm int}$ is a slowly varying
function of the quantity $B_I B_J$, which we denote $B$. The $B$-dependency can be approximated by fitting
$\Delta E_{IJ}^{\rm int}(B)$ to a generalized polynomial as follows \cite{Archontis05b}:
\begin{equation}
\Delta E_{IJ}^{\rm int}(B) \approx c_1^{IJ} + c_2^{IJ} B + c_3^{IJ} B^2 + c_4^{IJ} B^{-\frac{1}{2}} + c_5^{IJ} B^{-\frac{3}{2}}
\end{equation}
The $c_n$ are constant fitting coefficients. This approximation holds for a large range of $B$ values. The {\it fitting
coefficients} $c_n$ depend only the sets of coordinates $\textbf{r}_i \in I$ and $\textbf{r}_j \in J$: they can be computed
without information on the rest of the solute structure. In this sense, residue GB is residue-pairwise additive.

\section{X-PLOR syntax: the pick gbfit statement}
To characterize the GB interaction between a pair of residues in a protein design context, we can compute the fitting
coefficients $c_n$ introduced above, using the following statement:
\begin{verbatim}
pick gbfit (<selection 1>) (<selection 2>)
\end{verbatim}
The GB interaction between the two selections is computed as a function of $B$ = $B_I B_J$, where $B_I$, $B_J$ are the
residue B values of the two selections. The resulting fitting coefficients are stored in the user variables \$GBFIT1,
\$GBFIT2, ..., \$GBFIT5. Currently, XPLOR performs the fit using a collection of $B$ values uniformly distributed between
1 and 150 \AA. In the future, more options will be added to increase user control over the fit. 

\section{Examples}
The following example shows how to compute the solvation radius $B_I$ for a given residue.
\begin{verbatim}
Remarks Compute the residue B value for a given residue
topology
  @xplor3.9/toppar/amber2xplor/lib/masses_parm99.rtf
  @xplor3.9/toppar/amber2xplor/lib/amino_parm99SB.bbunif.rtf
end
parameters @xplor3.9/toppar/amber2xplor/lib/parm99SB.gbAA.prm end
parameter nbonds
    atom trunc cdie eps=1.0 weps=80.0 e14fac=0.83333333333
    cutnb 500. ctonnb 480. ctofnb 490. nbxmod 5
end end
segment chain sequence SER ALA CYS end end end
coor @data/3pept.pdb
parameter nbonds gbhct bato end end        { activate GB         }
flags exclude * include gbse gbint end
constraints interaction (known) (known) end
vector show (bsolv) (known)                {default values are 1.}
energy end                                 {energy command leads }
vector show (bsolv) (known)                {to updated values    }
vector do (rmsd = charge*charge/bsolv)(resid 1)
vector show sum (rmsd) (resid 1)
eval ($ccob=$result)                       {residue B is the     }
vector do (rmsd=charge*charge) (resid 1)   {harmonic average of  }
vector show sum (rmsd) (resid 1)           {atomic b's, weighted }
eval ($cc=$result)                         {by the atomic charges} 
eval ($bsolv=$cc/$ccob) 
stop
\end{verbatim}

The next example shows how to compute fitting coefficients and write out information for an energy matrix.
\begin{verbatim}
Remarks Fit GB interaction energy between residues I, J as a function of B
topology
  @xplor3.9/toppar/amber2xplor/lib/masses_parm99.rtf
  @xplor3.9/toppar/amber2xplor/lib/amino_parm99SB.bbunif.rtf
end
parameters @xplor3.9/toppar/amber2xplor/lib/parm99SB.gbAA.prm end
parameter nbonds
  atom trunc cdie eps=1.0 weps=80.0 e14fac=0.83333333333
  cutnb 500. ctonnb 480. ctofnb 490. nbxmod 5
end end
segment chain sequence SER ALA CYS end end end
coor @data/3pept.pdb                            {compute and fit     }                                             
pick gbfit (resid 1) (resid 2)                  {GB interaction;     }
display $GBFIT1 $GBFIT2 $GBFIT3 $GBFIT4 $GBFIT5 {display coefficients}
stop                                            {in matrix format    }
\end{verbatim}

Notice that the atomic solvation radii for a specific selection are updated when the GB self energy (\$GBSE) is computed
with the bato parameter:
\begin{verbatim}
parameter nbonds gbhct bato end end
\end{verbatim}

\chapter{Modelling dispersion interactions}

\section{Theory}
The solute-solvent interaction consists of the electrostatic part where the atomic charges in the low dielectric cavity (solute) 
interact with the high dielectric surrounding medium (solvent) as described by the Generalized Born (GB) model; and the 
non electrostatic (non polar) part which describes the cavity formation and the van der Waals solute-solvent interaction.
        
\subsection{Solute-Solvent van der Waals Dispersion Model}
In spirit of the Weeks-Chandler-Andersen (WCA) repulsive/attractive decomposion of the nonpolar contribution to the solvation
free energy \cite{Weeks71} we model the solute-solvent van der Waals dispersion interactions using the attractive part of the
Lennard-Jones potential. 

Following the continuum solute-solvent van der Waals (vdW) energy model of Gallicchio et al. \cite{Levy03} the average vdW
dispersion interaction of atom i with water is given by the integral of the attractive LJ potential between atom i and the
oxygen atom of the water molecule, over the solvent volume, where the water number density  $\rm \rho_w$  is assumed constant.

\begin{equation}
\rm \Delta G^{DI} = \sum_i \bar{U}_i^{vdw} 
\end{equation}

\begin{equation}\label{eq:disp}
\rm \bar{U}^{vdW}_i = -\rho_w \int_{solv} \frac{4\epsilon_{iw}\sigma_{iw}^6}{\left |r-r_i\right |^6} \,d^3r 
\end{equation}
$\epsilon_{iw} $ and $\sigma_{iw}$ are the LJ potential parameters for the solute atom-water oxygen pair.  The total solute-solvent
vdW dispersion interaction is given by the sum of the individual vdW interactions of all atoms of the solute. The integral in the
above equation can be re-written as the difference of the integral over the whole space and the solute region outside the vdW
radius $\rm R_i$. In other words, the solvation free energy of an isolated atom i fully solvated, is reduced by the presence of
all surrounding solute atoms j.    
\begin{equation}\label{eq:disp2}
\rm \bar{U}^{vdW}_i = - 4\epsilon_{ij}\sigma_{ij}^6 \rho_w \big(\frac{4\pi}{3R_i^3} 
    - \int_{r>R_i}^{solu} \frac{1}{\left|r-r_i\right|^6} \,d^3r\big) 
\end{equation}

\begin{equation}\label{eq:uvdw}
\rm \bar{U}^{vdW}_i =  -\frac{f_i}{R_i^3} + f_i (\frac{3}{4\pi}\int_{r>R_i}^{solu} \frac{1}{\left|r-r_i\right|^6} \,d^3r), \quad f_i  
     =  \frac{16\pi}{3}\epsilon_{ij}\sigma_{ij}^6 
\end{equation}
The integral of $\rm1/r^6$ over the solute region is approximated by two contributions as proposed by Onufriev \cite{Aguilar10}
and computed analytically. The main integral  $\rm I_i^{vdW}$ over the atomic vdW spheres and the correction integral
$\rm I_i^{neck}$ over the ``neck" shaped free space regions formed between pairs of vdW spheres.    

\begin{equation}\label{eq:integrals}
\rm  \frac{3}{4\pi}\int_{r>R_i}^{solu} \frac{1}{\left|r-r_i\right|^6} \,d^3r  \approx  I_i^{vdW} + I_i^{neck} 
\end{equation}

\begin{eqnarray}\label{eq:vdwneck}
%\rm I_i^{tot} & = &\rm \frac{3}{4\pi}\int_{r>R_i}^{solu} \frac{1}{\left|r-r_i\right|^6} \,d^3r  
    \approx  I_i^{vdW} + I_i^{neck} \nonumber \\ 
\rm I_i^{vdw} & = & \rm \frac{3}{4\pi}\int_{r>R_i}^{solu/vdw} \frac{1}{\left|r-r_i\right|^6} \,d^3r 
     = \sum_{j\not= i} I_{ij}^{vdw}(r_{ij},R_i,S^{vdw}R_j) \nonumber \\
\rm I_i^{neck} & = & \rm \frac{3}{4\pi}\int_{r>R_i}^{solu/neck} \frac{1}{\left|r-r_i\right|^6} \,d^3r 
    = \frac{3}{4\pi}S^{neck}\sum_{j\not= i} I_{ij}^{neck}(r_{ij},R_i,R_j)  
\end{eqnarray}
Both terms of eq. \ref{eq:integrals} are computed by the sum $\rm \sum_{j\not=i} I_{ij}$ of all atom pairwise interactions
expressed by the following analytical conditional functions: 

\begin{equation}\label{eq:vdwfun}
\rm    I_{ij}^{vdW}(R_i,R_j,r_{ij})= 
\begin{cases}
\rm    \frac{R_j^3}{(r_{ij}^2-R_j^2)^3},& \text{if } \rm r_{ij}\geq R_i + R_j\\
 \rm   \frac{1}{16r_{ij}}\Big(\frac{r_{ij}+3R_j}{(r_{ij}+R_j)^3} + \frac{3(R_j^2-R_i^2-(r_{ij}-R_i)^2)+2r_{ij}R_i}{R_i^4} \Big), 
   & \text{otherwise}
\end{cases}
\end{equation}

\begin{equation}\label{eq:neckfun}
\rm    I_{ij}^{neck}(R_i,R_j,r_{ij})= 
\begin{cases}
\rm   A_{ij}  (r_{ij}-B_{ij})^4(R_i + R_j + 2R_w - r_{ij})^4,& \text{if } \rm B_{ij} < r_{ij} < R_i + R_j + 2R_w \\
 \rm   0,              & \text{otherwise}
\end{cases}
\end{equation}
The neck term parameters A and B themselves depend on the atomic vdW radii of each pair and the water probe radius $\rm R_w$.
Finally the total solute-solvent vdW dispersion interaction is given by 
\begin{equation}
\Delta G^{DI} = \sum_i -\frac{f_i}{R_i^3} + \sum_i \sum_{j\not= i} f_i ( I_{ij}^{vdw}+ \frac{3}{4\pi}S^{neck}I_{ij}^{neck} )
\end{equation}

\subsection{Gaussian Nonpolar Solvent Model}
The Lazaridis and Karplus (LK) model \cite{Lazaridis99} expresses the total solvation free energy of a particular molecular
conformation as a sum over contributions from individual groups of atoms, as follows: 
\begin{eqnarray}\label{eq:lk}
\rm \Delta G^{LK}  & = & \rm \sum_i \Delta G^{solv}_i \nonumber \\
\rm \Delta G^{solv}_i  & = & \rm G_i^{ref} - \sum_{j\not= i} \int_{V_j} f_i(r_{ij}) dV \nonumber \\
 & = & \rm G_i^{ref} - \sum_{j\not= i} f_i(r_{ij}) V_j
\end{eqnarray} 

Each contribution reflects the change in the solvation free energy due to the transfer of  the corresponding group from the
unfolded (fully solvated) to the folded (partially solvated or burried) conformation. This transfer is accompanied by a partial
or total replacement of the surrounding high dielectric solvent by the less polar solute medium, a change in the solvent
orientation around the solute and a modification in the solute-solvent interactions. The solvation energy of a fully solvent
exposed group i is given by an empirically determined reference value $\rm G_i^{ref}$. The same group inside the solute is
screened from solvent by the surrounding groups each contributing to a reduction in the solvation energy of group i. This
reduction is expressed by the integral over the volume of the surrounding solute groups of a gaussian energy density function :
\begin{equation}
\rm f_i(r_{ij}) = \frac{G_i^{free}}{2 \pi^{3/2}\lambda_i r_{ij}^2} e^{-(\frac{r_{ij}-R_i}{\lambda_i})^2}
\end{equation}
which depends on the distance $\rm r_{ij}$, the vdW radius $\rm R_i$, the gaussian correlation length $\rm \lambda_i$ and
$\rm G_i^{free}$. This last parameter is such that, when group i is fully buried the total solvation energy becomes zero.
In the LK model, the integrals are approximated by the product of the density function of group i and the atomic volume of the
surrounding solute group j:
\begin{equation}
\rm \Delta G^{LK}  = \sum_i G_i^{ref} - \sum_i \sum_{j \not=i} f_i(r_{ij}) V_j 
\end{equation}

\section{Implementation}
\textbf{The solute-solvent vdW dispersion interaction} was inserted at the level of non bonded Generalized Born (GB) evaluation
with an individual subroutine. After an energy call the parameters of the model described above are read and stored. That is,
type and atom-based van der Waals radii ($\rm R^{vdw}$) the solvent type (oxygen atom for water) and parameters A, B if the neck
correction term is turned on. In a first stage the van der Waals interaction of each isolated atom of the solute with the continuum
solvent is accumulated to give the `` self " part of the dispersion energy. The factor at the numerator of Eq. \eqref{eq:uvdw}
depends on the water density number and the Lennard-Jones B coefficient for each solute-solvent atom pair. At a second stage,
the interaction of all solute atoms j surrounding each atom i, is evaluated and summed, to give the `` interaction " part which
accounts for the replacement of the surrounding solvent in the isolated atom state by solute. As shown in Eq. \eqref{eq:vdwneck}
van der Waals radii of the surrounding atoms are scaled down by a factor $\rm S^{vdw}$.  The neck term correction of the simplistic
representation of the solute as van der Waals spheres, accounts for omitted space between atomic vdW spheres within the solute
boundary. The total neck contribution is scaled by a factor $\rm S^{neck}$ (Eq. \ref{eq:vdwneck}).  The parameters A and B present
in Eq. \eqref{eq:neckfun} depend on the pair of $\rm R_i^{vdw}, R_j^{vdw}$ and the water probe radius $\rm R_w$. A set of A and B
values has been evaluated on a 2-dimensional equally spaced grid ($\rm R_i^{vdW}\times R_j^{vdW}$) using the numerical method NSR6
developed by Onufriev et. al \cite{Aguilar12}. If the van der Waals radii of a pair do not coincide with a grid node then A and B
parameters are obtained by cubic spline interpolation. The second derivative values of neck parameters are also read from the
parameter file. 

The solute-solvent van der Waals dispersion interaction term is pairwise decomposable, the derivative has analytical form and the
forces are readily obtained by
\begin{eqnarray}
\rm \frac{\partial\, \Delta G_i^{DI}}{\partial\, x_i} & = & \rm \sum_{j \not= i} f_i\frac{\partial \,I^{vdW}_{ij}}{\partial \,x_i} 
    + \frac{3}{4\pi}S^{neck}f_i \frac{\partial \,I^{neck}_{ij}}{\partial \,x_i} \nonumber \\
& = & \rm \sum_{j \not= i} \big \{f_i\frac{\partial \,I^{vdW}_{ij}}{\partial \,r_{ij}} 
  + \frac{3}{4\pi}S^{neck}f_i \frac{\partial \,I^{neck}_{ij}}{\partial \,r_{ij}}\big \} \frac{(x_i-x_j)}{r_{ij}}
\end{eqnarray}
\begin{equation}
\frac{\partial\, I_{ij}^{vdW}}{\partial\, r_{ij}} = 
\begin{cases} 
   \sum_{j\not= i}\frac{-6r_{ij}R_j^3}{(r_{ij}^2-R_j^2)^4},& \text{if } \rm r_{ij}\geq R_i + R_j\\
    \sum_{j\not= i}\frac{-2}{16r_{ij}}\Big(\frac{r_{ij}+4R_j}{(r_{ij}+R_j)^4} + \frac{3r_{ij} -4R_i}{R_i^4}  \Big) 
    - \frac{1}{16r_{ij}^2} I_{ij}^{vdW}(r_{ij}< R_i+R_j),              & \text{otherwise}
\end{cases}
\end{equation}
\begin{equation}
\rm \frac{\partial\, I_{ij}^{neck}}{\partial\, r_{ij}} = 
\begin{cases} 
\rm 4A_{ij}  (r_{ij}-B_{ij})^3(R_i + R_j + 2R_w - r_{ij})^3 (R_i+R_j+2(R_w-r_{ij})+B_{ij}) , \\   
   & \hspace{-5.5cm} \text{if } \rm B_{ij} < r_{ij} <  R_i + R_j + 2R_w  \\
 0,   & \hspace{-2.5cm} \text{otherwise} 
\end{cases}
\end{equation}

\vspace{1cm}
\noindent{\textbf{The Lazaridis-Karplus gaussian nonpolar solvent model}} is put within the non-bonding intramolecular
interactions calculation path with a separate subroutine. At first the type-based parameters of the model $\rm G^{ref}, G^{free}$
and $\lambda$ are being read and stored in arrays. A separate type-based array is needed for use of the model with AMBER,
because it was initially developed for CHARMM and there is a poor correspondence between carbon atoms types CT and CH1E/CH2E/CH3E.
The nonpolar part of the solvation free energy is computed in two steps. First, we compute the sum of atomic solvation free
energies ($\rm G^{ref}$) in their isolated-state completely surrounded by solvent   (first term of Eq. \eqref{eq:lll}); and then
we subtract the sum of the desolvation (replacement of solvent by solute) of each atom from all remaining solute atoms j (second
term of Eq. \eqref{eq:lll}). All atom pairs contribute to the LK model, so 1-2 and 1-3 pairs excluded from van der Waals and
electrostatic interactions, are taken into account. All hydrogen atoms are considered part of the heavy atom they are attached
to, and are excluded from the calculation, as in the initial model. 

\begin{equation}\label{eq:lll}
\rm \Delta G^{LK}  = \sum_i G_i^{ref} - \sum_i \sum_{j > i} (f_i(r_{ij}) V_j + f_j(r_{ij}) V_i)
\end{equation}
When using the constraint interaction command to compute the interaction energy between two selected groups of atoms, the first
term of equation \eqref{eq:lll} is evaluated only for those atoms which belong in both groups. If the two groups do not share any
atoms, only the second term is computed describing the desolvation of the first group from the second and the desolvation of the
second from the first. The command \emph{cons inte ( resid $\rm R_1$ or resid $\rm R_2$ )( resid $\rm R_1$ or resid $\rm R_2$ ) end}
computes the solvation energy of both residues $\rm R_1$ and $\rm R_2$ as follows
\begin{equation}
\rm \Delta G^{LK}(R_1,R_2)  = \sum_{i\in R_1,R_2} G_i^{ref} - \sum_{i\in R_1,R_2} \sum_{i < j\in R_1,R_2} (f_i(r_{ij}) V_j + f_j(r_{ij}) V_i)
\end{equation}
But, the command \emph{cons inte ( resid $\rm R_1$ )( resid $\rm R_1$ or resid $\rm R_2$ ) end} computes the solvation energy of
residue $\rm R_1$ in the presence of residue $\rm R_2$ correctly taking into account the desolvation of $\rm R_1$ from $\rm R_2$
but it also includes the desolvation of residue $\rm R_2$ from residue $\rm R_1$. To eliminate the latter undesirable contribution
we remove $\rm f_j(r_{ij}),\, j\in R_2 $ by setting the atom-based parameters of residue $R_2$ to zero, with the command
\emph{parameters GNSP ( resid $\rm R_2$ ) 0.0 0.0 1.0 end}. Now the solvation energy of residue $\rm R_1$ in the presence of
$\rm R_2$ is given by 
\begin{equation}
\rm \Delta G^{LK}(R_1,R_1-R_2)  = \sum_{i\in R_1} G_i^{ref} - \sum_{i\in R_1} \sum_{i < j\in R_1,R_2} f_i(r_{ij}) V_j 
\end{equation}

The solvation free energy depends on the distance between pairs of atoms ($\rm r_{ij}$), and its derivative has the following
analytical form 
\begin{equation}
\rm \frac{\partial  \,\Delta G^{LK}}{\partial \, x_i} = \sum_{j\not=i}\frac{2}{r_{ij}} \big \{ 
(\frac{1}{r_{ij}} +\frac{r_{ij}-R_i}{\lambda_i^2})f_iV_j + (\frac{1}{r_{ij}}+\frac{r_{ij}-R_j}{\lambda_j^2})f_jV_i \big\} (x_i-x_j)
\end{equation}

\section{Syntax}
\subsection{Solute-solvent van der Waals dispersion energy}
The dispersion term is assigned the variable name \textbf{\$GBDI} and is activated by the flag statement:
\textbf{flags include GBDI end}

\subsubsection{Setting the GBDI options}
\noindent\textbf{NBONDs \textless nbonds-statement\textgreater  \textbar \textless gborn-nbonds-statement\textgreater 
\textbar \textless gbdi-nbonds-statement\textgreater \, END}\\
\textbf{\textless gbdi-nbonds-statement\textgreater :==} 

\textbf{GBDI GBDN} \,\, Flags activating the main GBDI term and the neck contribution. Default: inactive.

\textbf{WTYPE}=\textless string\textgreater \,\,Solvent chemical type.  Default: OW (TIP3P oxygen atom).

\textbf{WRHO}=\textless real\textgreater \,\, Solvent density number. Default: 1.

\textbf{SGBDI}=\textless real\textgreater \,\, $\rm R_j^{vdW}$ scaling factor. Default: 1.

\textbf{SNECK}=\textless real\textgreater \,\,Neck term scaling factor. Default: 1.

\textbf{RWAT}=\textless real\textgreater \,\, Water probe radius. Default: 1.4.

\subsubsection{Setting up the parameters}

The type-based  parameters of the vdW dispersion model are being set with a parameter statement. \\ 

\noindent\textbf{PARAmeter \{\textless parameter-statement\textgreater\} END}\\   
\textbf{\textless parameter-statement\textgreater :==}\\

\noindent{\textbf{DSPN \textless RvdW-statement\textgreater \textless neckAB-statement\textgreater \,END} \\
\textbf{\textless RvdW-statement\textgreater :==} \\
GNOD \\
\textbf{\textless integer\textgreater \, \textless real\textgreater\, ... \textless real\textgreater} \, defines the size of
the grid and assigns a vdW radius foreach node of the grid. \\ 
\textbf{\textless neckAB-statement\textgreater :==} \\
NCKA \\ 
\textbf{\textless real\textgreater\, ... \textless real\textgreater} \,  assigns a value of the neck-A parameter to all nodes
of a row in the ($\rm R_i^{vdW}\times R_j^{vdW}$) grid. \\ 
NCKB \\ 
\textbf{\textless real\textgreater\, ... \textless real\textgreater} \,  assigns a value of the neck-B parameter to all nodes
of a row in the ($\rm R_i^{vdW}\times R_j^{vdW}$) grid. \\
NC2A \\ 
\textbf{\textless real\textgreater\, ... \textless real\textgreater} \,  assigns a value of the second derivative of neck-A
parameter to  all nodes of a row in the ($\rm R_i^{vdW}\times R_j^{vdW}$) grid, used for cubic spline interpolation.\\
NC2B \\ 
\textbf{\textless real\textgreater\, ... \textless real\textgreater }\,  assigns a value of the second derivative of neck-B
parameter to  all nodes of a row in the ($\rm R_i^{vdW}\times R_j^{vdW}$) grid, used for cubic spline interpolation. \\

\subsubsection{Neck parameters calculated with an external program}
The analytical expression of the neck correction (Eq. \ref{eq:vdwneck}) to the $\rm 1/|r|^6$ integral over the solute space
(sum of atomic van der Waals spheres) uses  parameters $\rm A_{ij}\,\, B_{ij}$, which depend on the van der Waals radii of the
atom pair and the water probe radius.    

\subsection{Lazaridis-Karplus interaction energy}
The LK term is assigned the variable name \textbf{\$GNSM} and is activated by the flag statement: \textbf{flags include GNSM end}. 

\subsubsection{Setting GNSM}
\noindent\textbf{NBONDs \textless nbonds-statement\textgreater  \textbar \textless gborn-nbonds-statement\textgreater 
\textbar \textless gbdi-nbonds-statement\textgreater \textbar \textless gnsm-nbonds-statement\textgreater  END} \\ \\
\textbf{\textless gnsm-nbonds-statement\textgreater :==} 

\textbf{GNSM} \,\, Flag activating the GNSM term. Default: inactive.

\subsubsection{Setting up the parameters}
The type- and atom-based  parameters of the LK model are being set with a parameter statement. 

\noindent\textbf{PARAmeter \{\textless parameter-statement\textgreater\} END}\\   
\textbf{\textless parameter-statement\textgreater :==}\\
GNSP \\
\textbf{\textless type\textgreater \, \textless real\textgreater\, \textless real\textgreater\ \textless real\textgreater} 
\, adds $\rm G^{ref}$, $\rm G^{free}$ and $\rm \lambda$ parameters for the atom type to the parameter database. \\ 
\textbf{\textless selection\textgreater \, \textless real\textgreater\, \textless real\textgreater\ \textless real\textgreater} 
\, adds $\rm G^{ref}$, $\rm G^{free}$ and $\rm \lambda$ parameters for the selected atoms to the parameter database. 

\subsection{Examples}
\textbf{Minimization and MD with GB/DI/LK} 
\begin{lstlisting}[numbers=left, numbersep=5pt,numberstyle=\tiny\color{black}]
topology
@xplor3.9/toppar/amber2xplor/lib/masses_parm99.rtf
@xplor3.9/toppar/amber2xplor/lib/amino_parm99SB.bbunif.rtf
@xplor3.9/toppar/amber2xplor/lib/solvents.rtf
@xplor3.9/toppar/amber2xplor/lib/ions.rtf
end

parameters
@parm99SB.plus.prm		      !{plus DI and LK parameters}
end

structure @allh_model.psf end
coordinates @allh_model.pdb

@LK_charmm2amber.str   !{type conversion from charmm19 to amber99}

parameter
   nbonds
    atom trunc cdie eps=1 e14fac=0.83333333333
    ctonnb=97. ctofnb=98. cutnb=99. nbxmod=5 toler=100. 
    gbhct eps=1. weps=80.
    gbdi wtype = OW sgbdi=0.6211 wrho=0.033428  !{GBDI parameters}
    gbdn sneck=0.4058 rwat=1.4  		!{GBDN parameters}
    gnsm 					!{GNSM option}
  end
end

flags include gbse gbin gbdi gnsm end

energy end

minimize powell nstep=50 end 

energy end 

display $gbse $gbin $gbdi $gnsm  

vector do (vx = maxwell(250)) (all)
vector do (vy = maxwell(250)) (all)
vector do (vz = maxwell(250)) (all)

dynamics verlet 
 nstep=1000 timest=0.001 iasvel=current
 nprint=250 iprfrq=250
end

stop

\end{lstlisting}

\chapter{Some data structures for protein design}

\bibliography{/home/simonson/tex/bibtex/current}

\end{document}




%%% Local Variables:
%%% mode: latex
%%% TeX-master: t
%%% End:

\selectlanguage{frenchb}

\pdfbookmark[0]{Resume}{resume}

\section*{Résumé}

{\large\bf\noindent Titre de la thèse}

\bigskip

Le CPD ou \og Computational protein design\fg est la recherche par modèlisation moléculaire des séquences d'acides aminés compatibles avec une structure protéique ciblée.
L'objectif est de concevoir une fonction nouvelle et/ou d'ajouter un nouveau comportement.
Le CPD est en développement au sein de notre laboratoire depuis quelques années, avec le logiciel Proteus qui a plusieurs succès à son actif.
Au cours de cette Thèse, Nous avons enrichi Proteus sur plusieurs points, avec notament l'ajout d'une nouvelle méthode d'exploration de l'espace d'état de type Monte Carlo avec échange de réplique (REMC). Une série de comparaisons entre les algorithmes d'exploration de Proteus ont été effectuées. Ces comparaisons portent sur trois familles de protéines (SH2,SH3 et PDZ) avec dans chaque famille deux ou trois représentants dont la taille varie entre 57 et 109 acides aminés. Le jeu de données constitué a également été utilisé pour optimiser certain paramètre de REMC : distribution des températures, nombre de répliques, fréquences des échanges, etc. Les résultats montrent que le REMC avec un protocole de huit marcheurs et des températures comprises entre 3 et 0,2 est systématiquement meilleur en terme d'énergie ou équivalent que tous les autres protocoles testés (le REMC huit marcheurs avec des températures plus écartées, le MCER avec quatre marcheurs, le Monte-Carlo, l'optimisation itérative sur chaque position). 

  Toujours pour évaluer proteus, nous avons utilisé le programme toulbar2 d'une équipe de l'Université de Toulouse (UMR792) qui propose un algorithme de recherche de type Dead End Elimination (DEE). Cet algorithme trouve le minimum global en éliminant successivement les configurations ne pouvant pas appartenir à ce minimum. Cette méthode fonctionne bien si l'espace de recherche  n'est pas trop grand. Nous avons alors fixé plusieurs acides aminés de nos protéines afin de diminuer la taille de l'espace de recherche. Si on laisse libre entre dix et vingts acides aminés selon les protéines alors toulbar2 donne un résultat en moins de 24 heures de calculs sans demander plus de 8 Go de mémoire vive. Des comparaisons sont en cours pour des espaces allant de zéro à vingts acides aminés libres. Les premiers résultats montrent que proteus trouve systématiquement le minimum global. 

\bigskip

\textbf{Mots-clés :} modélisation moléculaire, conception de protéine par ordinateur, Proteus, Monte Carlo, domaine PDZ

\vfill

\pdfbookmark[0]{Abstract}{abstract}

\selectlanguage{english}

\section*{Abstract}

{\large\bf\noindent Thesis title}

\bigskip

XXX

\bigskip

\textbf{Keywords:} molecular modeling, computational protein design, Proteus, Monte Carlo PDZ domain


%%% Local Variables:
%%% mode: latex
%%% TeX-master: "../these"
%%% End:

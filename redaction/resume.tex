\selectlanguage{frenchb}

\pdfbookmark[0]{Resume}{resume}

\section*{Résumé}

{\large\bf\noindent Computational protein design: un outil pour l'ingénierie des protéines et la biologie synthétique}

\bigskip

Le CPD ou \og Computational protein design\fg est la recherche par modèlisation moléculaire des séquences d'acides aminés compatibles avec une structure protéique ciblée.
L'objectif est de concevoir une fonction nouvelle et/ou d'ajouter un nouveau comportement.
Le CPD est en développement au sein de notre laboratoire depuis quelques années, avec le logiciel Proteus qui a plusieurs succès à son actif.
Au cours de cette Thèse, Nous avons enrichi Proteus sur plusieurs points, avec notament l'ajout d'une méthode d'exploration de type Monte Carlo avec échange de réplique (REMC). Une série de comparaisons entre trois méthodes stochastiques de Proteus ont été effectuées: le REMC, le Monte Carlo (MC) et une heuristique conçue pour le CPD, le Multistart Steepest Descent ou MSD. Ces comparaisons portent sur neuf protéines de trois domaines (SH2, SH3 et PDZ). Nous avons fixé le type de plusieurs acides aminés de nos protéines afin de restraindre l'espace de recherche. Ainsi, grâce à des techniques d'optimization combinatoire, la séquence et la conformation qui minimisent notre fonction d'énergie (le GMEC) est déterminée dans tous les tests avec moins de 10 positions de la chaîne polypeptidique laissées libres, jusqu'à environ dans deux tiers des tests avec 20 positions libres. Globament, le REMC et le MSD donnent de très bonnes séquences en terme d'énergie, avec souvent un accord au GMEC lorsqu'il est connu. Le MSD domine sur les tests à 30 positions actives. Mais le REMC avec huit marcheurs et des paramètres optimisés est plus souvent le meilleur sur les tests tout actif. De plus, comparé à une enumération exacte des séquences sous optimales, le REMC fourni un échantillon de séquences de très bonne diversité.

Dans la seconde partie de ce travail, nous avons paramétré notre modèle pour la conception de domaines PDZ. Note approche du CPD est fondé sur la Physique; notre fonction d'énergie se base sur la différence entre l'état replié de la protéine et son état déplié. Pour l'état replié, nous avons utilisé un modèle de solvant GB/NEA avec une constante diélectrique égale à 8, puis  deux modèle de solvant, le GB/NEA et un nouveau modèle, le GB/FDB avec une constante diélectrique égale à 4. Pour l'état déplié, nous utilisons un ensemble d'énergie d'acide aminés dites énergies de références. Ces énergies de références sont déterminées par une procédure de maximisation de la vraisemblance qui permet de reproduire la composition en acides aminés d'un ensemble d'homologues naturels. Les séquences conçues par Proteus sont comparées aux séquences naturelles. Nos séquences sont globalement similaire aux séquences Pfam, au sens des scores BLOSUM40, avec de très score pour les résidus au coeur de la protéine. Le modèle FDB donne toujours des séquences similaires à des homologues naturels modérément éloignés et l'outil de reconnaissance de pli Superfamily appliquées à ces séquences donne d'excellents résultats. Nos séquences ont également été comparées à celles du logiciel Rosetta. La qualité, selon les mêmes critères que précédement, est très comparable. Mais les séquences de Rosetta restent beaucoup plus proches de la séquence native que celles de Proteus.   


\bigskip

\textbf{Mots-clés :} modélisation moléculaire, conception de protéine par ordinateur, Proteus, Monte Carlo, domaine PDZ

\vfill

\pdfbookmark[0]{Abstract}{abstract}

\selectlanguage{english}

\section*{Abstract}

{\large\bf\noindent Computational protein design: a tool for protein engineering and synthetic biology}

\bigskip

XXX

\bigskip

\textbf{Keywords:} molecular modeling, computational protein design, Proteus, Monte Carlo PDZ domain


%%% Local Variables:
%%% mode: latex
%%% TeX-master: "../these"
%%% End:


\chapter{PDZ}
\label{chap:PDZ}


\section{Introduction}

Nous cherchons maintenant à évaluer performances de notre modèle CPD sur un ensemble de protéines. Les domaines PDZ ("Postynaptic density-95 / Discs large / Zonula occludens-1") sont de petits domaines globulaires qui établissent  des réseaux d'interactions entre protéines dans la cellule(1-6).   Ils forment des interactions spécifiques avec des protéines cibles, généralement en reconnaissant quelques acides aminés à l'extrémité C-terminale. En raison de leur importance biologique, les domaines PDZ et leur interaction avec les protéines cibles ont été largement étudiées et utilisées en conception in sillico. Des ligands ont été conçus  pour moduler l'activité de domaines PDZ impliqués dans diverses pathologie (7-9).des domaines  PDZ des ligands PDZ redessinés ont été utilisés pour élucider les principes du repliement des protéines et de l'évolution (10-13).Et ces domaines avec leurs ligands peptidiques fournissent des "benchmarks" pour tester les méthodes informatiques elles-mêmes (14-16).
\paragraph{}
A partir d'une sélection de domaines PDZ,nous optimisons les énergies de référence E/ref  en utilisant un formalisme du maximum de vraisemblance. La performance du modèle est testée en générant des séquences par proteus pour chaque protéine de la sélection. Pour cela, nous utilisons les résultats précédents, en particulier ceux de la section (12), pour définir les valeurs des paramètres de l'optimisation du Monte-Carlo. Nous confrontons nos résultats à ceux de la fonction d'énergie de Rosetta , fonction, plus empirique que la notre, qui a connue le plus de succès(29-31).    



\paragraph{La fonction Rosetta}

La fonction Rosetta (29-31) comprend un terme de répulsion de Lennard-Jones, un terme Coulomb, un terme de liaison hydrogène,un terme de solvatation Lazaridis-Karplus (32) et des énergies de référence d'état dépliées. Il a un
Un grand nombre de paramètres spécifiquement optimisés pour CPD, qui offrent des performances optimales, mais une interprétation physique moins transparente que Proteus qui lui fournit la capacité de calculer les énergies libres formellement.

\subsection{Description}
La production de nos séquences calculées a été effectuée par des simulations Monte Carlo où toutes les positions de la chaîne polypeptidique ont été autorisées à muter librement, excepté celles occupées par une glycine ou une proline qui conservent leur type d'acide aminé. Ce qui nous procurent des milliers de variantes pour chaque domaine étudié. Nos tests comprennent une validation croisée où les énergies de référence optimisées sur un sous-ensemble de notre sélection sont utilisés sur un entre sous-ensemble de cette même sélection. Nous, réalisons également,une série de simulations Monte Carlo de deux domaines PDZ où le potentiel chimique hydrophobe des types d'acides aminés est progressivement augmenté, polarisant artificiellement la composition de la protéine. Comme Le biais hydrophobe augmente, les acides aminés hydrophobes envahissent progressivement  la protéine
De l'intérieur, formant un noyau hydrophobe devenu plus grand que le naturel.
La propension de chaque position du noyau à devenir hydrophobe à un niveau de biais plus ou moins élevé peut être considéré comme un indice d'hydrophobicité déterminé en fonction de la structure qui nous renseigne sur la propension du coeur à supporter des mutations.

\section{Le modèle d'état déplié}
\subsection{Le maximum de vraisemblance des énergies de référence}

L'énergie utilisée est ici est l'énergie de pliage de la protéine, c'est-à-dire la différence entre
Ses énergies d'état pliées et dépliées (33).Un mouvement élémentaire possible est une "mutation",
Nous modifions le type sidechain t → t0 à une position choisie i dans la protéine pliée, en assignant
Un rotamer r0 particulier à la nouvelle chaîne latérale. Nous considérons la même mutation dans
État déplié. Pour une séquence S particulière, l'énergie d'état dépliée est de la forme:
E^u=\sum_{i\inS}E^r(t_i)
La somme se fait sur tous les acides aminés de S,t_i représente le type  à la position i.

\paragraph{}Les grandeurs E^r(t) \equiv E_t^r sont appelées "énergies de référence".Ils peuvent être considérés comme des potentiels chimiques effectifs de chaque type d'acide aminé. Le changement d'énergie de repliement d'une mutation a donc la forme:
\DeltaE=\DeltaE^f - \DeltaE^u=(E^f(...t'_i,r'_i...) - E^f(...t_i,r_i...)) -(E^r(t'_i) - E^r(t_i))
avec \DeltaE^f and \DeltaE^u les changements d'énergie dans l'état plié et déplié, respectivement.
Les énergies de référence sont des paramètres essentiels dans le modèle de simulation. Notre objectif ici
Est de les choisir empiriquement afin que la simulation produise des fréquences d'acides aminés
Qui correspondent à un ensemble de valeurs cibles, par exemple des valeurs expérimentales dans la base de données Pfam.

Plus précisément, nous les choisirons telles qu'elles maximisent la probabilité des
Séquences cibles. C'est à dire, nous celles qui sont le plus vraisemblable étant donnée l'observation des séquences cibles.

Soit S une séquence particulière. Sa probabilité de Boltzmann est:

p(S)=\frac{1}{Z}exp(-\beta\DeltaG_S),
où \DeltaG_S=G_S^f - E^u_S est l'énergie libre de repliement de S, G^f_S est de l'état replié,\beta =\frac{1}{kT} est la température inverse et Z une constante de normalisation (la fonction de partition). Nous avons alors

kTln p(S) = \sum_{i\inS} E^r(t_i) - G^f_S - kTln Z = \sum_{t\inaa}n_S(t)E^r_t - G^f_S - kTlnZ,

où la somme à droite se fait sur l'ensemble des types d'acides aminés et n_S(t) est le nombre d'acide aminés de type t dans S.

Nous considérons maintenant un ensemble S de N séquences cibles \textit{\textbf{S}}; On appelle \textit{\textbf{L}} la probabilité d'observer l'
nsemble entier. \textit{\textbf{L}} est fonction des paramètres du modèle E_t^r.Comme nous voulons le maximum de \textit{\textbf{L}} sur les E_t^r , nous se réfère à  \textit{\textbf{L}} comme leur vraisemblance.

Nous avons:

kTln \textit{\textbf{L}} = \sum_S\sum_{i\inaa} n_S(t)E^r(t) - \sum_SG^f_S - NkTln Z = \sum_{t\inaa}n_S(t)E^r_t - \sum_SG^f_S - NkTlnZ,

avec N(t) le nombre d'acide aminé de type t dans l'ensemble \textit{\textbf{S}}.Le facteur de normalisation Z (ou fonction de partition) est la somme sur l'ensemble les séquences possibles \textit{\textbf{R}}:

Z=\sum_Sexp(-\beta\DeltaG_\textit{\textbf{R}})=\sum_\textit{\textbf{R}} exp(-\beta\DeltaG^f_\textit{\textbf{R}})\prod_{t\inaa}exp(\betan_\textit{\textbf{R}}(t)E^r_t)
Pour maximiser \textit{\textbf{L}}, nous considerons la dérivé de Z selon chacunes des E_t:

\frac{\partialZ}{\partialE^r_t}= \frac{\sum_\textit{\textbf{R}}n_\textit{\textbf{R}}(t)exp()-\beta\DeltaG_\textit{\textbf{R}}}{\frac_\textit{\textbf{R}}exp(-\beta\DeltaG_\textit{\textbf{R}})} = \langlen(t)\rangle.

La quantité à droite est la moyenne de Boltzmann du nombre n(t) des acides aminés t sur toutes les séquences possibles. En pratique, c'est la population moyenne de t que nous voudrions obtenir dans une longue simulation Monte Carlo. 

Pour que ln \textit{\textbf{L}} soit maximal il faut que ses dérivées par rapport à l'E_t^r soient nulles.

\frac{1}{N}\frac{\partial}{\partialE^r_t}ln \textit{\textbf{L}} = \frac{1}{N}\sum_Sn_S(t) - \langle(t)\rangle = \frac{N(t)}{N} - \langlenn(t)\rangle

et donc
\textit{\textbf{L}} maximum  \Rightarrow \frac{N(t)}{N} = \langlenn(t)\rangle, \forall t \in aa

Ainsi, pour maximiser L, nous devrions choisir {E^r_t} telle qu'une longue simulation donne les mêmes fréquences d'acides aminés que l'ensemble cible.



\subsection{Recherche du maximum de vraisemblance}


Nous utilisons trois méthodes pour approcher les valeurs {E^r_t}.

\begin{enumerate}
\item La première consiste à avancer dans la direction du gradient de ln(\textit{\textbf{L}}) en utilisant la règle itérative suivante (40):
E^r_t(n+1) = E^r_t(n) + \alpha\frac{\partial}{\partialE^r_t}ln(\textit{\textbf{L}})=E^r_t(n) + \deltaE(n^{exp}_t - \langle n(t)\rangle_n)

avec \alpha une constante, n^{exp}_t = \frac{N(t)}{N} la population moyenne d'acide aminé de type t dans l'ensemble ciblé ,
\langle\rangle_n indique une moyenne sur une simulation effectuée en utilisant les énergies de références courantes {E^r_t(n)}, et \deltaE une constante empirique avec la dimension d'une énergie,correspondant à l'amplitude de mise à jour. Cette procédure de mise à jour est répétée jusqu'à convergence. Nous appelons cette méthode, la méthode de mise à jour linéaire.

\item La deuxième méthode est une variante de la première dans laquelle le \deltaE n'est pas constant, mais ajusté au cours de la simulation de la façon suivante. La règle (11) est utilisées trois fois avec trois valeurs differentes et constantes pour le \deltaE ceci avec un jeu d'énergie de références identiques. une interpolation parabolique est efectuée sur les trois valeurs de la fonction proxy obtenues, le minimum de la parabole est calculée et est utilisée comme \deltaE pour le cycle suivant, en terme duquel les énergies sont mises à jours.

\item La troisième méthode, utilisée précédement (26,27), utilise une règle de mise à jour logarithmique:

  avec kT l'énergie thermique, fixée empiriquement à 0,5 kcal/mol. Nous l'appelons la méthode logarithmique. Dans les dernières itérations, certaines valeurs ont tandane à converger lentement,avec des oscillations. Par conséquent, une règle modifié où une énergie au cycle n et l'énergie au cycle  n-1 sont moyenenée avec un poids respectifs de 2/3 et 1/3.

\end{enumerate}
 
Chaque itération, dans la suite, ont étés effectué avec 500 000000 de pas par réplique de REMC.

\section{Méthodes de calcul}
  
\subsection{Fonction énergétique efficace pour l'état replié}

La matrice énergétique a été calculée avec la fonction d'énergie efficace suivante pour
État plié:
E = Ebonds + Eangles + Edihe + Eimpr + Evdw + ECoul + Esolv

Les six premiers termes de l'équation (13) représentent l'énergie interne de la protéine. Ils sont tirés de
La fonction d'énergie empirique Amber ff99SB (42), légèrement modifiée pour le CPD.

Les charges du backbone ont été remplacées par un ensemble unifié, obtenu en faisant la moyenne sur l'ensemble des types  d'acides aminés et ajuster légèrement pour rendre la partie backbone de chaque acide aminé neutre (43).
Le dernier terme à droite de l'équation (13), Esolv, représente la contribution du solvant.Nous avons utilisé un modèle de solvant implicite "Generalized Born + Surface Area" ou GBSA (44):

E_solv = D_GB + E_{surf} = \frac{1}{2}(\frac{1}{\epsilon_W} - \frac{1}{\epsilon_P})\sum_{ij} q_iq_j (r^2_{ij} + b_ib_jexp[-\frac{r^2_{ij}}{4b_ib_j}])^{-\frac{1}{2}} + \sum_i \sigma_iA_i

Ici, \epsilon_W et \epsilon_P sont les constantes diélectriques du solvant et de la protéine; r_{ij} est la distance entre les atomes i, j et b_i est le "rayon de solvatation" de l'atome i (44,45). A_i est la surface exposé accessible au solvant de l'atome i.\sigma_i  

\sigmai est un paramètre qui représente la préférence de chaque atome à être exposé ou caché du solvant. Les atomes du soluté sont divisés en quatre groupes avec pour chacun un \sigmai en cal/mol/Å^2:

\begin{enumarete}
\item non polaire -5   
\item aromatique  -40  
\item polaire     -80  
\item ionique     -100 

\end{enumarate}

ON attribue aux atomes d'hydrogème un coefficient de surface de 0. Les surface sont calculées par l'algorithme de Lee et Richards (46),  qui est implementé dans le programme XPLOR, en utilisant un rayon de ``????'' de 1,5 Å. Les simulations MC utlisent une constante dielectrique \sigmap = 4 ou 8.

Dans le terme énergétique GB, le rayon de solvatation atomique B_i approxime la distance de i à la surface de la protèine et est une fonction des coordonnées de tous les atomes de protéines. La forme B_i correspond à une variante GB que nous appelons GB/HCT, d'après ses auteurs (44),avec les paramètres du modèle optimisées pour une utilisation avec le champ de force Amber (45).Comme b_i dépend des coordonnés de tous les atomes du soluté (44), une approxamiation supplémentaire est nécessaire pour rendre le terme énergitique GB additif par paire et pour rendre la matrice d'énergie définissable (27,28).
Nous utilisons une approximation NEA ("Nativce Environment Approxition"), dans laquelle le rayon de solvatation b_i de chaque groupe (backbone, châine laterral ou ligand) est calculé à l'avance , le reste du système étant fixé à sa séquence et sa conformation native.
La contribution de l'énergie de surface E_{surf} n'est pas non plus additif par pair, car dans la strcuture de la protéine, la surface enfoui par une châine latérale peut également être enfuie par une autre chaîne.Alors, nous avons utilisé la méthoede de Street et al (49). Dans cette méthode, la surface enfuie d'une chaîne latérale est calculée en additionnant la chaîne latérale voisine et les groupes backbones. Pour chaque groupe voisin, la zone de contact avec la chaîne latérale en question est calculé indépendamment des autres groupes. Les zones de contact sont additionnées. Pour éviter de sur-evaluer la surface enfuie, un facteur est appliqué aux zones de contact des chaînes latérales impliquées. Des études précédentes ont montré qu'un facteur de 0.65 fonctionne bien (45,48).  


\subsection{Les énergies de référence de l'état déplié}

Dans le modèle CPD, l'énergie de l'état dépliée dépend de la composition de la séquence par l'ensemble des énergies de référence E^r_t (équation 1). Ici, les énergies de référence ont été attribuées en fonction des types d'acides aminés t, mais aussi de la position de chaque acide aminé dans la structure repliée à travers son caractère enfoui ou exposé au solvant. Ainsi, pour une type donné (Ala, par exemple), il y a deux valeurs distinctes de R^r_t , une enfuie et une exposée. Cette approche se justifie par trois éléments. Tout d'abord, nous supposons que la structure résiduelle est présente dans l' état déplié, de sorte que les acides aminés conservent en partie leur caractère enfui/exposé. Deuxièmement, nous supposons que le modèle d'état déplié compense de manière systématique des erreurs dans la fonction d'énergie de l'état plié, de sorte que  la structure pliée contribue indirectement aux énergies de référence. Troisièmement, cette stratégie rend le modèle moins sensible aux variations de la longueur des boucles de surface et au rassio  de résidus de surface sur  enterrés, qui peut varier considérablement selon les homologues (voir plus bas).  

Par conséquent, le modèle devrait être transférable dans une famille de protéines.Distinguer les positions enterrées / exposées double le nombre de paramètres E^r_t à ajuster.Inversement, pour réduire le nombre de paramètres, nous groupons les acides aminés en classes homologues voir table (t1).Dans chaque class c , et pour chaque type de position (enfoui ou exposé), les énergies de référence ont la forme

E^r_t = E^r_c + \deltaE^r_t

avec E^r_c est un paramètre ajustable,tandis que \deltaR^r_t est une constante, calculée comme la différence d'énergie de mécanique moléculaire entre les types d'acides aminés de classe c,supposé en conformation dépliée où chaque acide aminé interagit uniquement avec lui-même et avec le solvant.


    \begin{table}[!htbp]
      \centering

      \begin{tabular}{ccc}

        \toprule
        Groupe & acides aminés & propriétés\\
        \cmidrule{1-3}

        1   & Ala,Cys,Thr & petit\\
        2   & Ser &\\
        \cmidrule{1-3}
        3   & Glu,Asp & chargé négativement\\
        \cmidrule{1-3}
        4   & Gln,Asn & polaire\\
        \cmidrule{1-3}
        5   & Ile,Leu,Val & apolaire\\
        \cmidrule{1-3}
        6   & Met & non polaire\\
        \cmidrule{1-3}
        7   & Hip,Hid,Hie & chargé positivement\\
        8   & Arg \\
        9   & Lys \\
        \cmidrule{1-3}
        10  & Phe,Trp & aromatique\\
        11  & Tyr \\
        \cmidrule{1-3}
        12  & Gly,Pro & non mutable\\
        \bottomrule


      \end{tabular}      
      \caption{Les groupes d'acides aminés utilisés pour l'optimisation des énergies de référence.}
\label{tab:AA_groupes}      
    \end{table}


Plus précisément, nous effectuons des simulations MC d'un peptide étendu (le peptide Syndecan1; voir
Ci-dessous) et calculons les énergies moyennes pour chaque type d'acide aminé à chaque position peptidique (à l'exclusion des positions terminales). Nous prenons les différences entre les types d'acides aminés et les moyennons sur les positions peptidiques.

Pendant, la maximisation de la vraisemblance, E_c est optimisé tandis que \deltaE_t est fixe. Pour optimiser les valeurs E^r_t, nous utilisons une trois méthodes (\ll) avec des fréquences cibles correspondent aux fréquences expérimentales soient des classes d'acides aminés, soient des types d'acides aminés. Le choix se faisant en générale par un début d'optimisation sur les classes, puis lorsque la convergence est correctement établie sur ces classe, nous relâchons la contraintes des classes pour optimiser sur les types.  


\section{Séquences expérimentales et modèles structurels}


\section{L'ensemble des protéines PDZ}

Nous sélectionnons huit protéines de la famille PDZ dont les structures cristallines sont connues, avec les trois présentes dans l'ensemble étudié au chapitre précédent: 1G9O,1R6J et 2BYG, aux quelles sont ajoutés 1IHJ, 1N7E, 3K82 et Cask , Tiam1 représenté par .... et ... dans la base de données PDB. Cela constitue un ensemble où le nombre de positions actives , c'est à dire les postions qui vont être muté , est du même ordre pour chaque séquence d'acide aminé des protéines ( voir le tableau \ref{tab:protéines_PDZ}).



    \begin{table}[!htbp]
      \centering

      \begin{tabular}{ccc}

        \toprule
        Code PDB & résidus & nombre de positions actives\\
        \cmidrule{1-3}
        1G9O  & 	9-99	 & 	76	 \\
        1IHJ  & 	13-105	 & 	82	 \\
        1N7E  & 	668-761	 & 	79	 \\
        1R6J  & 	193-273	 & 	72	 \\
        2BYG  & 	186-282	 & 	82	 \\
        3K82  & 	305-402	 & 	80	 \\
        Cask  & 	487-568	 & 	74	 \\
        Tiam1 & 	838-930	 & 	84	 \\
        \bottomrule

      \end{tabular}      
      \caption{La sélection de domaines protéiques PDZ}
\label{tab:protéines_PDZ}      
    \end{table}




\paragraph{Alignements Blast croisés}
Pour caractériser les homologies dans cet ensemble, une série de requête blast est effectuée sur chaque paire de séquences en utilisant le programme blastp avec les options comme indiqué en (45).Il apparaît que 1R6J et TiAM1 sont atypiques dans l'ensemble avec, aucun homologue avec une E-value inférieure à 1e-7 et plusieurs E-value supérieur à 10. 3K82 est la protéine plus consensuelle , ayant d'une part une homologie avec toutes les autres à au plus 6e-04 , et d'autre part ayant 4 homologues à moins de 2e-10, pour pourcentage d'identité compris entre 30 et 46.Globalement, il n'y a que peu d'homologies, la plus forte n'étant que de 3e-15 entre 3K82 et 2BYG pour un pourcentage d'identité de 37. Les détails sont dans le tableau \ref{tab:Xblast}.


    \begin{table}[!htbp]
        \raggedleft{}
\scalebox{0.75}{
      \begin{tabular}{ccccccccc}

        \toprule
        Protein & 1G9O        & 1IHJ      & 1N7E        & 1R6J        & 2BYG       & 3K82       & CASK       & TIAM1 \\
        \cmidrule{1-9}

        1G9O    & 2e-66 (100)& 5e-10 (40) & 0.002 (25)  & 3e-07 (25)  & 2e-11 (35) & 1e-12 (30) & 5e-05 (25) & 9e-07(35)\\     
        1IHJ    & 5e-10 (40) & 3e-68 (100)&  2e-07 (27) & [18]        & 2e-08 (27) & 9e-14 (46) & 4e-06 (35) & [16]    \\
        1N7E    & 0.002 (25) & 2e-07 (27) &  3e-67 (100)& [21]        & 3e-14 (36) & 2e-10 (37) & 9e-12 (30) & 5e-05 (35)\\
        1R6J    & 3e-07 (25) & [18]       &    [21]     & 1e-59 (100) &  [17]      & 1e-06 (32) & 0.007 (32) & [18]    \\
        2BYG    & 2e-11 (35) & 2e-08 (27) &  3e-14 (37) & [17]        & 7e-71 (100)& 3e-15 (37) & 2e-07 (28) & 5e-05 (41)\\
        3K82    & 1e-12 (30) & 9e-14 (46) &  2e-10 (36) & 1e-06 (32)  & 3e-15 (37) & 4e-70 (100)& 1e-07 (27) & 6e-04(33) \\
        Cask    & 5e-05 (25) & 4e-06 (35) &  9e-12 (30) & 0.007 (32)  & 2e-07 (28) & 1e-07 (27) & 7e-61 (100)& 5e-04(33) \\
        Tiam1   & 9e-07 (35) &  [16]      &  5e-05 (35) & [18]        & 5e-05 (41) & 6e-04 (33) & 5e-04 (33) & 1e-68 (100)\\
        \bottomrule

      \end{tabular} 
}     
      \caption{E-value et pourcentage d'identité des alignements Blast native versus native pour nos séquences PDZ.S'il n'y a pas de touche avec une E-value inférieure à 10,[] donne le pourcentage d'identité du couple dans l'alignement des 6 séquences sauvages.}
\label{tab:Xblast}      
    \end{table}


\paragraph{similarité des homologues}


Pour définir les fréquences d'acide aminés cibles pour maximiser nos vraisemblances,nous sélectionnons un ensemble de séquences homologues pour chacunes de nos 8 protéines. Pour cela, nous effectuons des recherches blast avec comme requête la séquence extraites du fichier PDB sur la base de données ``siwwprot + trEmBL'' d'Uniprot avec la matrice BLOSUM62 sans l'option \og filtre\fg et avec l'option \og Gapped\fg. Nous obtenons un premier ensemble pour chaque cas en se limitant aux homologues de bonnes qualité au regard de E-value et du pourcentage d'identité, tout en conservant  en même temps une certaine diversité.Cela oblige pour certaines protéine à accepter des E-values plus haute que  1e-40, notamment 1IHJ et 1G9O , respectivement 1e-32 et 1e-10 ,pour avoir un nombre d'homologue suffisant. Ensuite, les redondances les plus flagrantes sont enlevées manuellement. Finalement,les ensembles se composent de 42 à 126 homologues,avec des pourcentages d'identité supérieurs à 66 \% excepté pour 1IHJ où il a fallut descendre jusqu'à 38\% d'identité.Voir le tableau pour les détails \ref{tab:select_homo}.


    \begin{table}[!htbp]
      \centering

      \begin{tabular}{cccc}

        \toprule
        protéines & \% identité \\
        \cmidrule{1-4}
     1G9O  & 62  &    1e-32  &  67-95 \\
     1IHJ  & 42  &    1e-10  &  38-95 \\
     1N7E  & 48  &    1e-45  &  84-95 \\
     1R6J  & 85  &    1e-43  &  85-95 \\
     2BYG  & 43  &    1e-41  &  78-95 \\
     3K82  & 50  &    1e-46  &  81-95 \\
     Cask  & 126 &    7e-28  &  60-85 \\
     Tiam1 & 50  &    2e-23  &  60-85 \\

        \bottomrule

      \end{tabular}      
      \caption{Sélection des homologues.}
\label{tab:select_homo}      
    \end{table}



    Pour chaque  ensemble d'homologues, notons le H, nous calculons la moyenne sur toutes les séquences et toues les prositions pour obtenir des fréquences globales d'acides aminés. Les fréquences sont déterminées séparément pour les positions enfouies et exposées.Notons les {f^b_t(H),f^e_t(H)} , où l'indice t représente un type d'acide aminé et les exposants e et b  représentent respectivement aux positions enfuie et exposées. Enfin les ensembles de fréquences moyennes des huit protéines sont eux-mêmes moyennés , ce qui donne deux ensembles cibles distincts de fréquences d'acides aminés f^b_t et f^e_t pour chaque type, et de même pour chaque classe de type.
    
\paragraph{}

Pour réaliser les calculs Monte Carlo, les structures ont été préparés et les matrices d' énergie calculées à l'aide d'une procédure décrites précédement (15,50).Deux sgments manquants dans le domaine Tiam1 (résidus 851-854 et 868-869) ont été construits en utilisant le programme Modeller (51). Le ligand peptidique a été retiré de la structure PDB avant de calculer la matrice d'énergie.Pour chaque paire d'acide aminés, l'énergie d'interaction a été obtenue après 15 pas de minimisation de l'énergie, avec le backbone fixé et seulement les interactions de la paire entre les autres chaînes et le backbone. Cette courte minimisation simplifie l 'approximation discret. Les rotamères de chaînes latérales utilisés sont une version légérement étendue de la librairie de Tuffery et cal (52), qui poséde un total de 254 rotamères (sur l'ensemble des types d'acides aminés).Cette extension comprends des orientations d'hydrogème supplémentaires pour les groupes OH et SH (48). Cette bibliothèque de rotamères a été choisié pour sa simplicité et parce qu'elle a donné de très bonnes performances dans les tests de placement de chaînes en comparaison au programme spécialisé scwlr4 qui utilise une bibliothèque beaucoup plus grande (53,54).


    
    \begin{table}[!htbp]
      \centering

      \begin{tabular}{ccccccccc}

        \toprule
        aa & 1G9O & 1IHJ & 1N7E & 1R6J & 2BYG & 3K82 & cask & tiam1 \\
        \cmidrule{1-9}
     ALA  & 5.8  & 10.4 & 14.5  & 8.3  &  12.1 &  4.6 &    4.6   &     7.1  \\
     CYS  & 3.0  & 1.6  & 0.1   & 2.6  &  0.2  &  0.4 &    3.0   &     0.0  \\
     THR  & 3.0  & 1.4  & 6.1   &  8.9  &  6.7  &  2.5 &    4.4   &     3.0  \\
        \cmidrule{1-9}
     SER  & 4.2  & 8.5  & 3.2   & 7.2  &  1.8  &  7.1 &    4.4   &     4.8  \\
        \cmidrule{1-9}
     GLU  & 7.4  & 1.4  & 0.0   & 0.3  &  0.1  &  6.3 &    6.3   &     5.9  \\
     ASP  & 6.3  & 3.1  & 5.9   & 0.2  &  8.0  &  2.4 &    3.9   &     3.0  \\
        \cmidrule{1-9}
     ASN  & 2.9   & 0.3  & 2.9  & 3.3  &  3.9  &  2.4 &    0.7   &     2.9  \\
     GLN  & 3.2  & 3.0  & 0.0   & 0.7  &  1.1  &  4.7 &    1.4   &     0.1  \\
        \cmidrule{1-9}
     ILE  & 7.0  & 22.1 & 23.4  & 17.0 &  13.3 &  3.3 &    19.7  &     11.6  \\
     VAL  & 25.8 & 16.4 & 7.9   & 18.8 &  18.6 &  1.8 &    13.8  &     13.1  \\
     LEU  & 17.2 & 13.6 & 29.9  & 14.6 &  18.8 &  5.3 &    15.1  &     25.5  \\
        \cmidrule{1-9}
     MET  & 1.2  & 0.8  & 0.1   & 2.6  &  0.0  &  0.6 &    8.4   &     1.5  \\
        \cmidrule{1-9}
     HID  & 0.0  & 0.0  & 0.0   & 0.0  &  0.0  &  0.0 &    0.0   &     0.0  \\
     HIE  & 0.0  & 0.0  & 0.0   & 0.0  &  0.0  &  0.0 &    0.0   &     0.0  \\
     HIP  & 0.0  & 0.7  & 0.0   & 3.4  &  2.6  &  0.3 &    1.2   &     0.1  \\
        \cmidrule{1-9}
     ARG  & 2.8  & 6.5  & 2.9   & 0.3  &  0.2  &  4.4 &    0.6   &     2.9  \\
        \cmidrule{1-9}
     LYS  & 0.1  & 1.8  & 0.2   & 5.7  &  4.4  &  2.7 &    7.1   &     5.8  \\
        \cmidrule{1-9}
     TRP  & 0.1 & 0.0  & 0.0    & 0.0  &  0.0  &  0.1 &    0.0   &     0.0  \\
        \cmidrule{1-9}
     PHE  & 6.4  & 6.8  & 0.2  & 4.2  &  3.08 &  4.3 &    3.9   &     5.9  \\
        \cmidrule{1-9}
     TYR  & 3.4  & 0.3  & 2.8  & 1.1  &  2.6  &  5.4 &    0.0   &     5.7  \\
        \cmidrule{1-9}
     GLY  & 0.1  & 0.9  & 0.0  & 0.5  &  2.3  &  1.0 &    0.0   &     0.0  \\
     PRO  & 0.1  & 0.3  & 0.0  & 0.0  &  0.1  &  0.2 &    0.6   &     0.0  \\

        \bottomrule


      \end{tabular}      
      \caption{Compositions en acides aminés des séquences expérimentales homologues aux positions enfouies et actives. pour les 8 protéines.}
\label{tab:freq_AA_ALL}      
    \end{table}


\subsection{simulation Monte Carlo}    

La conception des séquence est réalisé avec Proteus,

D'abord,pour optimiser les énergies de référence, nous faisons des simulations où environ la moitié des position peuvent muter à la fois.
Ensuite,les modèles optimisé ont été testés avec des simulations où tous les positions sauf sauf celles nativement en Gly ou Pro, sont libre de muter. C'est équalement de cas pour les simulations de la titration hydrophobe.

Dans les deux cas des mutations se sont produites au hasard, soumises uniquement à la fonction d'énergie MMGBSA
Qui entraîne la simulation. Les simulations Monte Carlo utilisent des mouvements à une ou deux positions,où les rotamères,les types d'acides aminés ou les deux peuvent changer.Pour les mouvements à deux positions, la deuxième position est choisi parmi celles qui avaient une énergie d'interaction significative avec la première (c'est à dire 10 kcal/mol ou plus). De plus, l'échantillonnage est amélioré par l'échange de réplique (REMC), où plusieurs simulations MC sont exécutées en parallèle, à différenres températures. Des échanges périodiques



\subsection{Génération de séquence Rosetta}





%%% Local Variables:
%%% mode: latex
%%% TeX-master: "../these"
%%% End:

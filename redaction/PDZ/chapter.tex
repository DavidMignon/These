
\chapter{PDZ}
\label{chap:PDZ}


\section{Introduction}

Nous cherchons maintenant à évaluer performances de notre modèle CPD sur un ensemble de protéines. Les domaines PDZ ("Postynaptic density-95 / Discs large / Zonula occludens-1") sont de petits domaines globulaires qui établissent  des réseaux d'interactions entre protéines dans la cellule(1-6).   Ils forment des interactions spécifiques avec des protéines cibles, généralement en reconnaissant quelques acides aminés à l'extrémité C-terminale. En raison de leur importance biologique, les domaines PDZ et leur interaction avec les protéines cibles ont été largement étudiées et utilisées en conception in sillico. Des ligands ont été conçus  pour moduler l'activité de domaines PDZ impliqués dans diverses pathologie (7-9).des domaines  PDZ des ligands PDZ redessinés ont été utilisés pour élucider les principes du repliement des protéines et de l'évolution (10-13).Et ces domaines avec leurs ligands peptidiques fournissent des "benchmarks" pour tester les méthodes informatiques elles-mêmes (14-16).
\paragraph{}
A partir d'une sélection de domaines PDZ,nous optimisons les énergies de référence E/ref  en utilisant un formalisme du maximum de vraisemblance. La performance du modèle est testée en générant des séquences par proteus pour chaque protéine de la sélection. Pour cela, nous utilisons les résultats précédents, en particulier ceux de la section (12), pour définir les valeurs des paramètres de l'optimisation du Monte-Carlo. Nous confrontons nos résultats à ceux de la fonction d'énergie de Rosetta , fonction, plus empirique que la notre, qui a connue le plus de succès(29-31).    


\section{L'ensemble des protéines PDZ}
Nous sélectionnons huit protéines de la famille PDZ, avec les trois présentes dans l'ensemble étudié au chapitre précédent: 1G9O,1R6J et 2BYG, aux quelles sont ajoutés 1IHJ, 1N7E, 3K82 et Cask , tiam1 représenté par .... et ... dans la base de données PDB. Cela constitue un ensemble où le nombre de positions actives , c'est à dire les postions qui vont être muté , est du même ordre pour chaque séquence d'acide aminé des protéines ( voir le tableau \ref{tab:protéines_PDZ}).



    \begin{table}[!htbp]
      \centering

      \begin{tabular}{ccc}

        \toprule
        Code PDB & résidus & nombre de positions actives\\
        \cmidrule{1-3}
        1G9O  & 	9-99	 & 	76	 \\
        1IHJ  & 	13-105	 & 	82	 \\
        1N7E  & 	668-761	 & 	79	 \\
        1R6J  & 	193-273	 & 	72	 \\
        2BYG  & 	186-282	 & 	82	 \\
        3K82  & 	305-402	 & 	80	 \\
        Cask  & 	487-568	 & 	74	 \\
        Tiam1 & 	838-930	 & 	84	 \\
        \bottomrule

      \end{tabular}      
      \caption{La sélection de domaines protéiques PDZ}
\label{tab:protéines_PDZ}      
    \end{table}




\paragraph{Alignements Blast croisés}
Pour caractériser les homologies dans cet ensemble, une série de requête blast est effectuée sur chaque paire de séquences en utilisant le programme blastp avec les options comme indiqué en (45).Il apparaît que 1R6J et TiAM1 sont atypiques dans l'ensemble avec, aucun homologue avec une E-value inférieure à 1e-7 et plusieurs E-value supérieur à 10. 3K82 est la protéine plus consensuelle , ayant d'une part une homologie avec toutes les autres à au plus 6e-04 , et d'autre part ayant 4 homologues à moins de 2e-10, pour pourcentage d'identité compris entre 30 et 46.Globalement, il n'y a que peu d'homologies, la plus forte n'étant que de 3e-15 entre 3K82 et 2BYG pour un pourcentage d'identité de 37. Les détails sont dans le tableau \ref{tab:Xblast}.


    \begin{table}[!htbp]
        \raggedleft{}
\scalebox{0.75}{
      \begin{tabular}{ccccccccc}

        \toprule
        Protein & 1G9O        & 1IHJ      & 1N7E        & 1R6J        & 2BYG       & 3K82       & CASK       & TIAM1 \\
        \cmidrule{1-9}

        1G9O    & 2e-66 (100)& 5e-10 (40) & 0.002 (25)  & 3e-07 (25)  & 2e-11 (35) & 1e-12 (30) & 5e-05 (25) & 9e-07(35)\\     
        1IHJ    & 5e-10 (40) & 3e-68 (100)&  2e-07 (27) & [18]        & 2e-08 (27) & 9e-14 (46) & 4e-06 (35) & [16]    \\
        1N7E    & 0.002 (25) & 2e-07 (27) &  3e-67 (100)& [21]        & 3e-14 (36) & 2e-10 (37) & 9e-12 (30) & 5e-05 (35)\\
        1R6J    & 3e-07 (25) & [18]       &    [21]     & 1e-59 (100) &  [17]      & 1e-06 (32) & 0.007 (32) & [18]    \\
        2BYG    & 2e-11 (35) & 2e-08 (27) &  3e-14 (37) & [17]        & 7e-71 (100)& 3e-15 (37) & 2e-07 (28) & 5e-05 (41)\\
        3K82    & 1e-12 (30) & 9e-14 (46) &  2e-10 (36) & 1e-06 (32)  & 3e-15 (37) & 4e-70 (100)& 1e-07 (27) & 6e-04(33) \\
        Cask    & 5e-05 (25) & 4e-06 (35) &  9e-12 (30) & 0.007 (32)  & 2e-07 (28) & 1e-07 (27) & 7e-61 (100)& 5e-04(33) \\
        Tiam1   & 9e-07 (35) &  [16]      &  5e-05 (35) & [18]        & 5e-05 (41) & 6e-04 (33) & 5e-04 (33) & 1e-68 (100)\\
        \bottomrule

      \end{tabular} 
}     
      \caption{E-value et pourcentage d'identité des alignements Blast native versus native pour nos séquences PDZ.S'il n'y a pas de touche avec une E-value inférieure à 10,[] donne le pourcentage d'identité du couple dans l'alignement des 6 séquences sauvages.}
\label{tab:Xblast}      
    \end{table}


\paragraph{similarité des homologues}


Pour avoir une composition naturelle en acide aminé de nos protéines,nous sélectionnons un ensemble de séquences homologues pour chacunes de nos 8 protéines.Pour cela, sur le site Web d'Uniprot (http://www.uniprot.org/ ), nous effectuons des requêtes sur la base de données ``siwwprot + trEmBL'' avec la matrice BLOSUM62 sans l'option \og filtre\fg et avec l'option \og Gapped\fg. Nous obtenons un premier ensemble pour chaque cas en se limitant aux homologues de bonnes qualité au regard de E-value et du pourcentage d'identité , tout en conservant  en même temps une certaine diversité.Cela oblige pour certaines protéine à accepter des E-values plus haute que  1e-40, notamment 1IHJ et 1G9O , respectivement 1e-32 et 1e-10 ,pour avoir un nombre d'homologue suffisant.Ensuite, les redondances les plus flagrantes sont enlevées manuellement.Finalement,les ensembles se composent de 42 à 126 homologues,avec des pourcentages d'identité supérieurs à 66 \% excepté pour 1IHJ où il a fallut descendre jusqu'à 38\% d'identité.Voir le tableau pour les détails \ref{tab:select_homo}.


    \begin{table}[!htbp]
      \centering

      \begin{tabular}{cccc}

        \toprule
        protéines & \% identité \\
        \cmidrule{1-4}
     1G9O  & 62  &    1e-32  &  67-95 \\
     1IHJ  & 42  &    1e-10  &  38-95 \\
     1N7E  & 48  &    1e-45  &  84-95 \\
     1R6J  & 85  &    1e-43  &  85-95 \\
     2BYG  & 43  &    1e-41  &  78-95 \\
     3K82  & 50  &    1e-46  &  81-95 \\
     Cask  & 126 &    7e-28  &  60-85 \\
     Tiam1 & 50  &    2e-23  &  60-85 \\

        \bottomrule

      \end{tabular}      
      \caption{Sélection des homologues.}
\label{tab:select_homo}      
    \end{table}


\paragraph{La fonction Rosetta}

La fonction Rosetta (29-31) comprend un terme de répulsion de Lennard-Jones, un terme Coulomb, un terme de liaison hydrogène,un terme de solvatation Lazaridis-Karplus (32) et des énergies de référence d'état dépliées. Il a un
Un grand nombre de paramètres spécifiquement optimisés pour CPD, qui offrent des performances optimales, mais une interprétation physique moins transparente que Proteus qui lui fournit la capacité de calculer les énergies libres formellement.

\subsection{Description}
La production de nos séquences calculées a été effectuée par des simulations Monte Carlo où toutes les positions de la chaîne polypeptidique ont été autorisées à muter librement, excepté celles occupées par une glycine ou une proline qui conservent leur type d'acide aminé. Ce qui nous procurent des milliers de variantes pour chaque domaine étudié. Nos tests comprennent une validation croisée où les énergies de référence optimisées sur un sous-ensemble de notre sélection sont utilisés sur un entre sous-ensemble de cette même sélection. Nous, réalisons également,une série de simulations Monte Carlo de deux domaines PDZ où le potentiel chimique hydrophobe des types d'acides aminés est progressivement augmenté, polarisant artificiellement la composition de la protéine. Comme Le biais hydrophobe augmente, les acides aminés hydrophobes envahissent progressivement  la protéine
De l'intérieur, formant un noyau hydrophobe devenu plus grand que le naturel.
La propension de chaque position du noyau à devenir hydrophobe à un niveau de biais plus ou moins élevé peut être considéré comme un indice d'hydrophobicité déterminé en fonction de la structure qui nous renseigne sur la propension du coeur à supporter des mutations.

\section{Le modèle d'état déplié}
\subsection{Le maximum de vraisemblance des énergies de référence}

L'énergie utilisée est ici est l'énergie de pliage de la protéine, c'est-à-dire la différence entre
Ses énergies d'état pliées et dépliées (33).Un mouvement élémentaire possible est une "mutation",
Nous modifions le type sidechain t → t0 à une position choisie i dans la protéine pliée, en assignant
Un rotamer r0 particulier à la nouvelle chaîne latérale. Nous considérons la même mutation dans
État déplié. Pour une séquence S particulière, l'énergie d'état dépliée est de la forme:
E^u=\sum_{i\inS}E^r(t_i)
La somme se fait sur tous les acides aminés de S,t_i représente le type  à la position i.

\paragraph{}Les grandeurs E^r(t) \equiv E_t^r sont appelées "énergies de référence".Ils peuvent être considérés comme des potentiels chimiques effectifs de chaque type d'acide aminé. Le changement d'énergie de repliement d'une mutation a donc la forme:
\DeltaE=\DeltaE^f - \DeltaE^u=(E^f(...t'_i,r'_i...) - E^f(...t_i,r_i...)) -(E^r(t'_i) - E^r(t_i))
avec \DeltaE^f and \DeltaE^u les changements d'énergie dans l'état plié et déplié, respectivement.
Les énergies de référence sont des paramètres essentiels dans le modèle de simulation. Notre objectif ici
Est de les choisir empiriquement afin que la simulation produise des fréquences d'acides aminés
Qui correspondent à un ensemble de valeurs cibles, par exemple des valeurs expérimentales dans la base de données Pfam.

Plus précisément, nous les choisirons telles qu'elles maximisent la probabilité des
Séquences cibles. C'est à dire, nous celles qui sont le plus vraisemblable étant donnée l'observation des séquences cibles.

Soit S une séquence particulière. Sa probabilité de Boltzmann est:

p(S)=\frac{1}{Z}exp(-\beta\DeltaG_S),
où \DeltaG_S=G_S^f - E^u_S est l'énergie libre de repliement de S, G^f_S est de l'état replié,\beta =\frac{1}{kT} est la température inverse et Z une constante de normalisation (la fonction de partition). Nous avons alors

kTln p(S) = \sum_{i\inS} E^r(t_i) - G^f_S - kTln Z = \sum_{t\inaa}n_S(t)E^r_t - G^f_S - kTlnZ,

où la somme à droite se fait sur l'ensemble des types d'acides aminés et n_S(t) est le nombre d'acide aminés de type t dans S.

Nous considérons maintenant un ensemble S de N séquences cibles \textit{\textbf{S}}; On appelle \textit{\textbf{L}} la probabilité d'observer l'
nsemble entier. \textit{\textbf{L}} est fonction des paramètres du modèle E_t^r.Comme nous voulons le maximum de \textit{\textbf{L}} sur les E_t^r , nous se réfère à  \textit{\textbf{L}} comme leur vraisemblance.

Nous avons:

kTln \textit{\textbf{L}} = \sum_S\sum_{i\inaa} n_S(t)E^r(t) - \sum_SG^f_S - NkTln Z = \sum_{t\inaa}n_S(t)E^r_t - \sum_SG^f_S - NkTlnZ,

avec N(t) le nombre d'acide aminé de type t dans l'ensemble \textit{\textbf{S}}.Le facteur de normalisation Z (ou fonction de partition) est la somme sur l'ensemble les séquences possibles \textit{\textbf{R}}:

Z=\sum_Sexp(-\beta\DeltaG_\textit{\textbf{R}})=\sum_\textit{\textbf{R}} exp(-\beta\DeltaG^f_\textit{\textbf{R}})\prod_{t\inaa}exp(\betan_\textit{\textbf{R}}(t)E^r_t)
Pour maximiser \textit{\textbf{L}}, nous considerons la dérivé de Z selon chacunes des E_t:

\frac{\partialZ}{\partialE^r_t}= \frac{\sum_\textit{\textbf{R}}n_\textit{\textbf{R}}(t)exp()-\beta\DeltaG_\textit{\textbf{R}}}{\frac_\textit{\textbf{R}}exp(-\beta\DeltaG_\textit{\textbf{R}})} = \langlen(t)\rangle.

La quantité à droite est la moyenne de Boltzmann du nombre n(t) des acides aminés t sur toutes les séquences possibles. En pratique, c'est la population moyenne de t que nous voudrions obtenir dans une longue simulation Monte Carlo. 

Pour que ln \textit{\textbf{L}} soit maximal il faut que ses dérivées par rapport à l'E_t^r soient nulles.

\frac{1}{N}\frac{\partial}{\partialE^r_t}ln \textit{\textbf{L}} = \frac{1}{N}\sum_Sn_S(t) - \langle(t)\rangle = \frac{N(t)}{N} - \langlenn(t)\rangle

et donc
\textit{\textbf{L}} maximum  \Rightarrow \frac{N(t)}{N} = \langlenn(t)\rangle, \forall t \in aa

Ainsi, pour maximiser L, nous devrions choisir {E^r_t} telle qu'une longue simulation donne les mêmes fréquences d'acides aminés que l'ensemble cible.



\subsection{Recherche du maximum de vraisemblance}


Nous utilisons trois méthodes pour approcher les valeurs {E^r_t}.

\begin{enumerate}
\item La première consiste à avancer dans la direction du gradient de ln(\textit{\textbf{L}}) en utilisant la règle itérative suivante (40):
E^r_t(n+1) = E^r_t(n) + \alpha\frac{\partial}{\partialE^r_t}ln(\textit{\textbf{L}})=E^r_t(n) + \deltaE(n^{exp}_t - \langle n(t)\rangle_n)

avec \alpha une constante, n^{exp}_t = \frac{N(t)}{N} la population moyenne d'acide aminé de type t dans l'ensemble ciblé ,
\langle\rangle_n indique une moyenne sur une simulation effectuée en utilisant les énergies de références courantes {E^r_t(n)}, et \deltaE une constante empirique avec la dimension d'une énergie,correspondant à l'amplitude de mise à jour. Cette procédure de mise à jour est répétée jusqu'à convergence. Nous appelons cette méthode, la méthode de mise à jour linéaire.

\item La deuxième méthode est une variante de la première dans laquelle le \deltaE n'est pas constant, mais ajusté au cours de la simulation de la façon suivante. La règle (11) est utilisées trois fois avec trois valeurs differentes et constantes pour le \deltaE ceci avec un jeu d'énergie de références identiques. une interpolation parabolique est efectuée sur les trois valeurs de la fonction proxy obtenues, le minimum de la parabole est calculée et est utilisée comme \deltaE pour le cycle suivant, en terme duquel les énergies sont mises à jours.

\item La troisième méthode, utilisée précédement (26,27), utilise une règle de mise à jour logarithmique:

  avec kT l'énergie thermique, fixée empiriquement à 0,5 kcal/mol. Nous l'appelons la méthode logarithmique. Dans les dernières itérations, certaines valeurs ont tandane à converger lentement,avec des oscillations. Par conséquent, une règle modifié où une énergie au cycle n et l'énergie au cycle  n-1 sont moyenenée avec un poids respectifs de 2/3 et 1/3.

\end{enumerate}
 
Chaque itération, dans la suite, ont étés effectué avec 500 000000 de pas par réplique de REMC.

\section{Méthodes de calcul}
  
\subsection{Fonction énergétique efficace pour l'état replié}

La matrice énergétique a été calculée avec la fonction d'énergie efficace suivante pour
État plié:
E = Ebonds + Eangles + Edihe + Eimpr + Evdw + ECoul + Esolv

Les six premiers termes de l'équation (13) représentent l'énergie interne de la protéine. Ils sont tirés de
La fonction d'énergie empirique Amber ff99SB (42), légèrement modifiée pour le CPD.

Les charges du backbone ont été remplacées par un ensemble unifié, obtenu en faisant la moyenne sur l'ensemble des types  d'acides aminés et ajuster légèrement pour rendre la partie backbone de chaque acide aminé neutre (43).
Le dernier terme à droite de l'équation (13), Esolv, représente la contribution du solvant.Nous avons utilisé un modèle de solvant implicite "Generalized Born + Surface Area" ou GBSA (44):

E_solv = D_GB + E_{surf} = \frac{1}{2}(\frac{1}{\epsilon_W} - \frac{1}{\epsilon_P})\sum_{ij} q_iq_j (r^2_{ij} + b_ib_jexp[-\frac{r^2_{ij}}{4b_ib_j}])^{-\frac{1}{2}} + \sum_i \sigma_iA_i

Ici, \epsilon_W et \epsilon_P sont les constantes diélectriques du solvant et de la protéine; r_{ij} est la distance entre les atomes i, j et b_i est le "rayon de solvatation" de l'atome i (44,45). A_i est la surface exposé accessible au solvant de l'atome i.\sigma_i  


%%% Local Variables:
%%% mode: latex
%%% TeX-master: "../these"
%%% End:

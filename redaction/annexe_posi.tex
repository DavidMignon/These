\chapter*{Annexe du chapitre \ref{chap:Comparaison}: Sélection des positions }
\label{chap:annexeposi}
Cette annexe détaille la sélection des positions actives employée dans le chapitre \ref{chap:Comparaison}.
Pour les tests avec une seule position active, comme des temps de calcul le permettent toutes les positions sont testées. Il y a huit cent quatre tests par algorithme.
Pour les autres groupes de tests (cinq, dix, vingt et trente positions actives), cinq tests sont effectués par protéine, c'est-à-dire quarante-cinq tests par algorithme.

Pour définir complètement les tests, il reste à décrire le choix des positions actives.
Il y a peu d'intérêt à tester des situations avec des positions actives qui n'interagissent pas ou faiblement entre elles.
En effet, s'il existe une position active i dont chaque résidu est sans interaction avec tous les résidus possibles aux autres positions actives, déterminer le meilleur état pour i est proche d'un test dans lequel i est la seule position active. Ce qui nous ramène vers les tests à une seule position active. Toutefois, cela n'est pas exactement la même question, parce que les positions actives différentes de i peuvent influencer la position de la chaîne latérale de positions inactives qui à leur tour peuvent influencer l'état en i.

Ainsi, le choix des positions actives ne se fait pas par tirage aléatoire, car le risque d'obtenir des positions avec peu d'interactions est trop grand. Il se fait sous contrainte d'interaction. Pour cela, nous utilisons la notion de voisinage de proteus, voir \ref{para:voisin}.

On dit qu'un ensemble de $N$ positions $\mathcla{V}$ est un \og voisinage fort \fg  pour le seuil $s_{vois}$ si pour toutes les paires de positions $(i,j)$ de $\mathcla{V}$, $i$ et $j$ sont voisines pour $s_{vois}$ .

Pour sélectionner les positions actives, nous nous basons sur la liste des voisins de chaque position à un seuil donné. Le programme proteus peut fournir ces listes en mode verbeux, sans effectuer d'optimisation. Pour cela, nous utilisons le mode Monte-Carlo avec une trajectoire de zéro pas. 

La recherche de voisinages forts se fait alors de la façon suivante:

\begin{enumerate}
\item $s_{vois}$ est initialisé à $10$.
\item construction des listes de voisins au seuil $s_{vois}$  
\item recherche de cinq voisinages forts , par extension progressive d'une paire de positions voisines si succès pour les neuf protéines arrêt.
\item sinon , les listes de voisins sont reconstruites avec $s_{vois}$  diminué de 1 kcal/mol. 
\item si $ s_{vois} \gep 1 $ retour à l'étape 2 sinon arrêt.  
\end{enumerate}

Nous obtenons les 45 voisinages forts, qui constitue notre sélection pour le groupe à 5 et 10 positions actives en utilisant un seuil égal à 10, ils sont dans les tableaux \ref{tab:select5} et \ref{tab:select10}. Pour les sélections 20-actives et 30-actives, il a fallu descendre le seuil à 1 pour obtenir les 45 voisinages forts proches de l'objectif, il manque une ou deux positions dans quelques cas, voir les tables \ref{tab:select20} et \ref{tab:select30}.

\begin{table}[!htbp]
  \centering
  \caption{ La sélection des positions pour tests 5-actives }
    \begin{tabular}{cl}      
      \toprule
      Nom & positions actives \\
      \cmidrule{1-2}
      1A81 1 & 10 13 16 84 86 \\
      1A81 2 & 20 21 24 27 116 \\
      1A81 3 & 35 38 56 105 107 \\
      1A81 4 & 44 47 52 65 67 \\
      1A81 5 & 82 84 86 87 90 \\
      1ABO 1 & 64 66 90 93 100 \\
      1ABO 2 & 72 74 80 104 111 \\
      1ABO 3 & 79 82 102 111 115 \\
      1ABO 4 & 83 86 104 105 106 \\
      1ABO 5 & 93 100 102 113 116 \\
      1BM2 1 & 101 106 140 141 146 \\
      1BM2 2 & 120 128 131 132 135 \\
      1BM2 3 & 58 61 127 128 129 \\
      1BM2 4 & 74 75 98 100 105 \\
      1BM2 5 & 85 87 95 110 128 \\
      1CKA 1 & 136 138 158 175 190 \\
      1CKA 2 & 149 166 169 171 181 \\
      1CKA 3 & 151 153 157 159 172 \\
      1CKA 4 & 164 170 172 184 187 \\
      1CKA 5 & 172 174 182 186 187 \\
      1G9O 1 & 10 13 54 57 92 \\
      1G9O 2 & 15 39 42 54 57 \\
      1G9O 3 & 24 26 28 39 42 \\
      1G9O 4 & 48 53 57 59 88 \\
      1G9O 5 & 75 78 79 86 88 \\
      1M61 1 & 12 20 23 24 27 \\
      1M61 2 & 17 20 24 37 49 \\
      1M61 3 & 27 33 51 100 102 \\
      1M61 4 & 5 8 10 11 36 \\
      1M61 5 & 59 71 84 87 94 \\
      1O4C 1 & 20 21 32 34 46 \\
      1O4C 2 & 2 71 79 81 82 \\
      1O4C 3 & 33 45 63 71 73 \\
      1O4C 4 & 43 45 63 71 85 \\
      1O4C 5 & 8 33 82 83 86 \\
      1R6J 1 & 194 237 239 270 272 \\
      1R6J 2 & 199 201 211 218 232 \\
      1R6J 3 & 213 218 227 232 238 \\
      1R6J 4 & 221 227 232 267 269 \\
      1R6J 5 & 241 254 258 267 269 \\
      2BYG 1 & 189 191 221 244 246 \\
      2BYG 2 & 205 224 239 245 248 \\
      2BYG 3 & 232 233 265 272 274 \\
      2BYG 4 & 238 240 243 276 278 \\
      2BYG 5 & 253 261 264 265 274 \\
      
      \bottomrule
    \end{tabular}      
  \label{tab:select5}      
\end{table}


\begin{table}[!htbp]
  \centering
  \caption{La sélection des positions pour tests 10-actives}  
  \begin{tabular}{cl}
      
    \toprule
    Nom & positions actives \\
    \cmidrule{1-2}
    1A81 1  & 13 15 39 41 53 86 89 90 93 103 \\
    1A81 2  & 39 41 53 55 64 66 76 89 92 103 \\
    1A81 3  & 51 53 64 66 68 74 76 82 88 92 \\
    1A81 4  & 76 82 87 88 90 91 92 95 97 99 \\
    1A81 5  & 9 10 11 16 41 51 53 66 88 89 \\
    1ABO 1  & 64 72 74 79 89 91 101 103 108 111 \\
    1ABO 2  & 66 68 80 82 88 90 100 102 104 111 \\
    1ABO 3  & 69 70 72 74 80 81 106 113 114 115 \\
    1ABO 4  & 71 78 83 84 94 99 101 104 105 106 \\
    1ABO 5  & 72 79 82 94 99 102 104 106 111 115 \\
    1BM2 1  & 119 120 121 122 123 125 131 134 135 140 \\
    1BM2 2  & 125 126 127 129 130 133 134 136 137 147 \\
    1BM2 3  & 83 99 101 106 108 135 140 141 146 148 \\
    1BM2 4  & 85 95 97 110 118 120 125 128 131 132 \\
    1BM2 5  & 99 101 106 139 140 141 142 143 144 146 \\
    1CKA 1  & 134 135 160 161 162 173 174 175 176 179 \\
    1CKA 2  & 137 139 143 151 153 157 159 172 182 186 \\
    1CKA 3  & 138 140 147 149 150 155 166 169 181 188 \\
    1CKA 4  & 140 141 153 154 155 157 174 175 184 186 \\
    1CKA 5  & 151 153 157 166 168 173 174 176 178 179 \\
    1G9O 1  & 10 11 13 14 15 16 53 54 57 92 \\
    1G9O 2  & 15 17 24 26 39 42 48 51 53 88 \\
    1G9O 3  & 26 28 39 42 48 53 57 59 88 90 \\
    1G9O 4  & 34 35 58 60 68 70 74 75 89 91 \\
    1G9O 5  & 71 73 74 77 80 81 82 83 84 85 \\
    1M61 1  & 10 12 20 23 24 27 35 49 102 104 \\
    1M61 2  & 17 20 21 24 37 39 40 47 49 58 \\
    1M61 3  & 34 36 46 48 59 61 71 83 84 87 \\
    1M61 4  & 5 6 11 36 46 48 61 69 83 84 \\
    1M61 5  & 59 61 70 71 75 77 83 86 87 92 \\
    1O4C 1  & 31 33 45 47 61 63 73 86 89 100 \\
    1O4C 2  & 50 51 52 53 63 72 73 77 85 89 \\
    1O4C 3  & 61 62 63 71 72 73 79 85 88 89 \\
    1O4C 4  & 73 74 75 76 77 89 92 94 96 101 \\
    1O4C 5  & 90 91 93 96 98 99 101 102 103 104 \\
    1R6J 1  & 193 194 195 197 199 218 232 236 267 269 \\
    1R6J 2  & 199 209 211 213 218 227 232 238 265 267 \\
    1R6J 3  & 201 204 205 209 211 218 241 258 265 267 \\
    1R6J 4  & 209 211 213 218 227 238 241 258 265 267 \\
    1R6J 5  & 238 240 241 242 246 257 258 261 265 267 \\
    2BYG 1  & 194 196 203 205 224 233 239 245 274 276 \\
    2BYG 2  & 203 205 207 224 227 233 239 243 245 276 \\
    2BYG 3  & 206 207 222 245 248 253 256 261 264 265 \\
    2BYG 4  & 221 222 245 248 251 253 256 261 264 265 \\
    2BYG 5  & 247 248 249 250 251 252 259 262 263 275 \\
    
    \bottomrule
    
  \end{tabular}
\label{tab:select10}      
    \end{table}


\begin{table}[!htbp]
  \caption{ La sélection des positions pour tests 20-actives }
  \scalebox{0.85}{
    \begin{tabular}{cl}
      
      \toprule
      Nom & positions actives \\
      \cmidrule{1-3}
      1A81 1 & 9 11 12 13 15 16 17 19 20 25 28 39 40 41 42 43 44 45 114 117 \\
      1A81 2 & 9 11 12 13 15 16 17 19 20 25 28 39 40 41 42 43 44 45 47 117 \\
      1A81 3 & 9 11 12 13 15 16 17 19 41 43 48 51 68 74 84 86 109 114 117 \\
      1A81 4 & 12 13 15 16 17 19 20 25 28 39 40 41 42 43 44 45 47 86 114 117 \\ 
      1A81 5 & 13 15 16 19 41 43 48 51 60 64 68 70 71 74 84 86 87 88 109 114 117 \\
      1ABO 1 & 64 66 67 68 82 86 87 88 89 90 91 101 102 103 108 111 113 116 \\
      1ABO 2 & 64 65 66 67 84 87 88 89 90 91 93 100 101 102 103 108 111 113 116 \\
      1ABO 3 & 65 66 67 87 88 89 90 91 93 94 95 100 101 102 103 106 108 111 113 116 \\
      1ABO 4 & 64 65 66 67 69 87 88 89 90 91 93 100 101 102 103 106 108 111 113 116 \\
      1ABO 5 & 66 67 68 82 86 87 88 89 90 91 93 100 101 102 103 106 108 111 113 116 \\
      1BM2 1 & 55 56 60 61 62 83 84 85 86 87 95 97 99 110 118 125 127 133 150 152 \\
      1BM2 2 & 55 56 60 61 62 83 84 85 86 87 95 97 99 110 118 125 127 128 129 152 \\
      1BM2 3 & 55 56 58 60 61 62 64 67 69 73 83 84 85 86 87 129 132 133 150 152 \\
      1BM2 4 & 55 56 60 61 62 69 83 84 85 86 87 95 97 99 110 129 132 133 150 152 \\
      1BM2 5 & 58 60 60 61 61 62 64 67 69 73 75 83 84 85 86 129 132 133 150 152 \\
      1CKA 1 & 134 135 136 137 138 139 150 151 160 161 162 163 164 170 171 172 173 179 189 190 \\
      1CKA 2 & 134 135 136 137 139 150 151 153 160 161 162 163 164 170 171 172 173 179 189 190 \\
      1CKA 3 & 134 136 137 139 150 151 157 158 160 161 162 163 164 170 171 172 173 179 189 190 \\
      1CKA 4 & 136 137 139 150 151 153 158 159 160 161 162 163 164 170 171 172 173 179 189 190 \\
      1CKA 5 & 137 139 150 151 153 158 160 161 162 163 164 170 171 172 173 174 175 179 189 190 \\
      1G9O 1 & 9 10 11 13 14 15 31 34 38 54 57 58 60 68 90 91 92 94 95 96 \\
      1G9O 2 & 9 11 13 14 15 16 31 34 38 54 57 58 60 68 90 91 92 94 95 96 \\
      1G9O 3 & 9 11 13 14 15 31 34 38 54 55 57 58 60 68 90 91 92 94 95 96 \\
      1G9O 4 & 9 11 13 15 16 17 54 57 58 59 60 61 68 89 90 91 92 94 95 96 \\
      1G9O 5 & 10 11 13 15 16 17 54 57 58 60 61 68 89 90 91 92 94 95 96 \\
      1M61 1 & 34 35 36 37 38 39 42 46 47 48 49 50 61 63 69 71 77 78 81 82 \\
      1M61 2 & 35 36 37 38 39 42 46 47 48 49 50 61 63 69 71 77 78 81 82 83 \\
      1M61 3 & 38 39 42 46 47 48 49 50 61 63 69 71 77 78 81 82 83 84 85 87 \\
      1M61 4 & 42 46 47 48 49 50 61 63 69 71 77 78 81 82 83 84 85 87 88 98 \\
      1M61 5 & 5 7 50 61 63 69 71 77 78 81 82 83 84 85 87 88 98 103 104 109 \\
      1O4C 1 & 32 33 34 35 43 45 63 65 71 73 79 80 81 82 83 84 85 86 87 89 \\
      1O4C 2 & 3 4 5 7 8 9 11 17 31 32 33 34 35 43 45 63 65 71 73 79 \\
      1O4C 3 & 1 3 4 5 6 7 8 9 11 17 31 32 33 35 43 45 65 81 82 83 \\
      1O4C 4 & 1 2 3 4 5 6 7 8 9 11 12 13 14 17 19 35 65 81 82 83 \\
      1O4C 5 & 1 3 4 5 6 7 8 9 11 12 17 31 32 33 34 35 65 81 82 83 \\
      1R6J 1 & 193 194 195 197 214 215 217 218 233 235 236 237 239 240 241 242 247 269 270 273 \\
      1R6J 2 & 193 194 197 198 199 217 233 235 236 237 238 239 240 241 242 247 268 270 272 273 \\
      1R6J 3 & 193 195 197 217 233 235 236 239 240 241 242 244 245 247 268 269 270 272 273 \\
      1R6J 4 & 193 195 197 217 233 235 236 237 239 241 242 244 245 247 268 269 270 272 273 \\
      1R6J 5 & 193 194 197 198 199 233 236 237 239 240 241 247 268 269 270 272 273 \\
      2BYG 1 & 186 187 188 189 190 191 192 215 216 219 244 246 270 271 273 274 278 280 281 282 \\
      2BYG 2 & 186 187 188 189 190 215 216 219 221 223 240 243 270 271 273 274 278 280 281 282 \\
      2BYG 3 & 186 187 188 189 190 215 216 219 221 223 240 243 244 270 271 273 278 280 281 282 \\
      2BYG 4 & 186 187 188 189 190 215 216 219 221 223 240 243 244 270 274 276 278 280 281 282 \\
      2BYG 5 & 187 189 190 191 192 215 216 219 221 223 240 243 244 270 274 276 278 280 281 282 \\
      
      \bottomrule
      
    \end{tabular}
  }      
\label{tab:select20}      
\end{table}


\begin{table}[!htbp]
  \centering
  \caption{ La sélection des positions pour tests 30-actives }
  \scalebox{0.75}{
    \begin{tabular}{cl}
      
      \toprule
        Nom & positions actives \\
        \cmidrule{1-2}
        1A81 1 & 9 11 12 13 15 16 17 19 20 25 26 27 28 29 36 38 39 40 41 42 43 48 51 68 74 84 86 109 114 117 \\
        1A81 2 & 9 10 11 12 13 15 16 17 19 20 25 28 39 41 43 48 51 68 74 83 84 86 87 88 90 91 93 109 114 117 \\
        1A81 3 & 9 11 12 13 15 16 17 19 20 25 27 28 36 38 39 40 41 42 43 48 51 68 74 84 86 109 114 117 \\
        1A81 4 & 9 10 11 12 13 15 16 17 19 20 25 28 36 39 40 41 42 43 44 45 48 51 68 74 84 86 109 114 117 \\
        1A81 5 & 9 10 11 12 13 15 16 17 19 20 25 28 39 40 41 42 43 44 45 47 48 51 52 68 74 84 86 109 114 117 \\
        1ABO 1 & 64 65 66 67 68 70 71 72 75 78 79 80 81 82 83 86 87 88 89 90 91 93 100 101 102 103 108 111 113 116 \\
        1ABO 2 & 64 65 66 67 68 72 75 78 80 81 82 83 84 86 87 88 89 90 91 93 94 100 101 102 103 104 108 111 113 116 \\
        1ABO 3 & 64 66 67 68 70 71 72 78 82 86 87 88 89 90 91 93 94 95 96 99 100 101 102 103 104 105 108 111 113 116 \\
        1ABO 4 & 64 65 66 67 70 71 72 68 82 86 87 88 89 90 91 93 94 95 96 99 100 101 102 103 104 105 108 111 113 116 \\
        1ABO 5 & 65 66 67 70 71 72 75 78 80 82 86 87 88 89 90 91 93 94 95 96 99 100 101 102 103 108 111 113 116 \\
        1BM2 1 & 55 56 58 60 61 62 83 84 85 86 87 95 97 99 110 118 119 120 121 122 125 127 128 129 130 131 132 133 150 152 \\
        1BM2 2 & 56 58 60 61 62 83 84 85 86 87 95 97 99 110 118 119 120 121 122 123 125 127 128 129 130 131 132 133 150 152 \\
        1BM2 3 & 58 60 61 62 83 84 85 86 87 95 97 99 108 109 110 118 120 121 122 123 125 127 128 129 132 133 134 135 150 152 \\
        1BM2 4 & 55 56 58 60 61 62 83 84 85 86 87 95 97 99 108 109 110 118 120 121 125 127 128 129 130 131 132 133 150 152 \\
        1BM2 5 & 56 58 60 61 62 67 83 84 85 86 87 95 97 99 110 111 112 113 115 118 125 127 128 129 130 131 132 133 150 152 \\
        1CKA 1 & 134 135 136 137 139 140 141 142 143 144 146 147 148 149 150 151 158 159 160 161 162 163 164 170 171 172 173 179 189 190 \\
        1CKA 2 & 134 135 136 137 139 143 144 146 147 148 149 150 151 158 159 160 161 162 163 164 170 171 172 173 179 186 187 188 189 190 \\
        1CKA 3 & 135 136 137 139 144 146 147 148 149 150 151 157 158 159 160 161 162 163 164 170 171 172 173 179 186 187 188 189 190 \\
        1CKA 4 & 136 137 139 140 141 142 143 144 146 147 148 151 158 159 160 161 162 163 164 170 171 172 173 179 184 186 187 188 189 190 \\
        1CKA 5 & 134 136 137 139 140 141 142 143 144 146 147 148 151 158 159 160 161 162 163 164 170 171 172 173 179 182 187 188 189 190 \\
        1G9O 1 & 9 10 11 13 15 24 31 34 38 40 41 42 43 46 48 49 50 51 54 57 58 60 68 89 90 91 92 94 95 96 \\
        1G9O 2 & 9 11 13 15 31 32 34 38 40 41 42 43 46 48 49 50 51 54 57 58 60 68 89 90 91 92 94 95 96 \\
        1G9O 3 & 9 10 11 13 15 31 32 34 38 40 41 42 43 46 48 49 50 51 54 57 58 60 68 89 90 91 92 94 95 96 \\
        1G9O 4 & 10 11 13 14 15 31 32 34 38 40 41 42 43 46 48 49 50 51 54 57 58 60 61 68 89 90 91 92 94 95 96 \\
        1G9O 5 & 10 11 13 14 15 31 32 40 41 42 43 46 48 49 50 51 54 57 58 60 61 62 68 87 89 90 91 92 94 95 96 \\
        1M61 1 & 12 14 15 20 34 35 36 37 38 39 42 46 47 48 49 50 61 63 69 71 77 78 81 82 83 84 85 87 88 98 \\
        1M61 2 & 6 7 8 10 11 12 14 15 20 21 34 35 36 37 38 39 42 46 47 48 49 50 61 63 69 71 77 78 81 82 \\
        1M61 3 & 5 7 35 36 37 38 39 42 46 47 48 49 50 61 63 69 71 77 78 81 82 83 84 85 87 88 98 103 104 109 \\
        1M61 4 & 7 8 10 11 12 14 15 20 34 35 36 37 38 39 42 46 47 48 49 50 61 63 69 71 77 78 81 82 83 84 \\
        1M61 5 & 8 10 11 12 14 15 20 34 35 36 37 38 39 42 46 47 48 49 50 61 63 69 71 77 78 81 82 83 84 85 \\
        1O4C 1 & 1 2 3 4 5 6 7 8 9 11 17 31 32 33 34 35 43 45 63 65 71 81 82 83 90 91 92 93 94 96\\
        1O4C 2 & 1 3 4 5 7 8 9 11 17 31 32 33 34 35 43 45 63 65 71 73 79 80 81 82 83 90 91 92 93 96\\
        1O4C 3 & 1 3 4 5 6 7 8 9 11 17 31 32 33 34 35 43 45 63 65 71 73 79 80 81 82 83 90 91 92 93 96 \\
        1O4C 4 & 1 3 4 5 7 8 9 11 17 31 32 33 34 35 43 45 63 65 71 73 79 80 81 82 83 84 91 92 93 96 \\
        1O4C 5 & 1 3 4 5 6 7 8 9 11 17 31 32 33 34 35 43 45 63 65 71 73 79 80 81 82 83 84 92 93 96 \\
        1R6J 1 & 193 194 195 197 198 199 217 218 219 220 221 222 223 224 225 226 227 228 229 233 235 236 237 239 247 268 269 270 272 273 \\
        1R6J 2 & 193 194 195 197 198 199 217 220 221 222 223 224 225 226 227 228 229 233 235 236 237 239 240 247 268 269 270 272 273 \\
        1R6J 3 & 193 194 195 197 198 199 208 217 220 221 222 223 224 225 226 227 228 229 230 233 235 236 237 239 247 268 269 270 272 273 \\
        1R6J 4 & 193 194 195 197 198 199 217 220 221 222 223 224 225 226 227 228 229 233 235 236 237 239 240 247 249 268 269 270 272 273 \\
        1R6J 5 & 194 195 197 198 199 217 220 221 222 223 224 225 226 227 228 229 233 235 236 237 239 240 247 249 250 268 269 270 272 273 \\
        2BYG 1 & 186 187 188 189 190 191 192 215 216 219 221 223 240 243 244 246 250 251 252 253 254 255 256 257 259 260 278 280 281 282 \\
        2BYG 2 & 186 187 188 189 190 191 192 198 215 216 219 221 223 240 243 244 251 252 253 254 255 256 257 259 260 278 280 281 282 \\
        2BYG 3 & 186 187 188 189 190 191 192 215 216 219 221 223 240 243 244 251 252 253 254 255 256 257 259 260 278 246 280 281 282 \\
        2BYG 4 & 186 187 188 189 190 191 192 215 216 219 221 223 240 243 244 245 246 251 252 253 254 255 256 257 259 260 278 281 282 \\
        2BYG 5 & 186 187 188 189 190 191 192 215 216 219 221 223 240 243 244 246 251 252 253 254 255 256 257 259 260 278 280 281 282 \\

        \bottomrule

      \end{tabular}   
}   
\label{tab:select30}      
\end{table}


%%% Local Variables:
%%% mode: latex
%%% TeX-master: "these"
%%% End:

\documentclass[a4paper,12pt]{article}
\usepackage[hmargin=2.5cm,vmargin=2.5cm]{geometry}

\usepackage[utf8]{inputenc}   % le fichier .tex est en UTF-8     
\usepackage[francais]{babel}  %typo française                    
\usepackage[T1]{fontenc}      % encodage des fonts latex         
\usepackage{lmodern}                                             
\usepackage{microtype}        % typo supplémentaires             


\usepackage{graphicx} %inclusion de graphiques


\usepackage{hyperref} % liens dans le pdf
\hypersetup{%
  pdftitle={Title},
  pdfauthor={Author1, Author2},
  pdfkeywords={keywords}
  pdfsubject={article},
  colorlinks=true,
  linkcolor=black,
  urlcolor=black,
  citecolor=black
}



\begin{document}



    \begin{table}[!htbp]
      \centering


      \begin{tabular}{|p{0.2\linewidth}|p{0.35\linewidth}|p{0.45\linewidth}|}

        \hline
        Tag type  & Tag name & Description \\
        \hline
          Exploration method  & Mode &  define the execution mode, the possible values are  HEURISTIC,MONTECARLO,MEANFIELD and POSTPROCESS  \\  \hline    
                        & Trajectory\_Length  &  the length of a Monte Carlo trajectory \\  \cline{2-3}
        Number of steps & Trajectory\_Number  &  the number of Monte Carlo trajectories    \\  \cline{2-3}
                        & Cycle\_Number  &    proteus gives a sequence at each cycle in the HEURISTIC mode   \\ \cline{2-3}  
                        & Sequence\_Loop\_Number  &  the maximun number of iteration over the structure at each cycle.(only in the HEURISTIC mode)    \\ \hline  
        Choice of the starting sequence/structure &  Seq\_Input\_File &   a input file with the starting values    \\ \cline{2-3}
                                                  & Rseed\_Definition  &   The seed value for the random number generator (the generator sets the starting values if Seq\_Input\_File is not defined   \\    \hline 
        Energy function &  Optimization\_Configuration &   definition of the energy function\\               \cline{2-3}
                        &  Group\_definition &   group energies and group interaction energies are the basic elements of the energy function\\  \hline  
        Restrictions on sequence/rotamer space & Space\_Constraints   &  restrict the possible states or force residues to have the same amino acid  \\ \hline                
                         & Surf\_Ener\_Factor  &   energy parameter\\               \cline{2-3}
        Model parameters & Dielectric\_Constant &  energy parameter \\               \cline{2-3}
                         & Lambda\_Parameter &  usefull for the Mean Field mode \\              \cline{2-3} 
                         & Temperature & usefull for the Mean Field or Monte Carlo mode  \\          \hline     

      \end{tabular} 

      \caption{ Possible commands in the proteus command file}      

      \label{table1}

  \end{table}

\end{document}






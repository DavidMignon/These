\chapter{Comparaisons d'algorithmes}
\label{chap:resultats_comparaisons}


\section{Les méthodes pratiques} 

Nous cherchons maintenant à déterminer les performances et les qualités des différents algorithmes de proteus.
Pour évaluer les qualités des différents algorithmes de proteus, nous effectuons un ensemble de tests. 
Plusieurs questions se présentent alors,premièrement grâce à l' algorithme de type toulbar2 il est possible d'obtenir la séquence/conformation qui possède l'énergie de dépliement la plus haute. Cela constitue une information important qui va nous servir de point de comparaison. Mais le facteur temps pose problème, nous ne savons pas à l'avance quand toulbar2 termine. Et il apparaît d'emblée illusoire d'espérer voir toulbar2 converger dans toutes les situations qui nous intéressent dans un temps raisonnable, en particulier pour une situation où  toutes les positions du  backbone sont autorisé à muter et pour chaque type d'acide aminé la chaîne latérale peut prendre toutes les positions précédemment calculées dans la matrice. 


Dans la suite on appelle position active, une position pour laquelle, lors de la recherche par proteus, tous les types d'acides aminés sont autorisés et tous les rotamères de chaque type d'acide aminé sont autorisés.


\subsection{les protéines}
 
Les tests sont effectués sur neuf protéines choisies pour avoir des longueurs de backbone variés , plusieurs familles SCOP(?) représentées, mais aussi plusieurs structures pour chaque famille présente. Ainsi l'ensemble se décompose en deux protéines SH3 de 56 et 57 résidus, de trois protéines PDZ de longueur comprise entre 82 et 97 résidus  et enfin de trois protéines SH2  longues de 105 ou 109 résidus (Table~\ref{tab:protéines}).  


    \begin{table}[!htbp]
      \centering

      \begin{tabular}{cccc}

        \toprule
        Code PDB & résidus & nombre de positions & famille\\
        \cmidrule{1-4}
        1ABO & 	64-119	 & 	56	 & SH3 \\
        1CKA & 	134-190	 & 	57	 & SH3 \\
        1R6J & 	192-273	 & 	82	 & PDZ \\
        1G9O & 	9-99	 & 	91	 & PDZ \\
        2BYG & 	186-282	 & 	97	 & PDZ \\
        1BM2 & 	55-152	 & 	98	 & SH2 \\
        1O4C & 	1-105	 & 	105	 & SH2 \\
        1M61 & 	4-112	 & 	109	 & SH2 \\
        1A81 & 	9-117	 & 	109	 & SH2 \\
        \bottomrule

      \end{tabular}      
      \caption{Les protéines}
\label{tab:protéines}      
    \end{table}

\subsection{Description des tests}

Les tests sont réparties en deux ensembles:
\begin{enumerate}
\item un ensemble de tests où toutes les positions du backbone sont actives (cela correspond aux situations de design complet de protéines) 
\item un ensemble de tests où le nombre de positions actives est gardé sous contrôle .De façon à maîtriser la taille de l'espace d'exploration
\end{enumerate}


\paragraph{L'ensemble 'Tout actif'}

Pour le premier ensemble de tests,La totalité de la matrice d'énergie est exploitée et pour chaque position l'espace d'exploration correspond à l'espace d'état déclaré dans le fichier .bb.
Comme l'espace des séquences/confirmations à explorer est gigantesque, nous ne faisons pas de tentative de recherche du GMEC  par recheche exacte. 

Nous effectuons des recherches avec les algorithmes suivants:

\begin{itemize}
\item heuristique  (noté H par la suite) ;
\item Monte-Carlo (noté MC);
\item ``'Replica Exchange'' (RE);
\end{itemize}


\paragraph{L'ensemble 'Nombre d'actifs limité'}

L'ensemble ``Nombre d'actifs limité''' est composé de six groupes de tests avec un nombre de positions actives fixe:  


\begin{enumerate}
\item aucune position active
\item une position active 
\item cinq positions 
\item dix  positions 
\item vingt positions 
\item trente positions 
\end{enumerate}

Lorsqu'une position n'est pas active, on fixe l'acide aminé de la position en utilisant l'acide aminé de la séquence native.

Le groupe `aucune position active ` est constitué d'un test par algorithme pour chaque protéine. Il y a donc neuf tests par algorithme.
Ce sont les tests pendant lesquels  la séquence d'acides aminés est fixe et correspond à la séquence native de la protéine.

Pour les tests avec une seule position active, comme des temps de calcul le permettent, nous décidons d'être exhaustif:
Toutes les positions sont testées, Il y a alors huit cent quatre tests par algorithme.
Pour tous les autres groupes de tests (5,10,20 et 30 positions actives) cinq tests sont effectués par protéine, c'est-à-dire quarante-cinq tests par algorithme.

\paragraph{le choix des positions actives}

Pour définir complètement les tests ,il reste maintenant à décrire comment le choix des positions actives pour les groupes numéro trois à numéro six a été effectué.
Il y a peu d'intérêt à tester des situations avec position active sans interaction avec les autres positions actives. 
En effet s'il existe une position active P dont chaque résidu est sans interaction avec tous les résidus possibles des autres positions actives, déterminer le meilleur état pour P est proche du test du groupe 2 avec P comme position active. Notons qu'en même que cela n'est pas exactement la même question parce que les positions actives différentes de P peuvent influencer la position de la chaîne latérale de positions inactives qui à leur tour peuvent influencer l'état de P.
Ainsi le choix des positions actives se fait non pas pas tirage aléatoire, car le risque d'avoir les positions peu en inter action est trop grand. Il se fait sous contrainte d'interaction.

\paragraph{positions en interactions}
Pour cela nous utilisons la notion de voisinage incluse dans proteus: 
Deux positions P et Q sont en interaction s'il existe un rotamère $r_P$ de P et un rotamère $r_Q$ de Q tels que:
\begin{displaymath}
 | E(r_P,r_Q) | > S_{Vois}
\end{displaymath} 
avec $S_{Vois}$ un seuil donné par utilisateur à la configuration de proteus (voir chap. ?? pour les détails).

Définissons maintenant la notion de n-uplet en interaction par la donnée de n positions avec $n \in \{5,10,20,30\}$ et d'un seuil  $S_{Vois}$  tels que pour toute paire de positions (P,Q) du n-uplet, P et Q sont en interaction.
\paragraph{}
Pour déterminer les positions actives,nous exécutons proteus en mode verbeux sans effectué d'optimisation.
Il existe plusieurs façons de procéder ,ici nous utilisons le mode Monte-Carlo avec une trajectoire de zéro pas. Ces exécutions produisent en sortie standard la liste des voisins pour chaque position au seuil donnée en paramètre.
Pour chaque protéine, nous exécutons proteus avec  $S_{Vois}$ égal à dix , cinq et un à tour de rôle. Nous obtenons quatre listes de voisins. 
Ensuite, nous cherchons dans les listes, par un script dédié, les n-uplets en interaction en partant de la liste de voisins au sens le plus fort, c'est-à-dire 10, vers la liste de voisins au sens le plus faible ($0.1$).La recherche s'arrête si cinq n-uplets sont trouvés.

Nous obtenons quarante cinq n-uplets pour le groupe à cinq (respectivement dix, vingt et trente ) positions actives pour un $S_{Vois}$ égale à dix (respectivement dix, un et un) (voir le détails en annexe~\ref{chap:annexe}). Chaque n-uplet nous créons un fichier de configuration de proteus dans lequel la balise <Space\_Constraints> fixe les positions restantes au type d'acide aminé présent dans la séquence native. 


\subsubsection{Définition de protocole comparable}

Nous voulons comparer les algorithmes très différents. Un algorithme peut garantir l'obtention du minimum global en énergie (GMEC) s'il termine, mais ne garantit pas qu'il termine, un autre permet un contrôle très fin du temps d'exécution sans garantie du GMEC, et d'autres enfin ont des objectifs plus ambitieux  qu'uniquement l'obtention du GMEC.
Mais le GMEC reste le meilleur point de commun. Nous allons donc y concentrer les comparaisons.

Nous devons noter également que l'obtention du GMEC est théorique, en pratique nous n'avons pas de preuve que le code de l'algorithme exact que nous utilisons soit sans bogue. Cependant,nous mettons de côté cette éventualité et dans toute la suite GMEC désigne aussi bien le minimum global en énergie que le résultat de toulbar2 lorsqu'il termine.  
\subparagraph{}
Le Monte-Carlo et le ``'Replica exchange'' possèdent de nombreux paramètres de configuration, ce qui rend l'ensemble des protocoles possibles très grand. Se pose alors la question de l'optimisation du protocole. L'objectif que nous nous fixons ici,n'est pas de recherche un protocole optimum pour chacun des tests, mais de tester un protocole optimisé par algorithme.
Nous allons alors dans un premier temps , recherche les meilleurs protocoles Monte-Carlo et les meilleurs protocoles ``'Replica Exchange'' sur l'ensemble de tests ``tout actifs''
Puis, sur la base des résultats obtenue, nous ferons l'ensemble des tests à positions actives limitées avec les meilleurs protocoles.
Le programme toulbar2 possède aussi de nombreuses options. Deux paramétrages différents seront utilisés.


Pour rendre les protocoles comparables, le temps d'exécution maximum est fixé à vingt-quatre heures, sauf mention contraire.
Toulbar2 donne sa meilleure séquence/conformation en dernier, il n'y a donc pas post-traitement nécessaire.
C'est également le cas pour le MC à condition de configurer l'impression de la trajectoire avec la balise $Print\_Threshold=0$. dans le fichier de configuration.
Pour le RE et l'Heuristique, un tri des séquences selon l'énergie est nécessaire. Mais il n'y a pas beaucoup de séquences: 
\begin{enumerate}
\item L' Heuristique fournit une séquence/conformation à chaque cycle.
\item Le RE avec $Print\_Threshold=0$ produit autant de fichiers de séquences/rotamères que de marcheurs.Chacun ne contenant pas plus de quelques dizaines de séquences/rotamères. 
\end{enumerate}

Nous pouvons donc négliger la durée du tri dans le temps total d'exécution.    


\subsection{Protocole Heuristique}

Pour l'algorithme Heuristique, il n'y a dans notre situation qu'un seul paramètre à renseigner: le nombre de cycles à effectuer. Quelques essais préliminaires sur la plus grosse protéine (Table~\ref{tab:protéines}) dans le cas où tout actif , montre que la version utilisée de proteus peut effectuer jusqu'à environ 110000 cycles sur nos machines de calculs en l'espace de vingt-quatre heures. Ainsi le protocole H est défini comme le protocole qui utilise le mode heuristique de proteus et qui effectue cent dix mille cycles. Nous définissons également les variantes H- , H+ et H++ comme des protocoles plus courts ou plus longs à facteur entier près (Table~\ref{tab:protoH}). Par ailleurs certains comparaisons de l'heuristique avec la MC ont été faites avec une version précédente de proteus notons h ce protocole. il diffère  aussi de H par le fait que les options d'optimisation du compilateur Intel -O2 contre -O3 pour H.    


    \begin{table}[!htbp]
      \centering

      \begin{tabular}{lr}

        \toprule
        Nom & nombre de cycles \\
        \cmidrule{1-2}
        H   & 110000 \\  
        H-  & 1100   \\  
        H+  & 330000 \\  
        H++ & 990000 \\  
        h   & 100000 \\  
        \bottomrule

      \end{tabular}      
      \caption{Les protocoles Heuristique}
\label{tab:protoH}      
    \end{table}

   \subsection{Protocoles Monte-Carlo}
\label{sec:MC}
On distingue deux ensembles de protocoles, Le premier où les noms  sont de la forme mc*, rassemble les protocoles utilisés pendant l'optimisation du Monte-Carlo.Le second est constitué du protocole retenue comme optimisé , plus une variante.     

\subsubsection{Optimisation}

Les éléments à paramétrer pour l'algorithme Monte-Carlo sont les suivants:

\begin{enumerate}
\item la température
\item le nombre de pas (avec le nombre de trajectoires et la longueur de trajectoire )
\item Le seuil de voisinage
\item Les probabilités de changements de la séquences/rotamère
\end{enumerate}

Ce qui représente un ensemble de protocoles trop grand pour une approche exhaustive. Nous allons faire varier les paramètres un par un.

La température est le paramètre principale du Monte-carlo, c'est lui qui contrôle le taux d'acceptation du critère de Metropolis.Alors la première étape de cette optimisation va consisté à faire variés la température , entre 0.001 et 0.5 , en conservant les autres paramètres fixés (protocoles de mc0 à mc5).Le nombre de pas total effectué est le produit de deux paramètres, le nombre de trajectoires et la longueur de trajectoire. les protocoles mc1b et mc2b testent l'effet d'une augmentation du nombre de pas. Tandis que mc2d teste l'effet de variation du rapport nombre de trajectoire sur la longueur.
Le protocole mc2e s'intéresse aux probabilités de changement de la trajectoire. Il y a cinq balises dans proteus qui contrôle ces changements:

\begin{description}
\item[<Prot>] donne la probabilités de modifications de  rotamère à une postions.
\item[<Prot\_Prot>] donne la probabilités de modifications de  rotamère à deux positions.
\item[<Mut>] donne la probabilités de modifications de type de résidu à une position.
\item[<Mut\_Prot>] donne la probabilités de modifications de  rotamères à deux positions.
\item[<Mut\_Mut>] donne la probabilités de modifications de type de résidu à deux positions.
\end{description}

La table~\ref{tab:protoMC} donne les probabilités utilisées par ces cinq paramètres dans l'ordre de la liste précédente. 

 Enfin mc4b se distingue des autres par un seuil de voisinage plus grand ((Table~\ref{tab:protoMC})).


\paragraph{}
Pour l'étape suivante, qui consiste à la comparaison avec les autres algorithmes, deux protocoles sont utilisés MC2d2 et MC2e2 ils inspirent de mc2d et mc2e , en étant adapté à la contrainte du temps de calculs de la comparaison d'algorithmes et un utilisant la nouvelle version de proteus. 
  

    \begin{table}[!htbp]
      \centering

      \begin{tabular}{llrrcc}

        \toprule
        Nom & Temp & Long. de trajectoire(mega) & Nb de trajectoires  & Voisin & Proba \\
        \cmidrule{1-6}
        mc0   & 0.001 &  3  &  1000  & 10 & 0; 1; 0.1; 0 ;0 \\      
        mc1   & 0.1   &  3  &  1000  & 10 & 0; 1; 0.1; 0 ;0 \\  
        mc2   & 0.2   &  3  &  1000  & 10 & 0; 1; 0.1; 0 ;0 \\ 
        mc3   & 0.3   &  3  &  1000  & 10 & 0; 1; 0.1; 0 ;0 \\               
        mc4   & 0.5   &  3  &  1000  & 10 & 0; 1; 0.1; 0 ;0 \\  
        mc5   & 0.7   &  3  &  1000  & 10 & 0; 1; 0.1; 0 ;0 \\  
        mc1b  & 0.1   &  6  &  1000  & 10 & 1; 1;   1; 1 ;0 \\  
        mc2b  & 0.2   &  6  &  1000  & 10 & 0; 1; 0.1; 0 ;0 \\      
        mc2c  & 0.2   &  3  & 10000  & 10 & 0; 1; 0.1; 0 ;0 \\   
        mc2d  & 0.2   &  3000  &  1 & 10  & 0; 1; 0.1; 0 ;0 \\ 
        mc2e  & 0.2   &  3  &  1000  & 10 & 1; 0; 0.1; 0 ;0 \\     
        mc4b  & 0.5   & 10  &   100  & 10 & 0; 1;   0; 1 ;0 \\
        \cmidrule{1-6}         
        MC2d2 & 0.2   &  6000  &  1  & 10 & 0; 1; 0.1; 0 ;0 \\      
        MC2e2 & 0.2   &  6000  &  1  & 10 & 1; 0; 0.1; 0 ;0 \\   

        \bottomrule   

        
      \end{tabular}      
      \caption{Les protocoles Monte-Carlo}
\label{tab:protoMC}      
    \end{table}

   \subsection{Les protocoles Replica Exchange} 

L'algorithme ``Replica Exchange'' (RE) est une extension du Monte-Carlo , Les paramètres d'un protocole RE sont ceux d'un protocole Monte-Carlo plus d'autres paramètres:

\begin{itemize}
\item le nombre de marcheurs
\item la température pour chaque marcheur
\item la période de ``swap'', c'est-à-dire la période  (en nombre de pas) à  laquelle le test de Hasting sur l'échange de température et éventuellement l'échange ,sont effectués.
\end{itemize}
Pour avoir des exécutions en parallèle sans plusieurs marcheurs  par coeur du processeur, nous limiter nos tests à quatre ou huit marcheurs.
La distribution des températures est un éléments déterminant dans le comportement des marcheurs, car c'est elle qui pilote en grande partie le taux d'acceptation des échanges de températures. Nous suivant l'idée proposée par Kofke de lui faire suivre une progression géométrique ( $ \frac{T_i}{T_{i+1}}=C $ , avec C une constante)~\citep{refRE1,refRE2,refRE3}. Ceci garantie alors que le taux d'acceptation d'échange entre $T_i et T_{i+1}$ soit égale pour tout nos i.De plus nous souhaitons centrer à peu près, nos distributions sur la température ambiante.

Voici les températures pour le RE quatre marcheurs:

\begin{itemize} 
\item 10, 1, 0.1 et 0.01
\item 2, 1, 0.5 et 0.25 
\item 1, 0.5, 0.25 et 0.125
\end{itemize} 

,et celles pour le RE huit marcheurs:

\begin{itemize} 
\item 3 , 2 , 1.333 , 0.888 , 0.592 , 0.395 , 0.263 et 0.175 
\item 10 , 3.16 , 1 , 0.316 , 0.1 , 0.0316 , 0.01 et 0.00316
\end{itemize} 

Ici les protocoles ne se font qu'avec une seule trajectoire par marcheur. Et la contrainte du temps de calculs se comprends comme vingt-quatre heures de calculs cumulées sur tous les marcheurs.
Ainsi les longueurs de trajectoire sont définit pour le RE quatre marcheurs comme le quart d'une trajectoire MC, pour le RE huit marcheurs comme le huitième.

La table~\ref{tab:protoRE} donne les probabilités utilisées par les cinq balises qui contrôlent les modifications de la séquence/conformation à chaque pas, dans l'ordre de la liste de la section~\ref{sec:MC}. 
    
    \begin{table}[!htbp]
      \centering

      \begin{tabular}{llrrrcc}

        \toprule
        Nom & marcheurs &Temp & Traj (mega)& seuil voisin  & Proba & swap period (mega)\\
        \cmidrule{1-7}
        RE4a   & 4 & 10<->0.01    &  1500 & 10 & 1; 0; 0.1; 0 ;0 &  7.5\\  
        RE4b   & 4 & 1<->0.125    &  1500 & 10 & 1; 0; 0.1; 0 ;0 &  7.5\\  
        RE4c   & 4 & 2<->0.25     &  1500 & 10 & 1; 0; 0.1; 0 ;0 &  7.5\\  
        RE8a1  & 8 & 10<->0.00316 &  750  & 10 & 1; 0; 0.1; 0 ;0 &  1\\  
        RE8a2  & 8 & 10<->0.00316 &  750  &  0 & 1; 0; 0.1; 0 ;0 &  2.5\\  
        RE8b1  & 8 & 3<->0.175    &  750  & 10 & 1; 0; 0.1; 0 ;0 &  7.5\\
        RE8b2  & 8 & 3<->0.175    &  750  & 10 & 0; 1; 0.1; 0 ;0 &  7.5\\
        RE8b3  & 8 & 3<->0.175    &  750  & 10 & 0; 1; 0.1; 0 ;0 &  1\\
        \bottomrule

      \end{tabular}      
      \caption{Les protocoles Replica Exchange}
\label{tab:protoRE}      
    \end{table}

   \subsection{Les protocoles Toulbar2} 


Après avoir convertit nos matrices au format wcsp grâce à un script dédié,nous pouvons utiliser toulbar2.
Le protocole toulbar2 de recherche du GMEC est le suivant:
Le binaire toulbar2 de version 0.9.7.0 est lancé avec les options -l=3 -m -d: -s, ce qui correspond au paramétrage conseillé dans la documentation CDP~\citep{reftoulbar1,reftoulbar2}. Si au bout de vingt-quatre heures l'exécution est terminée, le protocole est achevé. Sinon le programme est arrêté et une seconde version (la 0.9.6.0) est lancé avec les options -l=1 -dee=1 -m -d: -s. Au bout de vingt-quatre heures si le programme n'est pas terminée, il est arrêté. La dernière séquence/conformation imprimer en sortie est collectée. Le choix de la seconde version et du paramétrage fait suite à une discussion avec monsieur Seydou Traoré.  

Toulbar2 est également capable de fournir la listes des séquences/conformations dont l'énergie est comprise entre celle du GMEC, $E_{GMEC}$ et une autre $E_{upper\_bound}$ donnée en paramètre. Pour cette fonctionnalité nous utilisons le paramétrage:  -d: -a -s -ub=$E_{upper\_bound}$.

   \subsubsection{Superfamily/SCOP} 

Superfamily est un ensemble avec d'une base de données composée de modèle de Markov cachés~\citep{refSuperfamily}

\begin{itemize}
\item Une base de données de modèles de Markov cachés, où chaque modèle représente une structure 3D d'un domaine de la classification SCOP.
\item une série de script qui annote une séquence de protéine donnée en entrée à partir des informations de la base. Ici nous utilisons uniquement l'association au modèle 3D le plus vraisemblable. 
\end{itemize}

Nous travaillons avec la base de données à la version 1.75, et en conjonction, nous utilisons SAM (version 3.5)~\citep{refSam} et HMMER (version 3.0)~\citep{refHmmer} recommandée par l'équipe de Superfamily. Le paramétrage utilisé est celui par défaut.

\subsection{Similarité par position et par séquence}
   \subsubsection{Alignements Pfam} 
La base de données Pfam (Protein families database)~\citep{refPfam} regroupe les domaines protéiques connus en famille. Chaque famille  étant représentées des alignements multiples de séquences et des profiles de modèles de Markov cachés~\citep{refPfam}. Dans la suite nous n'utiliserons que l'alignement dit ``seed'' qui ne contient q'un petit ensemble de membres représentatifs de famille. Cela correspond pour nous aux familles PF00017 (domaine SH2), PF00018  (domaine SH3) et PF00595 (domaine PDZ).


\subsection{Taux d'identité}




L'entropie par position $H_i$ est alors définit de la façon suivante.
Pour i une position dans la séquence, notons $f_{i}(a)$ la fréquence en i, de la lettre a de l'alphabet A . Alors 
$H_i=  - \sum_{a \in A} f_{i}(a) * log( f_{i}(a) ) $  

Soient S et N deux séquences d'acides aminés de même longueur l.

Le Taux d'identité $Id(S,N)$ de S par rapport N est égal au pourcentage de position où l'acide aminé est identique dans S et N. C'est-à-dire

  $ Id(S,N) =\sum_{1<i<l} \mathds{1}(s_i,n_i)$ 

avec $s_i$ et $n_i$ l'acide animé en i de S et de N respectivement, et $\mathds{1}(x,y)$ la fonction qui vaut 1 lorsque x=y et 0 sinon. 


Le score d’identité est simplement le taux d’identité en terme d’acide aminé. On peut
noter plus rigoureusement :

où xi est l’acide aminé (ou gap) à la position i de la séquence concernée, et yi l’acide
aminé (ou gap) à la position i de la séquence comparée. On définit s(xi;yi) = 1 si les deux
séquences ont le même acide aminé en position i. Sinon s(xi;yi) = 0.


    \clearpage
    \section{Résultats} 

    \subsection{Optimisation du protocole Monte-Carlo}

L'optimisation des paramètres Monte-Carlo est faite par des test sur nos neufs protéines avec toutes les positions actives.


   \subsubsection{Température et meilleures énergies} 
Nous utilisons les protocoles de mc0 à mc5, voir table~\ref{tab:ener_mc} pour évaluer d'effet du paramètre température dans la recherche de séquences/rotamères de meilleurs énergies.La table~\ref{tab_ener_mc} présente les résultats obtenus arrondis à la Kcal/mol inférieure. L'énergie proteus est l'énergie de dépliement c'est-à-dire l'énergie qu'il faut fournir à la protéine pour la déplier.Donc les meilleurs dans la table (et dans toutes les autres tables ) sont les énergies les plus grandes. Les tests aux températures les plus froides (0.001 et 0.1) donnent les meilleurs résultats  pour la majorité des protéines. Cependant la dégradation des performances est lente avec l'augmentation des températures. Et les résultats pour les températures intermédiaires (0.2 et 0.3) sont souvent très proches des meilleurs résultats.    

    \begin{table}[!htbp]
      \centering


      \begin{tabular}{ccccccc}

     
        \toprule
         test & 0.001$^a$ & 0.1$^b$ & 0.2$^c$  & 0.3$^d$ & 0.5$^e$ & 0.7$^f$  \\
        \cmidrule{1-7}
        1ABO & -270 & -270 & -270 & -271 & -281  & -289 \\      
        1CKA & -251 & -247 & -252 & -252 & -261  & -267 \\  
        1BM2 & -482 & -486 & -483 & -486 & -516  & -541 \\  
        1M61 & -480 & -481 & -483 & -485 & -506  & -523 \\  
        1O4C & -532 & -527 & -533 & -536 & -563  & -590 \\  
        1G9O & -423 & -425 & -426 & -432 & -450  & -462 \\  
        1R6J & -411 & -411 & -412 & -417 & -435  & -449 \\  

        \bottomrule        
      \end{tabular}
      

      \caption{Meilleur énergie selon la température,($^a$ le protocole mc0,$^b$ mc1,$^c$  mc2,$^d$ mc3,$^e$ mc4,$^f$ mc5 )}    
      \label{tab:ener_mc}
    \end{table}
 
   \subsubsection{Température et taux d'identité de séquences} 
\label{sec:T_et_I}
Nous poursuivons la comparaison des protocole mc0-mc5 en regardant le taux d'identité entre les séquences d'acides aminés et la séquence native. Pour cela, nous reprenons les séquences/conformations obtenues et triées selon l'énergie décroissante. Nous sélectionnons dans cette liste, les dix mille premières séquence pour chaque test. Les résultats sont présentés table~\ref{tab:ident_mc}.Nous obtenons des taux d'identités globalement compris entre 20 et 40\% , avec le plus suivant des valeurs proches de 30\%. Les résultats à température 0.2 et 0.3, sont aussi bon que ceux à température 0.1 et légèrement meilleurs que pour 0.001. Ici aussi il y a une dégradation aux température les plus hautes, faible à 0.5, plus nette à 0.7.

    \begin{table}[!htbp]
      \centering

      \begin{tabular}{ccccccc}
      
        \toprule
         test & 0.001$^a$ & 0.1$^b$ & 0.2$^c$  & 0.3$^d$ & 0.5$^e$ & 0.7$^f$  \\
        \cmidrule{1-7}
        1ABO & 33 & 33 & 33 & 32 & 32  & 30 \\      
        1CKA & 26 & 27 & 27 & 27 & 26  & 26 \\  
        1BM2 & 26 & 27 & 27 & 28 & 25  & 23 \\  
        1M61 & 40 & 41 & 41 & 41 & 41  & 39 \\  
        1O4C & 21 & 21 & 21 & 21 & 20  & 19 \\  
        1G9O & 35 & 35 & 36 & 37 & 36  & 33 \\  
        1R6J & 33 & 33 & 32 & 32 & 31  & 29 \\  
        \bottomrule
        
      \end{tabular}
      

      \caption{Pourcentage d'identité de séquences selon la température,($^a$ le protocole mc0,$^b$ mc1,$^c$  mc2,$^d$ mc3,$^e$ mc4,$^f$ mc5 )}      
      \label{tab:ident_mc}
    \end{table}



   \subsubsection{Température et résultats Superfamily} 

Nous évaluons maintenant, la similarité que peuvent avoir nos séquences putatives avec la structure 3D du domaine de la protéine. Nous lancer Superfamily sur chaque ensemble de dix milles séquences (voir \ref{sec:T_et_I}). Le tableau \ref{tab:SF_mc} présente le nombre de séquences attribuées au domaine de la séquence native par cet outil. Les résultats sont bons, sauf pour les températures les plus chaudes, la grande majorité des séquences sont attribuées au domaine natif respectif (les domaines sur en Table~\ref{tab:protéines}).Ici, les deux températures intermédiaires font quasiment jeu égal avec les deux températures les plus froides.
    \begin{table}[!htbp]
      \centering
      
      \begin{tabular}{crrrrrr}      
        \toprule
         Protéine & 0.001$^a$ & 0.1$^b$ & 0.2$^c$  & 0.3$^d$ & 0.5$^e$ & 0.7$^f$  \\
        \cmidrule{1-7}
        1ABO & 7382  & 8374 & 6764 & 5033 & 2576  & 1255  \\      
        1CKA & 8045  & 8497 & 9139 & 9534 & 8060  & 2490  \\  
        1BM2 & 8073  & 8002 & 6861 & 7869 & 4458  & 2821  \\  
        1M61 & 9489  & 9662 & 9825 & 9777 & 9822  & 8744  \\  
        1O4C & 7124  & 7702 & 6909 & 7849 & 7623  & 4847  \\  
        1G9O & 10000 & 10000 & 10000 & 10000 & 10000  & 9942 \\  
        1R6J & 9878  & 9871 & 9796 & 8794 & 5387 & 3787 \\  

        \bottomrule
        
      \end{tabular}
      

      \caption{Résultats Superfamily pour les dix mille séquences de meilleur énergie selon la température,($^a$ le protocole mc0,$^b$ mc1,$^c$  mc2,$^d$ mc3,$^e$ mc4,$^f$ mc5 )}      
      \label{tab:SF_mc}
    \end{table}



   \subsubsection{Température et entropie} 

Il apparaît assez clairement, de part le principe du test de metropolis-Hasting, que le Monte-Carlo à basse température explore une partie de l'espace plus petite qu'au haute température. Il est alors légitime de mesurer cet effet sur les ensembles de dix milles séquences obtenues. Nous utilisons un alphabet A (voir table \ref{tab:Alphabet}) réduit à six classes d'acides aminés comme proposée dans~\citep{refAlphabet} qui permet de se focaliser sur les différences physico-chimiques des acides aminés.  

L'entropie par position $H_i$ est alors définit de la façon suivante.
Pour i une position dans la séquence, notons $f_{i}(a)$ la fréquence en i, de la lettre a de l'alphabet A . Alors 
$H_i=  - \sum_{a \in A} f_{i}(a) * log( f_{i}(a) ) $  

Puis, nous calculons la moyenne sur les postions des $exp(H_i)$ pour nos tests. Les résultats sont sur le tableau \ref{tab:Entro_mc}.


    \begin{table}[!htbp]
      \centering
      
      \begin{tabular}{ccccc}
        \toprule
        acide aminé & alphabet & & acide aminé & alphabet \\
        \cmidrule(r){1-2}   \cmidrule(r){4-5}     
        L & 1 & & S & 4 \\
        V & 1 & & T & 4 \\
        I & 1 & & P & 4 \\
        M & 1 & & E & 5 \\
        C & 1 & & D & 5 \\
        F & 2 & & N & 5 \\
        Y & 2 & & Q & 5 \\
        W & 2 & & K & 6 \\
        G & 3 & & R & 6 \\
        A & 4 & & H & 6 \\
        \bottomrule                
      \end{tabular}
      \caption{Alphabet réduit}      
      \label{tab:Alphabet}

    \end{table}

Nous observons la diminution systématique de l'entropie avec cette de la température. Cela représente une moindre diversité dans les séquences obtenues pour les températures des plus froides.

    \begin{table}[!htbp]
      \centering
      
      \begin{tabular}{lllll}
        \toprule
         Protéine & 0.001$^a$ & 0.1$^b$ & 0.2$^c$  & 0.3$^d$ \\
        \cmidrule{1-5}      
        1ABO & 1.68 & 1.80 & 1.84 & 2.14  \\  
        1CKA & 1.85 & 2.06 & 2.09 & 2.13  \\ 
        1BM2 & 1.88 & 1.94 & 1.96 & 2.11  \\ 
        1M61 & 1.53 & 1.60 & 1.62 & 1.79  \\ 
        1O4C & 2.18 & 2.21 & 2.23 & 2.3   \\ 
        1G9O & 1.64 & 1.68 & 1.84 & 2.07  \\ 
        1R6J & 1.75 & 1.79 & 1.94 & 2.20  \\ 
        \bottomrule               
      \end{tabular}
      \caption{Moyennes sur les positions des exp(entropies) pour les dix mille séquences de meilleur énergie,($^a$ le protocole mc0,$^b$ mc1,$^c$  mc2,$^d$ mc3)}      
      \label{tab:Entro_mc}
    \end{table}

   \subsubsection{Trajectoire et pourcentage d' identité} 

Le nombre de pas effectués dans les protocoles Monte-Carlo est le produit du nombre de trajectoires par le longueur de trajectoire. Nous comparons le pourcentage d'identité de séquences par rapport à la séquence native pour des protocoles ne variant que par le nombre de trajectoires ou la longueur de la trajectoire. Deux ensembles sont traités, mc1 et mc1b ( mc1 mais avec des trajectoires deux fois plus grandes) d'une part et mc2, mc2b , mc2c et mc2d d'autre part, avec mc2b ayant des trajectoire deux fois plus grandes de mc2, mc2c dix fois plus de trajectoire que mc2 et le même nombre de pas mais une seule trajectoire pour mc2d par rapport mc2.
Les résultats sont visibles à la table \ref{tab:Traj_ident}.L'effet du doublement de la longueur de la trajectoire existe mais est très faible. De même l'augmentation du nombre de trajectoires, pourtant drastique, n'apporte quasiment rien. Élément intéressant, à nombre de pas identique, il n'y pas de différence notable entre le protocole mille trajectoires et celui à une seule.  

    \begin{table}[!htbp]
      \centering
      
      \begin{tabular}{ccccccc}

        \toprule
        Protéine & mc1 & mc1b & mc2  & mc2b & mc2c & mc2d  \\
        \cmidrule{1-7}      
        1ABO & 33 & 33 & 33 & 33 & 33  & 33 \\      
        1CKA & 24 & 25 & 25 & 26 & 25  & 25 \\  
        1BM2 & 26 & 27 & 27 & 27 & 27  & 27 \\  
        1M61 & 40 & 40 & 41 & 42 & 41  & 41 \\  
        1O4C & 21 & 21 & 21 & 21 & 21  & 21 \\  
        1G9O & 35 & 35 & 36 & 36 & 36  & 36 \\  
        1R6J & 33 & 33 & 32 & 32 & 33  & 33 \\  
        \bottomrule
      \end{tabular}
      

      \caption{Pourcentage d'identité en variant la longueur et le nombre de trajectoires.}      
      \label{tab:Traj_ident}
    \end{table}


   \subsubsection{Mutations et pourcentage d' identité} 
Jusqu'à présent nous avons utilisé le même mode de modification de la séquence/conformation entre chaque pas Monte-Carlo pour tous les protocoles. Il s'agît du mode qui utilise les balises <Prot\_Prot> avec une valeur à 1 et <Mut> avec une valeur à 0.1. Cela veut dire qu'à chaque pas deux rotamères sont modifiés sans changement du type de résidu et qu'une troisième position change d'acide aminé avec une probabilité de 0.1. Nous allons comparer ce mode de modification (avec les protocoles mc1 et mc2 ) avec un mode où seulement  le rotamère est changé à une position sans changement de type et une seconde position change d'acide aminé la même probabilité que précédemment, 0.1 (avec les protocoles mc1b et mc2e). Les résultats sont visibles à la table ((Table~\ref{tab:mut_ident})). Nous voyons que le en ce qui concerne le premier couple de protocole, il n'y a un déclin pour tous les tests. Pour le second il y a également une dégradation des performances globales mais elle n'est pas systématique. Il y a même une amélioration pour la protéine 1R6J.  


    \begin{table}[!htbp]
      \centering
      
      \begin{tabular}{ccccc}      
          \toprule
          Protéine & mc1 & mc1b & mc2 & mc2e \\
          \cmidrule{1-3}
          
          1ABO & 33 & 30 & 33 & 32 \\      
          1CKA & 27 & 24 & 27 & 26 \\
          1BM2 & 27 & 22 & 27 & 27 \\
          1M61 & 41 & 35 & 41 & 41 \\
          1O4C & 21 & 18 & 21 & 20 \\
          1G9O & 35 & 31 & 36 & 29 \\
          1R6J & 33 & 27 & 32 & 33 \\ 
          \bottomrule
          
        \end{tabular}
        
        \caption{Pourcentage d'identité pour deux modes de mutations. }      
\label{tab:mut_ident}        
        \end{table}


    \subsection{Comparaison protocole Monte-Carlo contre heuristique}




\captionsetup[subfigure]{font=scriptsize}

   \begin{figure}
   \begin{subfigure}[b]{\linewidth}
     \centering
          \includegraphics[width=8.45cm]{simil_bypos_1ABO_h.png}~
          \includegraphics[width=8.45cm]{simil_byseq_1ABO_h.png} 
     \caption{protocole h}
   \end{subfigure}

   \begin{subfigure}[b]{\linewidth}
     \centering
          \includegraphics[width=8.45cm]{simil_bypos_1ABO_mc2.png}~ 
          \includegraphics[width=8.45cm]{simil_byseq_1ABO_mc2.png} 
     \caption{protocole mc2}
   \end{subfigure}

   \begin{subfigure}[b]{\linewidth}
     \centering
          \includegraphics[width=8.45cm]{simil_bypos_1ABO_mc3.png}~  
          \includegraphics[width=8.45cm]{simil_byseq_1ABO_mc3.png} 
     \caption{protocole mc3}
   \end{subfigure}

     \caption{Similarité par position et par séquence pour 1ABO}
\label{grah:simil_1ABO}
   \end{figure}

   \begin{figure}
   \begin{subfigure}[b]{\linewidth}
     \centering
          \includegraphics[width=8.45cm]{simil_bypos_1CKA_h.png}~
          \includegraphics[width=8.45cm]{simil_byseq_1CKA_h.png} 
     \caption{protocole h}
   \end{subfigure}

   \begin{subfigure}[b]{\linewidth}
     \centering
          \includegraphics[width=8.45cm]{simil_bypos_1CKA_mc2.png}~ 
          \includegraphics[width=8.45cm]{simil_byseq_1CKA_mc2.png} 
     \caption{protocole mc2}
   \end{subfigure}

   \begin{subfigure}[b]{\linewidth}
     \centering
          \includegraphics[width=8.45cm]{simil_bypos_1CKA_mc3.png}~  
          \includegraphics[width=8.45cm]{simil_byseq_1CKA_mc3.png} 
     \caption{protocole mc3}
   \end{subfigure}

     \caption{Similarité par position et par séquence pour 1CKA}
\label{grah:simil_1CKA}
   \end{figure}

   \begin{figure}
   \begin{subfigure}[b]{\linewidth}
     \centering
          \includegraphics[width=8.45cm]{simil_bypos_1BM2_h.png}~
          \includegraphics[width=8.45cm]{simil_byseq_1BM2_h.png} 
     \caption{protocole h}
   \end{subfigure}

   \begin{subfigure}[b]{\linewidth}
     \centering
          \includegraphics[width=8.45cm]{simil_bypos_1BM2_mc2.png}~ 
          \includegraphics[width=8.45cm]{simil_byseq_1BM2_mc2.png} 
     \caption{protocole mc2}
   \end{subfigure}

   \begin{subfigure}[b]{\linewidth}
     \centering
          \includegraphics[width=8.45cm]{simil_bypos_1BM2_mc3.png}~  
          \includegraphics[width=8.45cm]{simil_byseq_1BM2_mc3.png} 
     \caption{protocole mc3}
   \end{subfigure}

     \caption{Similarité par position et par séquence pour 1BM2}
\label{grah:simil_1BM2}
   \end{figure}

   \begin{figure}
   \begin{subfigure}[b]{\linewidth}
     \centering
          \includegraphics[width=8.45cm]{simil_bypos_1M61_h.png}~
          \includegraphics[width=8.45cm]{simil_byseq_1M61_h.png} 
     \caption{protocole h}
   \end{subfigure}

   \begin{subfigure}[b]{\linewidth}
     \centering
          \includegraphics[width=8.45cm]{simil_bypos_1M61_mc2.png}~ 
          \includegraphics[width=8.45cm]{simil_byseq_1M61_mc2.png} 
     \caption{protocole mc2}
   \end{subfigure}

   \begin{subfigure}[b]{\linewidth}
     \centering
          \includegraphics[width=8.45cm]{simil_bypos_1M61_mc3.png}~  
          \includegraphics[width=8.45cm]{simil_byseq_1M61_mc3.png} 
     \caption{protocole mc3}
   \end{subfigure}

     \caption{Similarité par position et par séquence pour 1M61}
\label{grah:simil_1M61}
   \end{figure}

   \begin{figure}
   \begin{subfigure}[b]{\linewidth}
     \centering
          \includegraphics[width=8.45cm]{simil_bypos_1O4C_h.png}~
          \includegraphics[width=8.45cm]{simil_byseq_1O4C_h.png} 
     \caption{protocole h}
   \end{subfigure}

   \begin{subfigure}[b]{\linewidth}
     \centering
          \includegraphics[width=8.45cm]{simil_bypos_1O4C_mc2.png}~ 
          \includegraphics[width=8.45cm]{simil_byseq_1O4C_mc2.png} 
     \caption{protocole mc2}
   \end{subfigure}

   \begin{subfigure}[b]{\linewidth}
     \centering
          \includegraphics[width=8.45cm]{simil_bypos_1O4C_mc3.png}~  
          \includegraphics[width=8.45cm]{simil_byseq_1O4C_mc3.png} 
     \caption{protocole mc3}
   \end{subfigure}

     \caption{Similarité par position et par séquence pour 1O4C}
\label{grah:simil_1O4C}
   \end{figure}

   \begin{figure}
   \begin{subfigure}[b]{\linewidth}
     \centering
          \includegraphics[width=8.45cm]{simil_bypos_1G9O_h.png}~
          \includegraphics[width=8.45cm]{simil_byseq_1G9O_h.png} 
     \caption{protocole h}
   \end{subfigure}

   \begin{subfigure}[b]{\linewidth}
     \centering
          \includegraphics[width=8.45cm]{simil_bypos_1G9O_mc2.png}~ 
          \includegraphics[width=8.45cm]{simil_byseq_1G9O_mc2.png} 
     \caption{protocole mc2}
   \end{subfigure}

   \begin{subfigure}[b]{\linewidth}
     \centering
          \includegraphics[width=8.45cm]{simil_bypos_1G9O_mc3.png}~  
          \includegraphics[width=8.45cm]{simil_byseq_1G9O_mc3.png} 
     \caption{protocole mc3}
   \end{subfigure}

     \caption{Similarité par position et par séquence pour 1G9O}
\label{grah:simil_1G9O}
   \end{figure}

   \begin{figure}
   \begin{subfigure}[b]{\linewidth}
     \centering
          \includegraphics[width=8.45cm]{simil_bypos_1R6J_h.png}~
          \includegraphics[width=8.45cm]{simil_byseq_1R6J_h.png} 
     \caption{protocole h}
   \end{subfigure}

   \begin{subfigure}[b]{\linewidth}
     \centering
          \includegraphics[width=8.45cm]{simil_bypos_1R6J_mc2.png}~ 
          \includegraphics[width=8.45cm]{simil_byseq_1R6J_mc2.png} 
     \caption{protocole mc2}
   \end{subfigure}

   \begin{subfigure}[b]{\linewidth}
     \centering
          \includegraphics[width=8.45cm]{simil_bypos_1R6J_mc3.png}~  
          \includegraphics[width=8.45cm]{simil_byseq_1R6J_mc3.png} 
     \caption{protocole mc3}
   \end{subfigure}

     \caption{Similarité par position et par séquence pour 1R6J}
\label{grah:simil_1R6J}
   \end{figure}



   \begin{figure}[t]
     \centering
     \begin{tabular}{cc}
       \includegraphics[width=8.45cm]{simil_vs_ener_1ABO.png} &
       \includegraphics[width=8.45cm]{simil_vs_ener_1BM2.png} \\
       \includegraphics[width=8.45cm]{simil_vs_ener_1CKA.png} &
       \includegraphics[width=8.45cm]{simil_vs_ener_1G9O.png} \\
       \includegraphics[width=8.45cm]{simil_vs_ener_1M61.png} &
       \includegraphics[width=8.45cm]{simil_vs_ener_1O4C.png} \\
       \includegraphics[width=8.45cm]{simil_vs_ener_1R6J.png} \\
     \end{tabular}
     
     \caption{}
\label{grah:simil_vs_ener}
   \end{figure}
 



   \begin{figure}[t]
     \centering
     \begin{tabular}{cc}
       \includegraphics[width=8.45cm]{centiles_1ABO.png} &
       \includegraphics[width=8.45cm]{centiles_1BM2.png} \\
       \includegraphics[width=8.45cm]{centiles_1CKA.png} &
       \includegraphics[width=8.45cm]{centiles_1G9O.png} \\
       \includegraphics[width=8.45cm]{centiles_1M61.png} &
       \includegraphics[width=8.45cm]{centiles_1O4C.png} \\
       \includegraphics[width=8.45cm]{centiles_1R6J.png} \\
     \end{tabular}
     
     \caption{Distribution des 100000 meilleurs séquences selon l'énergie.}
\label{grah:centiles}
   \end{figure}
 


    \begin{table}[h]
      \centering

      \begin{tabular}{cccccccccc}

        \toprule
        Protéine & h & MC3 & MC43 & RE1 & RE2 & RE5 & RE3 & RE32 & RE4 \\
        \cmidrule{1-10}
        1A81 & -521 & -538 & -522 & -525 & -520 & -520 & -514 & -512 & -518 \\
        1ABO & -272 & -274 & -268 & -273 & -269 & -273 & -268 & -271 & -272 \\
        1BM2 & -484 & -500 & -486 & -488 & -481 & -489 & -478 & -476 & -486 \\
        1CKA & -252 & -258 & -249 & -259 & -251 & -251 & -247 & -246 & -249 \\
        1G9O & -428 & -435 & -428 & -429 & -421 & -430 & -428 & -425 & -428 \\
        1M61 & -480 & -493 & -479 & -483 & -480 & -481 & -480 & -480 & -480 \\
        1O4C & -535 & -545 & -531 & -536 & -529 & -536 & -527 & -524 & -532 \\
        1R6J & -407 & -419 & -414 & -415 & -409 & -411 & -409 & -408 & -414 \\
        2BYG & -457 & -469 & -454 & -461 & -456 & -460 & -456 & -454 & -462 \\
  
        \bottomrule

      \end{tabular}      
      \caption{les meilleures énergies pour tous les résidus actifs}
\label{tab:best_ener_all_all}      
    \end{table}


   \begin{figure}[t]
     \centering
     \begin{tabular}{cc}
       \includegraphics[width=12cm]{1A81_casa-p8.png} &
     \end{tabular}
     
     \caption{Distribution des énergies selon la température (protocole RE3).}
\label{graph:Distrib_E_T}
   \end{figure}


   \begin{figure}[t]
     \centering
     \begin{tabular}{cc}
       \includegraphics[width=8.45cm]{1A81-RE1-T_traj.png} &
       \includegraphics[width=8.45cm]{1A81-RE2-T_traj.png} \\
       \includegraphics[width=8.45cm]{1A81-RE4-T_traj.png} &
       \includegraphics[width=8.45cm]{1A81-RE3-T_traj.png} \\
     \end{tabular}
     \caption{Variation de la température au court de la trajectoire de chaque marcheur (protocole RE1).}
\label{graph:TRAJ_T}
   \end{figure}

    \clearpage

   \begin{figure}[t]
     \centering
     \begin{tabular}{cc}
       \includegraphics[width=18cm]{best_all.png} \\
     \end{tabular}
     \caption{Tous les protocoles.}
\label{graph:best_ener_all_all}
   \end{figure}


    \clearpage


   \begin{figure}[t]
     \centering
     \begin{tabular}{cc}
       \includegraphics[width=8cm]{best_by_cat.png} &
       \includegraphics[width=8cm]{best_MC+H.png} \\
       \includegraphics[width=8cm]{best_RE4.png} &
       \includegraphics[width=8cm]{best_RE8.png} \\
     \end{tabular}
     \caption{Variation de la température au court de la trajectoire de chaque marcheur (protocole RE1).}
\label{graph:best_ener_by_algo}
   \end{figure}


    \clearpage

   \subsection{Tous les résidus inactifs}
 
   \subsubsection{Séquence native}

 
    \begin{table}[h]
      \centering

      \begin{tabular}{ccccccc}

        \toprule
        Protéine & GMEC & H- & MC0 & MC4- \\
        \cmidrule{1-5}
        1A81 & -585.1365 & 0 & -0.2547 & 0 \\
        1ABO & -320.1798 & 0 & 0 & 0 \\
        1BM2 & -553.5532 & 0 & -0.0564 & -0.0121 \\
        1CKA & -319.2787 & 0 & 0 & 0 \\
        1G9O & -481.1175 & 0 & -0.1394 & 0 \\
        1M61 & -555.9140 & 0 & 0 & 0 \\
        1O4C & -591.2115 & 0 & 0 & -0.1250 \\
        1R6J & -454.9340 & 0 & 0 & 0 \\
        2BYG & -507.0165 & 0 & 0 & 0 \\        
        \bottomrule


      \end{tabular}      
      \caption{L’énergie du GMEC et la différence avec les autres protocoles. Tous les résidus inactifs ()}
\label{tab:result_no_active}      
    \end{table}


   \subsubsection{Une position active}


    \begin{table}[h]
      \centering

      \begin{tabular}{ccc}


        \toprule
        Position & GMEC & MC4- \\
        \cmidrule{1-3}
        14 & -584.4693 & -0.0405 \\
        39 & -584.7378 & -0.0111 \\
        55 & -584.0477 & -0.0012 \\
        60 & -583.7763 & -0.0140 \\
        66 & -592.3835 & -0.0347 \\
        70 & -583.8950 & -0.0348 \\
        71 & -588.5916 & -0.0247 \\
        76 & -583.3815 & -0.0248 \\
        79 & -582.8485 & -0.0406 \\
        86 & -584.1412 & -0.0248 \\
        101 & -583.8406 & -0.0248 \\
        105 & -583.0197 & -0.0248 \\
        107 & -582.2241 & -0.0248 \\

        \bottomrule

      \end{tabular}      
      \caption{Liste des échecs pour 1A81}
\label{tab:result_1_active_1A81}      
    \end{table}


    \begin{table}[h]
      \centering

      \begin{tabular}{ccc}


        \toprule
        Position & GMEC & MC4- \\
        \cmidrule{1-3}
        2 & -553.3134 & -0.0040 \\
        3 & -553.5532 & -0.0121 \\
        5 & -553.0932 & -0.0179 \\
        6 & -553.5532 & -0.0121 \\
        8 & -556.1917 & -0.0148 \\
        10 & -551.4990 & -0.0149 \\
        11 & -551.8859 & -0.0149 \\
        12 & -550.8152 & -0.0148 \\
        13 & -553.4829 & -0.0451 \\
        14 & -553.5532 & -0.0121 \\
        15 & -553.5532 & -0.0121 \\
        17 & -553.5532 & -0.0121 \\
        18 & -553.0880 & -0.0121 \\
        19 & -553.5532 & -0.0270 \\
        20 & -553.0003 & -0.0121 \\
        21 & -553.5532 & -0.0121 \\
        22 & -553.1769 & -0.0121 \\
        29 & -553.5532 & -0.0121 \\
        34 & -553.5532 & -0.0270 \\
        36 & -555.3358 & -0.0317 \\
        37 & -553.5532 & -0.0121 \\
        41 & -553.5076 & -0.0121 \\
        46 & -552.9056 & -0.0149 \\
        49 & -553.5532 & -0.0121 \\
        51 & -553.5532 & -0.0179 \\
        55 & -551.8384 & -0.0121 \\
        56 & -553.5532 & -0.0121 \\
        57 & -561.0695 & -0.0121 \\
        58 & -553.5532 & -0.0121 \\
        62 & -553.5532 & -0.0121 \\
        65 & -553.5532 & -0.0121 \\
        66 & -551.2026 & -0.0179 \\
        68 & -552.6182 & -0.0148 \\
        70 & -553.5532 & -0.0121 \\
        72 & -552.2724 & -0.0121 \\
        73 & -553.5532 & -0.0121 \\
        75 & -553.5532 & -0.0179 \\
        77 & -553.0234 & -0.0466 \\
        80 & -553.5532 & -0.0121 \\
        81 & -553.5532 & -0.0121 \\
        82 & -548.0641 & -0.0121 \\
        83 & -553.5532 & -0.0121 \\
        85 & -550.1884 & -0.0122 \\
        86 & -552.7375 & -0.0148 \\
        87 & -550.6139 & -0.0121 \\
        90 & -552.8601 & -0.0009 \\
        91 & -553.5532 & -0.0121 \\
        92 & -553.5532 & -0.0121 \\
        93 & -553.2772 & -0.0148 \\
        94 & -553.3207 & -0.0251 \\
        96 & -553.5532 & -0.0121 \\
        \bottomrule

      \end{tabular}      
      \caption{Liste des échecs pour 1BM2}
\label{tab:result_1_active_1BM2}      
    \end{table}




    \begin{table}[h]
      \centering

      \begin{tabular}{ccc}


        \toprule
        Position & GMEC & MC4- \\
        \cmidrule{1-3}
        17 & -316.1693 & -0.0109 \\

        \bottomrule
      \end{tabular}      
      \caption{Liste des échecs pour 1CKA}
\label{tab:result_1_active_1CKA}      
    \end{table}

    \begin{table}[h]
      \centering

      \begin{tabular}{ccc}

        \toprule
        Position & GMEC & MC4 \\
        \cmidrule{1-3}
        58 & -561.9469 & -0.0138 \\
       \bottomrule        
      \end{tabular}      
      \caption{Liste des échecs pour 1M61}
\label{tab:result_1_active_1M61}      
    \end{table}
    

    \begin{table}[h]
      \centering
      
      \begin{tabular}{ccc}
        
        \toprule
        Position & GMEC & MC4- \\
        \cmidrule{1-3}
        1 & -591.2115 & -0.1380 \\
        2 & -591.2115 & -0.1250 \\
        3 & -591.2115 & -0.1250 \\
        4 & -590.7216 & -0.0319 \\
        5 & -590.5458 & -0.1071 \\
        6 & -591.2115 & -0.1521 \\
        7 & -590.7923 & -0.1429 \\
        8 & -591.2115 & -0.1250 \\
        9 & -591.2115 & -0.1728 \\
        10 & -591.2115 & -0.2572 \\
        11 & -589.9443 & -0.2489 \\
        12 & -591.1022 & -0.1137 \\
        13 & -589.9867 & -0.0535 \\
        14 & -591.2115 & -0.1250 \\
        15 & -589.4899 & -0.0436 \\
        16 & -591.2115 & -0.1521 \\
        17 & -590.4460 & -0.0557 \\
        18 & -589.0053 & -0.1366 \\
        19 & -590.7580 & -0.0348 \\
        20 & -591.2115 & -0.1250 \\
        21 & -591.2115 & -0.1600 \\
        22 & -591.2115 & -0.1250 \\
        23 & -590.5249 & -0.1530 \\
        24 & -590.7262 & -0.0630 \\
        25 & -591.2115 & -0.1250 \\
        26 & -591.2115 & -0.1250 \\
        27 & -590.8058 & -0.1194 \\
        28 & -591.2115 & -0.1250 \\
        29 & -591.2115 & -0.1571 \\
        30 & -590.5207 & -0.0221 \\
        31 & -590.5507 & -0.0530 \\
        32 & -591.2115 & -0.1571 \\
        33 & -591.2115 & -0.1234 \\
        34 & -590.7486 & -0.1258 \\
        35 & -591.2115 & -0.0378 \\
        36 & -589.1510 & -0.0974 \\
        37 & -591.0133 & -0.0941 \\
        38 & -589.2126 & -0.2743 \\
        39 & -589.0387 & -0.1890 \\
        40 & -590.8793 & -0.0883 \\
        41 & -589.4209 & -0.0409 \\
        42 & -591.2115 & -0.1250 \\
        43 & -587.9420 & -0.1315 \\
        44 & -589.8470 & -0.0595 \\
        45 & -591.2115 & -0.1712 \\
        46 & -588.8346 & -0.2668 \\
        47 & -589.9117 & -0.2773 \\
        48 & -588.6520 & -0.2625 \\
        49 & -591.2115 & -0.2120 \\
        50 & -590.6561 & -0.0807 \\
        51 & -591.1249 & -0.2986 \\
        52 & -589.7127 & -0.2734 \\
        53 & -590.7224 & -0.3012 \\
        54 & -590.8735 & -0.3615 \\
        55 & -588.6242 & -0.2007 \\
        56 & -591.2115 & -0.2120 \\
        57 & -591.2115 & -0.1250 \\
        58 & -590.6832 & -0.0743 \\
        59 & -591.2115 & -0.0378 \\
        60 & -591.1842 & -0.1082 \\
        61 & -590.6996 & -0.1272 \\
        62 & -595.8620 & -0.0899  \\
        63 & -591.2115 & -0.0974 \\
        64 & -588.8836 & -0.1014 \\
        65 & -591.2115 & -0.1571 \\
        66 & -590.1420 & -0.1533 \\
        67 & -587.5415 & -0.0433 \\
        68 & -590.1771 & -0.1541 \\
        69 & -591.2115 & -0.1250 \\
        70 & -590.4684 & -0.1066 \\
        71 & -591.2115 & -0.1250 \\
        72 & -591.2115 & -0.1311 \\
        73 & -591.2115 & -0.1250 \\
        74 & -588.7096 & -0.1169 \\
        75 & -590.2437 & -0.0505 \\
        76 & -591.2115 & -0.1521 \\
        78 & -587.6940 & -0.0821 \\
        79 & -589.9770 & -0.1380 \\
        80 & -591.1165 & -0.0661 \\
        81 & -590.2528 & -0.1229 \\
        82 & -589.8459 & -0.0724 \\
        83 & -590.2079 & -0.0513 \\
        84 & -591.2095 & -0.0433 \\
        85 & -590.8011 & -0.1154 \\
        87 & -590.7787 & -0.0744 \\
        88 & -590.2860 & -0.0857 \\
        89 & -591.2115 & -0.1250 \\
        90 & -590.2493 & -0.1084 \\
        91 & -589.5602 & -0.0694 \\
        92 & -589.3260 & -0.1838 \\
        93 & -590.4697 & -0.0188 \\
        94 & -587.4192 & -0.3392 \\
        95 & -590.0201 & -0.2937 \\
        96 & -590.6312 & -0.2723 \\
        97 & -595.0049 & -0.2864 \\
        98 & -590.0135 & -0.2013 \\
        99 & -589.7855 & -0.2932 \\
        100 & -591.2115 & -0.2120 \\
        101 & -591.2115 & -0.2120 \\
        102 & -591.2115 & -0.2572 \\
        103 & -595.4168 & -0.1277 \\
        104 & -589.9208 & -0.3581 \\
        \bottomrule
       


      \end{tabular}      
      \caption{Liste des échecs pour 1O4C}
\label{tab:result_1_active_1O4C}
    \end{table}

    \begin{table}[h]
      \centering

      \begin{tabular}{ccc}


        \toprule
        Position & GMEC & MC4- \\
        \cmidrule{1-3}
         4 & -453.4484 & -0.0155  \\
        20 & -452.6464 & -0.0114 \\
        32 & -454.9340 & -0.0092 \\
        68 & -454.4856 & -0.0060 \\
        73 & -454.7809 & -0.0155 \\
        77 & -454.1344 & -0.0155 \\
        79 & -453.4729 & -0.0155 \\
        \bottomrule
      \end{tabular}      
      \caption{Liste des échecs pour 1R6J }
\label{tab:result_1_active_1R6J}
    \end{table}

    \begin{table}[h]
      \centering

      \begin{tabular}{ccc}

        \toprule
        Position & GMEC & MC4- \\
        \cmidrule{1-3}
        1 & -505.2910 & -0.0132 \\
        3 & -506.7960 & -0.0254 \\
        4 & -505.5800 & -0.0023 \\
        5 & -506.8732 & -0.0948 \\
        49 & -505.5183 & -0.0135 \\
        59 & -507.0165 & -0.0100 \\
        85 & -506.6217 & -0.0101 \\
        88 & -505.2286 & -0.0097 \\
        95 & -506.3195 & -0.0131 \\
        \bottomrule
      \end{tabular}      
      \caption{Liste des échecs pour 2BYG }
\label{tab:result_1_active_2BYG}
    \end{table}


   \subsubsection{Cinq positions actives}


    \begin{table}[h]
      \centering

      \begin{tabular}{ccccc}


        
        \toprule
        Protéine & GMEC & H & MC4 & RE3 \\
        \cmidrule{1-5}
        1A81 1 & -579.3989 & 0 & 0 &  \\
        1A81 2 & -575.2254 & 0 & 0 &  \\
        1A81 3 & -582.7452 & 0 & 0 &  \\
        1A81 4 & -569.9383 & 0 & -5.3443 & 0 \\
        1A81 5 & -591.8143 & 0 & 0 &  \\
        1ABO 1 & -315.4497 & 0 & 0 &  \\
        1ABO 2 & -316.6637 & 0 & 0 &  \\
        1ABO 3 & -307.4824 & 0 & 0 &  \\
        1ABO 4 & -313.7710 & 0 & 0 &  \\
        1ABO 5 & -313.5695 & 0 & 0 &  \\
        1BM2 1 & -548.2341 & 0 & 0 &  \\
        1BM2 2 & -554.8135 & 0 & 0 &  \\
        1BM2 3 & -557.8629 & 0 & 0 &  \\
        1BM2 4 & -544.9791 & 0 & 0 &  \\
        1BM2 5 & -550.2956 & 0 & -0.0121 &  \\
        1CKA 1 & -315.0859 & 0 & 0 &  \\
        1CKA 2 & -309.7692 & 0 & 0 &  \\
        1CKA 3 & -317.3820 & 0 & 0 &  \\
        1CKA 4 & -314.8550 & 0 & 0 &  \\
        1CKA 5 & -312.0405 & -0.0001 & -0.0001 &  \\
        1G9O 1 & -469.9540 & 0 & 0 &  \\
        1G9O 2 & -476.4094 & 0 & 0 &  \\
        1G9O 3 & -479.7190 & 0 & 0 &  \\
        1G9O 4 & -478.9513 & 0 & 0 &  \\
        1G9O 5 & -480.7260 & 0 & 0 &  \\
        1M61 1 & -557.6647 & 0 & 0 &  \\
        1M61 2 & -546.9587 & 0 & 0 &  \\
        1M61 3 & -553.0731 & 0 & 0 &  \\
        1M61 4 & -555.0885 & 0 & 0 &  \\
        1M61 5 & -554.6356 & 0 & 0 &  \\
        1O4C 1 & -584.4267 & 0 & -0.0655 &  \\
        1O4C 2 & -584.8989 & 0 & -0.1437 &  \\
        1O4C 3 & -588.4971 & 0 & -0.1164 &  \\
        1O4C 4 & -587.7129 & 0 & -0.1400 &  \\
        1O4C 5 & -587.6514 & 0 & -0.1168 &  \\
        1R6J 1 & -444.5018 & 0 & 0 &  \\
        1R6J 2 & -449.3043 & 0 & -0.9421 & 0 \\
        1R6J 3 & -453.1139 & 0 & 0 &  \\
        1R6J 4 & -453.1139 & 0 & 0 &  \\
        1R6J 5 & -454.9340 & 0 & 0 &  \\
        2BYG 1 & -500.7946 & 0 & -0.0150 &  \\
        2BYG 2 & -506.2319 & 0 & 0 &  \\
        2BYG 3 & -506.8744 & 0 & -0.0131 &  \\
        2BYG 4 & -504.5135 & 0 & 0 &  \\
        2BYG 5 & -506.0052 & 0 & 0 &  \\
        \bottomrule
      \end{tabular}      
 \caption{Résultats 5 position actives}
\label{tab:result_5_actives}
\end{table}

   \subsubsection{Dix positions actives}


    \begin{table}[h]
      \centering

      \begin{tabular}{cccccc}


        \toprule
        Test & GMEC & toulbar2 & H & MC & RE \\
        \cmidrule{1-6}
        1A81 1 & yes & -583.9354 & 0. & 0. & \\
        1A81 2 & yes & -581.7802 & 0. & 0. & \\
        1A81 3 & yes & -587.4392 & -0.0001 & -0.1595 & \\
        1A81 4 & yes & -589.1322 & 0. & -0.0317 & \\
        1A81 5 & yes & -578.2558 & 0. & -0.0563 & \\
        1ABO 1 & yes & -309.1670 & -0.0675 & -0.9054 & \\
        1ABO 2 & yes & -308.8387 & 0. & 0. & \\
        1ABO 3 & yes & -303.8520 & 0. & 0. & \\
        1ABO 4 & yes & -310.0087 & 0. & -0.0128 & \\
        1ABO 5 & yes & -301.6727 & 0. & 0. & \\
        1BM2 1 & yes & -549.8638 & 0. & -0.0950. & \\
        1BM2 2 & yes & -541.5944 & 0. & 0. & \\
        1BM2 3 & yes & -543.7434 & 0. & 0. & \\
        1BM2 4 & yes & -549.0453 & 0. & 0. & \\
        1BM2 5 & yes & -544.1447 & 0. & -0.1082 & \\
        1CKA 1 & yes & -305.8477 & 0. & 0. & \\
        1CKA 2 & yes & -309.9886 & 0. & 0. & \\
        1CKA 3 & yes & -304.6618 & 0. & 0. & \\
        1CKA 4 & yes & -302.4894 & 0. & 0. & \\
        1CKA 5 & yes & -299.2329 & -0.2859 & -3.2525 & 0. \\
        1G9O 1 & yes & -466.6764 & 0. & 0. & \\
        1G9O 2 & yes & -478.8797 & 0. & 0. & \\
        1G9O 3 & yes & -477.2503 & -0.1366 & 0. & \\
        1G9O 4 & yes & -470.6458 & 0. & 0. & \\
        1G9O 5 & yes & -464.8659 & 0. & -3.9599 & 0.\\
        1M61 1 & yes & -550.0699 & 0. & -0.0776 & \\
        1M61 2 & yes & -538.6026 & -3.5105 & -4.5062 & 0.3215 \\
        1M61 3 & yes & -552.2673 & 0. & 0. & \\
        1M61 4 & yes & -550.0553 & 0. & 0. & \\
        1M61 5 & yes & -553.6559 & 0. & -0.0432 & \\
        1O4C 1 & yes & -587.4665 & 0. & -0.1121 & \\
        1O4C 2 & yes & -585.8545 & 0. & -0.1046 & \\
        1O4C 3 & yes & -580.3505 & 0. & -0.1519 & \\
        1O4C 4 & yes & -587.1548 & 0. & -0.1545 & \\
        1O4C 5 & yes & -590.2650 & 0. & -0.1753 & \\
        1R6J 1 & yes & -448.8351 & 0. & -2.4022 & -0.3986 \\
        1R6J 2 & yes & -448.4631 & 0. & -1.0398 & \\
        1R6J 3 & yes & -450.3950 & 0. & -0.0106 & \\
        1R6J 4 & yes & -451.7211 & 0. & 0. & \\
        1R6J 5 & yes & -450.9943 & 0. & -0.0162 & \\
        2BYG 1 & no  & -5.7485   & -505.6397 & -0.0337 & \\
        2BYG 2 & yes & -504.7389 & 0. & 0. & \\
        2BYG 3 & yes & -504.3048 & 0. & -0.0833 & \\
        2BYG 4 & yes & -504.3466 & 0. & -0.2149 & \\
        2BYG 5 & yes & -491.6095 & 0. & 0. & \\
        
        \bottomrule


 \end{tabular}      
 \caption{Résultats 10 positions actives }
\label{tab:result_10_actives}
\end{table}

   \subsubsection{Dix positions actives, mutations}


    \begin{table}[h]
      \centering

      \begin{tabular}{ccc}

        \toprule
        Protéine & H mut nb & MC mut nb \\
        \cmidrule{1-3}
        1A81 1 & 0  & 0 \\    
        1A81 2 & 0  & 0 \\
        1A81 3 & 0  & 2 \\
        1A81 4 & 0  & 0 \\
        1A81 5 & 0  & 0 \\
        1ABO 1 & 0  & 4 \\ 
        1ABO 2 & 0  & 0 \\
        1ABO 3 & 0  & 1 \\
        1ABO 4 & 2  & 2 \\
        1ABO 5 & 0  & 0 \\
        1BM2 1 & 0  & 2 \\
        1BM2 2 & 0  & 0 \\
        1BM2 3 & 0  & 0 \\
        1BM2 4 & 0  & 1 \\
        1BM2 5 & 0  & 2 \\
        1CKA 1 & 0  & 0 \\
        1CKA 2 & 0  & 1 \\
        1CKA 3 & 0  & 0 \\
        1CKA 4 & 0  & 0 \\
        1CKA 5 & 5  & 3 \\
        1G9O 1 & 0  & 0 \\
        1G9O 2 & 0  & 0 \\
        1G9O 3 & 0  & 0 \\
        1G9O 4 & 0  & 0 \\
        1G9O 5 & 0  & 3 \\
        1M61 1 & 0  & 2 \\
        1M61 2 & 3  & 7 \\
        1M61 3 & 0  & 0 \\
        1M61 4 & 0  & 0 \\
        1M61 5 & 0  & 0 \\
        1O4C 1 & 0  & 0 \\
        1O4C 2 & 0  & 0 \\
        1O4C 3 & 0  & 0 \\
        1O4C 4 & 0  & 0 \\
        1O4C 5 & 0  & 3 \\
        1R6J 1 & 0  & 3 \\
        1R6J 2 & 0  & 2 \\
        1R6J 3 & 0  & 0 \\
        1R6J 4 & 0  & 0 \\
        1R6J 5 & 0  & 0 \\
        2BYG 1 & no & no \\ 
        2BYG 2 & 0  & 0 \\
        2BYG 3 & 0  & 1 \\
        2BYG 4 & 1  & 3 \\
        2BYG 5 & 0  & 0 \\        
        \bottomrule

 \end{tabular}      
 \caption{Mutations 10 positions actives }
\label{tab:mutations_10_actives}
\end{table}

   \subsubsection{Vingt et trente positions actives}


    \begin{table}[h]
      \centering

      \begin{tabular}{cccccc}


        \toprule
        Test & GMEC & toulbar2 & H & MC & RE \\
        \cmidrule{1-6}
        1A81 1 & yes  &  -566.9106 & 0. & -0.3275 & -0.3851 \\         
        1A81 2 & yes  &  -564.6618 & -0.1705 & -2.4355 & -1.0069 \\   
        1A81 3 & yes  &  -572.7774 & 0. & -0.4640 & -0.6186 \\         
        1A81 4 & yes  &  -572.9780 & -0.3878 & -0.5748 & -0.6991 \\    
        1A81 5 & yes  &  -572.7410 & -0.0068 & -0.5088 & -0.1541 \\    
        1ABO 1 & yes  &  -299.6592 & -0.1205 & -1.1159 & -0.2153 \\   
        1ABO 2 & no   &  -13.8563  & -298.3854 & 0. & 0. \\               
        1ABO 3 & no   &  -1.2190   & -298.3854 & 0. & 0. \\                 
        1ABO 4 & no   &  -1.9940   & -297.8545 & -0.0076 & 0. \\            
        1ABO 5 & no   &  -3.5418   & -297.8009 & -0.9483 & -0.9483 \\       
        1BM2 1 & yes  &  -526.0936 & 0. & -0.0619 & -0.1584 \\         
        1BM2 2 & no   &  -7.5304   & -525.3588 & -0.0725 & -0.0143 \\     
        1BM2 3 & yes  &  -534.3861 & -0.0229 & -0.4762 & -0.2897 \\    
        1BM2 4 & no   &  -0.1186   & -526.8307 & -2.5883 & -0.0789 \\     
        1BM2 5 & yes  &  -535.3334 & -0.2396 & -0.3746 & -0.3746 \\    
        1CKA 1 & yes  & -295.8571  & 0.& 0. & 0. \\                   
        1CKA 2 & yes  & -295.3571  & 0. & 0. & 0. \\                   
        1CKA 3 & yes  & -293.8687  & 0. & 0. & 0.\\                   
        1CKA 4 & no   &  -4.3122   & -293.8687 & 0. & 0. \\               
        1CKA 5 & no   &  -4.2849   & -293.4203 & 0. & 0. \\           
        1G9O 1 & no   &  -2.0574   & -451.4604 & -1.2525 & -1.2525 \\ 
        1G9O 2 & no   &  -3.2106   & -453.2474 & -0.2177 & -0.1915 \\ 
        1G9O 3 & no   &  -1.9008   & -453.7856 & -0.4417 & -0.1019 \\ 
        1G9O 4 & no   &  -0.5030   & -456.7331 & -0.3855 & -0.1455 \\ 
        1G9O 5 & no   &  -0.4298   & -456.9981 & -0.1495 & -0.5114 \\ 
        1M61 1 & yes  & -528.0700 & 0. & 0. & 0. \\               
        1M61 2 & yes  & -528.7653 & 0. & 0. & 0. \\               
        1M61 3 & yes  & -530.0684 & 0. & 0. & 0. \\               
        1M61 4 & yes  & -534.5248 & 0. & 0. & 0.\\               
        1M61 5 & yes  & -548.0096 & 0. & -0.2521 & -0.1345 \\     
        1O4C 1 & no   &  -574.0047 & -0.3465 & -0.0690 & -0.0587 \\    
        1O4C 2 & no   &  -6.4214   & -574.8584 & -0.1963 & -0.3175 \\         
        1O4C 3 & yes  & -573.6314 &  0. & -0.3461 & -0.0997 \\             
        1O4C 4 & yes  & -575.8667 &  0. & -0.3640 & -0.1382 \\             
        1O4C 5 & no   & -573.3479 &  0. & -0.1131 & -0.2206 \\      
        1R6J 1 & yes  & -440.7417 &  0. & -0.2604 & -0.2002 \\        
        1R6J 2 & yes  & -437.2537 &  0. & -0.0071 & -0.0183 \\        
        1R6J 3 & yes  & -439.4335 &  0. & -0.0537 & -0.0732 \\       
        1R6J 4 & yes  & -439.5988 &  0. & -0.0639 & -0.0601 \\        
        1R6J 5 & yes  & -438.0222 &  0. & -0.0735 & -0.0244 \\        
        2BYG 1 & yes  & -496.2991 &  0. & -3.1878 & -0.0257 \\        
        2BYG 2 & yes  & -494.8723 &  0. & -0.0524 & -0.0831 \\        
        2BYG 3 & yes  & -494.4390 &  0. & -1.3564 & -0.0826 \\        
        2BYG 4 & yes  & -495.9213 &  0. & -0.1968 & -0.6022 \\        
        2BYG 5 & no   &  -1.8604   & -497.5123 & -0.0933 & -0.0386 \\   
       \bottomrule


 \end{tabular}   
 \caption{Résultats 20 positions actives }
\label{tab:result_20_actives}   
\end{table}


\begin{table}[h]
      \centering

      \begin{tabular}{cccccc}


        \toprule
        Protéine & GMEC & toulbar2 & H & MC & RE \\
        \cmidrule{1-6}
        1A81 1 & no & -1.2074    &    -562.9572 & -0.6353 &  \\          
        1A81 2 & no & -2.5520    &    -570.2620 & -0.0578 & \\          
        1A81 3 & no & -43.5263   &    -562.9572 & -2.4996 & -1.2025 \\         
        1A81 4 & no & -5.1300    &    -559.6145 & -0.0305 & \\          
        1A81 5 & no & -3.2417    &    -553.1077 & -1.9586 & -0.5791\\         
        1ABO 1 & no & -44.5504   &    -296.5680 & 0. & \\                   
        1ABO 2 & no & -12.7303   &    -294.8500 & 0. & \\                   
        1ABO 3 & no & -9.3870    &    -295.2689 & -0.2630 & \\          
        1ABO 4 & no & -10.7691   &    -296.5680 & 0. & \\                   
        1ABO 5 & no & -4.3907    &    -296.5680 & 0. & \\                    
        1BM2 1 & no & -22.5876   &    -556.1168 & -1.7290 & -1.6013 \\         
        1BM2 2 & no & -22.1386   &    -556.7539 & -1.9856 & -1.5876 \\     
        1BM2 3 & no & -22.5410   &    -556.1168 & -1.9990 & -1.1541 \\
        1BM2 4 & no & -15.2639   &    -556.8507 & -2.2127 & -2.3854 \\         
        1BM2 5 & no & -15.9890   &    -556.3240 & -2.83542 & -1.1937 \\        
        1CKA 1 & no & -6.2700    &    -293.4203 & 0. & \\                    
        1CKA 2 & no & -2.0995    &    -293.4203 & 0. & \\                    
        1CKA 3 & no & -47.0217   &    -291.9243 & 0. & \\                   
        1CKA 4 & no & -44.0830   &    -293.4203 & 0. & \\                   
        1CKA 5 & no & -8.8608    &    -293.2709 & 0. & \\                    
        1G9O 1 & no & -2.0816    &    -449.0890 & -1.5942 & 0. \\               
        1G9O 2 & no & -0.3270    &    -452.6676 & -0.3126 & \\          
        1G9O 3 & no & -17.7150   &    -450.0341 & -1.5667 & -1.5667 \\         
        1G9O 4 & no & -2.9758    &    -453.9682 & -1.4284 & -1.6202 \\          
        1G9O 5 & no & -445.8910  &    -1.6890 & -7.6985 & -2.3857 \\          
        1M61 1 & no & -14.4935   &    -0.0097 & -523.9321 & 0. \\            
        1M61 2 & no & -5.0899    &    -531.3717 & -1.8749 & -0.0083 \\         
        1M61 3 & no & -3.5795    &    -527.2659 & -0.0154 & \\           
        1M61 4 & no & -16.1511   &    -530.2666 & 0. & \\              
        1M61 5 & no & -23.0927   &    -522.5696 & 0. & \\                 
        1O4C 1 & no & -14.9064   &    -571.4882 & -0.3435 & \\         
        1O4C 2 & no & -58.1558   &    -570.1458 & -0.0795 & \\         
        1O4C 3 & no & -9.9221    &    -569.9777 & -0.1789 & \\          
        1O4C 4 & no & -5.7790    &    -568.9839 & -0.0423 & \\          
        1O4C 5 & no & -9.9221    &    -569.9777 & -0.1789 & \\          
        1R6J 1 & yes& -435.4258  &     0.0 & -0.0246 & \\               
        1R6J 2 & no & -14.9800   &    -435.0087 & -0.0957 & \\         
        1R6J 3 & no & -439.8187  &    -439.8187 & -0.0440 & \\               
        1R6J 4 & no &  -435.0087  &    -0.0 & -0.0957 & \\               
        1R6J 5 & no & -435.0970  &    -0.7036 & -1.8823 & -0.0781 \\          
        2BYG 1 & no & -17.9752   &    -492.6879 & -0.1592 & \\         
        2BYG 2 & no & -0.3832    &    -492.3568 & -0.1502 & \\          
        2BYG 3 & no & -0.1442    &    -492.6879 & -0.1593 & \\          
        2BYG 4 & no & -492.6821 &    -0.0958 & -0.0050 & \\          
        2BYG 5 & no & -0.5003    &   -492.1595 & -0.6876 & \\           
       \bottomrule


 \end{tabular}      
\label{tab:result_echec_30}      
\end{table}

    \subsection{Les temps de calculs} 
    
    \begin{figure}[h]
      \centering
      \begin{tabular}{cc}
        \includegraphics[width=12cm]{temps_de_calculs.png} &
      \end{tabular}
      
      \caption{Temps d'occupation du processeur selon le nombre de positions actives.}
\label{graph:temps_CPU}
    \end{figure}


   \subsection{Etude au voisinnage de GMECs}


    \begin{table}[h]
      \centering

      \begin{tabular}{ccccc}


        \toprule
        Protein & GMEC & H & MC & RE \\
        \cmidrule{1-5}
        1CKA 3 & -304.6618 & 0 & 0 & \\
        1CKA 4 & -302.4894 & 0 & 0 & \\
        1CKA 5 & -299.2329 & -0.2859 & -3.2525 & 0 \\
        1G9O 3 & -477.2503 & -0.1366 & 0 & \\
        1G9O 4 & -470.6458 & 0 & 0 & \\
        1G9O 5 & -464.8659 & 0 & -3.9599 &  0 \\
        1M61 1 & -550.0699 & 0 & -0.0776 & \\
        1M61 2 & -538.6026 & -3.5105 & -4.5062 & 0.3215 \\
        1M61 5 & -553.6559 & 0 & -0.0432 & \\
        \bottomrule       
      \end{tabular}      
\label{tab:voisinnage_rappel}      
    \end{table}


    \begin{table}[h]
      \centering

      \begin{tabular}{ccccccc}


        \toprule
        Protein & seq-rot nb gmec+1 & H rank  & MC rank  & seq nb gmec+1 & H mut nb & MC mut nb \\
        \cmidrule{1-7}
        1CKA 3 & 67669 & 1 & 1 & 227 & 0 & 0 \\
        1CKA 4 & 4649 & 1 & 1 & 498 & 0 & 0 \\
        1CKA 5 & 1388 & 78 & ? & 77 & 0 & 2 \\
        1G9O 3 & 354559 & 23 & 1 & 63 & 1 & 0 \\
        1G9O 4 & 22639 & 1 & 1 & 381 & 0 & 0 \\
        1G9O 5 & 8658395 & 1 & ? &  11 & 0 & 3 \\
        1M61 1 & 11199153 & ? & ? & 21 & 3 & 7 \\
        1M61 2 & 11199153 & 1 & 1 & 88 & 0 & 0 \\
        1M61 5 & 16417604 & 1 & 1 & 83 & 0 & 0 \\
        \bottomrule
      \end{tabular} 
\label{tab:etude_au_voisinnage}           
\end{table}


    \clearpage
    
    \begin{figure}[h]
      \centering
      \begin{tabular}{c} 
        \includegraphics[width=14cm]{1M61_2_gmec_vs_ph.png} 
      \end{tabular}
      
      \caption{test: 1M61 2, GMEC vs H}
\label{image:1M61_2_GMEC_vs_H}
    \end{figure}
    
        
    \begin{figure}[h]
      \centering
      \begin{tabular}{c} 
        \includegraphics[width=14cm]{1M61_2_gmec_vs_RE.png} 
      \end{tabular}
      
      \caption{test: 1M61 2, GMEC vs RE}
\label{image:1M61_2_GMEC_vs_RE}
    \end{figure}
    
        
    \begin{figure}[h]
      \centering
      \begin{tabular}{c} 
        \includegraphics[width=14cm]{1G9O_3_gmec_vs_ph.png} 
      \end{tabular}
      
      \caption{test: 1G9O 3, GMEC vs H}
\label{image:1G9O_3_GMEC_vs_H}
    \end{figure}
    
    
    \begin{figure}[h]
      \centering
      \begin{tabular}{c} 
        \includegraphics[width=14cm]{1G9O_5_gmec_vs_MC.png} 
      \end{tabular}
      
      \caption{test: 1G9O 5, GMEC vs MC}
\label{image:1G9O_5_GMEC_vs_MC}
    \end{figure}
    
    \begin{figure}[h]
      \centering
      \begin{tabular}{c} 
        \includegraphics[width=14cm]{1CKA_5_gmec_vs_ph.png} 
      \end{tabular}
      
      \caption{test: 1CKA 5, GMEC vs H}
\label{image:1CKA_5_GMEC_vs_H}
    \end{figure}

    \begin{figure}[h]
      \centering
      \begin{tabular}{c} 
        \includegraphics[width=14cm]{1CKA_5_gmec_vs_MC.png} 
      \end{tabular}
      
      \caption{test: 1CKA 5, GMEC vs MC}
\label{image:1CKA_5_GMEC_vs_MC}
    \end{figure}
    
    \clearpage


    \begin{figure}[h]
      \centering
      \begin{tabular}{c} 
        \includegraphics[width=14cm]{align_1CKA.png} 
      \end{tabular}
      
      \caption{Aligmenent du voisinnage: 1CKA 5}
\label{image:Align_Suboptimal}
    \end{figure}




   \subsection{Résultats Superfamily}


    \begin{table}[h]
           \raggedleft{}

      \begin{tabular}{cccccc}

        \toprule
        Protein & Match/seq size & Superfamily Evalue & superfamily success & Family Evalue & family success\\
        \cmidrule{1-6}
        1A81 & no & & & & \\
        1ABO & 51/58 & 4.4e-4 & 100\% & 2.8e-3 & 100\% \\
        1BM2 & 78/98 & 4.2e-5 & 100\% & 2.6e-3 & 100\% \\
        1CKA & 40/57 & 1.1e-5 & 100\% & 3.4e-3 & 100\% \\
        1G9O & 79/91 & 7.0e-7 & 100\% & 2.5e-3 & 100\%  \\
        1M61 & 97/109 & 7.2e-7 & 100\% & 2.6e-4 &  100\% \\
        1O4C & 95/104 & 2.1e-4 & 100\% & 4.5e-3 &  100\% \\
        1R6J & 74/82 & 9.8e-6 & 100\% & 4.6e-3 &  100\% \\
        2BYG & 59/97 & 1.4e-5 & 100\% & 7.1e-3 &  100\% \\
        \bottomrule        
      \end{tabular}      

\label{tab:superfamily_bestRE}       
\end{table}

    \clearpage



   \subsection{Résultats Heuristic (protocoles longs)}


    \begin{table}[h]
      \centering

      \begin{tabular}{ccccc}

        \toprule
        Proteins & GMEC & H & H+ & H++ \\
        \cmidrule{1-5}
        1ABO 1 & -309.1670 & -0.0675 & -0.0675 & 0 \\
        1CKA 5 & -299.2329 & -0.2859 & -0.0640 & 0 \\
        1G9O 3 & -477.2503 & -0.1366 & 0 & 0 \\
        1M61 2 & -538.6026 & -3.5105 & -2.1673 & -0.0188 \\
        \toprule


 \end{tabular}      
 \caption{Résultats pour 3 fois (resp 9 fois)plus de cycles heuristiques protocole H+ (resp H++)}
\label{tab:H+_H++}       
\end{table}


    \clearpage

   \subsection{densité en séquences }

    \begin{figure}[h]
      \centering
      \begin{tabular}{cc} 
        \includegraphics[width=10cm]{histo1_aa_Tambiante.png} &
      \end{tabular}
      
      \caption{.}
\label{graph:densité_en_séquences1}
    \end{figure}


    \begin{figure}[h]
      \centering
      \begin{tabular}{cc} 
        \includegraphics[width=10cm]{histo2_aa_Tambiante.png} &
      \end{tabular}
      
      \caption{.}
\label{graph:densité_en_séquences2}
    \end{figure}

    \clearpage




%%% Local Variables:
%%% mode: latex
%%% TeX-master: "../../rapport"
%%% End:

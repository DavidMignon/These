\paragraph{Protocoles Monte-Carlo}


Il y a deux ensembles de protocoles Monte-Carlo.Dans le premier, les noms  sont de la forme «mc*». Il rassemble les protocoles utilisés pour le paramétrage du Monte-Carlo.Le second est constitué des protocoles utilisés lors des comparaisons,les noms sont de la forme «MC*».  


Les éléments à paramétrer pour l'algorithme Monte-Carlo sont:

\begin{enumerate}
\item la température
\item le nombre de pas (avec le nombre de trajectoires et la longueur de trajectoire )
\item Le seuil de voisinage
\item Les probabilités de changements de la séquence/conformation
\end{enumerate}

Il y a cinq balises dans proteus qui contrôle ces changements:

\begin{description}
\item[<Prot>] donne la probabilité de modifications de  rotamère à une position.
\item[<Prot\_Prot>] donne la probabilité de modifications de  rotamère à deux positions.
\item[<Mut>] donne la probabilité de modifications de type de résidu à une position.
\item[<Mut\_Prot>] donne la probabilité de modifications de  rotamères à deux positions.
\item[<Mut\_Mut>] donne la probabilité de modifications de type de résidu à deux positions.
\end{description}

La table~\ref{tab:protoMC} donne les probabilités utilisées par ces cinq paramètres dans l'ordre de la liste précédente. 


Pour la partie comparaison avec les autres algorithmes, quatre protocoles sont utilisés. Les protocoles MCa et MCb s'inspirent fortement de mc2d et mc2e , en étant adapté à la contrainte du temps de calcul de la comparaison et en utilisant la nouvelle version de proteus.



    \begin{table}[!htbp]
      \centering

      \begin{tabular}{llrrcc}

        \toprule
        Nom & Temp & Long. de trajectoire(mega) & Nb de trajectoires  & Voisin & Proba \\
        \cmidrule{1-6}
        mc0   & 0.001 &  3  &  1000  & 10 & 0; 1; 0.1; 0 ;0 \\      
        mc1   & 0.1   &  3  &  1000  & 10 & 0; 1; 0.1; 0 ;0 \\  
        mc2   & 0.2   &  3  &  1000  & 10 & 0; 1; 0.1; 0 ;0 \\ 
        mc3   & 0.3   &  3  &  1000  & 10 & 0; 1; 0.1; 0 ;0 \\               
        mc4   & 0.5   &  3  &  1000  & 10 & 0; 1; 0.1; 0 ;0 \\  
        mc5   & 0.7   &  3  &  1000  & 10 & 0; 1; 0.1; 0 ;0 \\  
        mc1b  & 0.1   &  6  &  1000  & 10 & 1; 1;   1; 1 ;0 \\  
        mc2b  & 0.2   &  6  &  1000  & 10 & 0; 1; 0.1; 0 ;0 \\      
        mc2c  & 0.2   &  3  & 10000  & 10 & 0; 1; 0.1; 0 ;0 \\   
        mc2d  & 0.2   &  3000  &  1 & 10  & 0; 1; 0.1; 0 ;0 \\ 
        mc2e  & 0.2   &  3  &  1000  & 10 & 1; 0; 0.1; 0 ;0 \\     
        mc4b  & 0.5   & 10  &   100  & 10 & 0; 1;   0; 1 ;0 \\
        \cmidrule{1-6}
        MC0   & 0.01  &  1000  &  1  & 10 & 1; 0; 0.1; 0 ;0 \\            
        MCa   & 0.2   &  6000  &  1  & 10 & 1; 0; 0.1; 0 ;0 \\   
        MCa-  & 0.2   &  1000  &  1  & 10 & 1; 0; 0.1; 0 ;0 \\   
        MCb   & 0.2   &  6000  &  1  & 10 & 0; 1; 0.1; 0 ;0 \\      

        \bottomrule   
        
      \end{tabular}      
      \caption{Les protocoles Monte-Carlo}
\label{tab:protoMC}      
    \end{table}


   \paragraph{Protocoles "Replica Exchange"} 

Les paramètres d'un protocole RE sont ceux d'un protocole Monte-Carlo plus trois autres:

\begin{itemize}
\item le nombre de marcheurs
\item la température pour chaque marcheur
\item la période de «swap»
\end{itemize}
Pour les températures,nous suivons l'idée proposée par Kofke de lui faire suivre une progression géométrique ( $ \frac{T_i}{T_{i+1}}=C $ , avec C une constante)~\citep{refRE1,refRE2,refRE3}. Ceci garantie alors que le taux d'acceptation d'échange entre $T_i et T_{i+1}$ soit égale pour tout nos i.De plus, nous souhaitons centrer approximativement, nos distributions sur la température ambiante (environ 0.6 kcal/mol). 

Voici les températures pour le RE quatre marcheurs:

\begin{itemize} 
\item 10, 1, 0.1 et 0.01
\item 2, 1, 0.5 et 0.25 
\item 1, 0.5, 0.25 et 0.125
\end{itemize} 

,et celles pour le RE huit marcheurs:

\begin{itemize} 
\item 3 , 2 , 1.333 , 0.888 , 0.592 , 0.395 , 0.263 et 0.175 
\item 10 , 3.16 , 1 , 0.316 , 0.1 , 0.0316 , 0.01 et 0.00316
\end{itemize} 

    \begin{table}[!htbp]
      \centering

      \begin{tabular}{llrrrcc}

        \toprule
        Nom & marcheurs &Temp & Traj (mega)& seuil voisin  & Proba & swap period (mega)\\
        \cmidrule{1-7}
        RE4a   & 4 & 10-0.01    &  1500 & 10 & 1; 0; 0.1; 0 ;0 &  7.5\\  
        RE4b   & 4 & 1-0.125    &  1500 & 10 & 1; 0; 0.1; 0 ;0 &  7.5\\  
        RE4c   & 4 & 2-0.25     &  1500 & 10 & 1; 0; 0.1; 0 ;0 &  7.5\\  
        RE8a1  & 8 & 10-0.00316 &  750  & 0  & 1; 0; 0.1; 0 ;0 &  2.5\\  
        RE8a2  & 8 & 10-0.00316 &  750  & 10 & 1; 0; 0.1; 0 ;0 &  2.5\\  
        RE8b1  & 8 & 3-0.175    &  750  & 10 & 0; 1; 0.1; 0 ;0 &  7.5\\
        RE8b2  & 8 & 3-0.175    &  750  & 10 & 1; 0; 0.1; 0 ;0 &  7.5\\
        RE8b3  & 8 & 3-0.175    &  750  & 10 & 1; 0; 0.1; 0 ;0 &  1\\
        \bottomrule

      \end{tabular}      
      \caption{Les protocoles «Replica Exchange»}
\label{tab:protoRE}      
    \end{table}


   \paragraph{Protocoles Toulbar2} 
\label{proto_toulbar2}

Après avoir converti nos matrices au format «wcsp» grâce à un script dédié,
Le protocole de recherche du GMEC est le suivant:
L'exécutable toulbar2 de version 0.9.7.0 est lancé avec les options « -l=3 -m -d: -s», ce qui correspond au paramétrage conseillé dans la documentation CDP~\citep{reftoulbar1,reftoulbar2}. Si l'exécution se termine en moins de vingt-quatre heures, le protocole est achevé. Sinon le programme est arrêté et une seconde version (la 0.9.6.0) est lancée avec les options «-l=1 -dee=1 -m -d: -s». Au bout de vingt-quatre heures si le programme n'est pas terminé, il est arrêté. La dernière séquence/conformation imprimée en sortie est collectée. 

Toulbar2 en mode «Suboptimal» fournir la liste des séquences/conformations dont l'énergie est comprise entre celle qui correspond au GMEC, $E_{GMEC}$ et une autre $E_{upper\_bound}$ donnée en paramètre. Pour cela on utilise le paramétrage:  «-d: -a -s -ub=$E_{upper\_bound}$ ».La mémoire que toulbar2 peut allouer est limité à 30 Go.



   \paragraph{résultats 20 positions actives} 



    \begin{table}[h]
      \centering
\scalebox{0.8}{
      \begin{tabular}{ccccccc}


        \toprule
        test & GMEC & toulbar2 & H & MCb & RE8b2 & mut nb(toulbar2 vs H)\\
        \cmidrule{1-6}
        \cmidrule{7-7}
        1A81 1 & yes*  &  -566.9106 & 0.         & -0.3275 & -0.3851 & 0 \\         
        1A81 2 & yes*  &  -564.6618 & -0.1705    & -2.4355 & -1.0069 & 3 \\   
        1A81 3 & yes  &  -572.7774 & 0.         & -0.4640 & -0.6186 & 0 \\         
        1A81 4 & yes  &  -572.9780 & -0.3878    & -0.5748 & -0.6991 & 4 \\    
        1A81 5 & yes  &  -572.7410 & -0.0068    & -0.5088 & -0.1541 & 4 \\    
        1ABO 1 & yes  &  -299.6592 & -0.1205    & -1.1159 & -0.2153 & 2 \\   
        1ABO 2 & no*   &  -13.8563  & -298.3854  & 0.      & 0. & 8 \\               
        1ABO 3 & no*   &  -1.2190   & -298.9674  & 0.      & 0. & 9 \\                 
        1ABO 4 & no*   &  -1.9940   & -297.8545  & -0.0076 & 0. & 5 \\            
        1ABO 5 & no*   &  -3.5418   & -297.8009  & -0.9483 & -0.9483 & 9 \\       
        1BM2 1 & yes  &  -526.0936 & 0.         & -0.0619 & -0.1584 & 0 \\         
        1BM2 2 & no*   &  -7.5304   & -525.3588  & -0.0725 & -0.0143 & 8 \\     
        1BM2 3 & yes  &  -534.3861 & -0.0229    & -0.4762 & -0.2897 & 0 \\    
        1BM2 4 & no*   &  -0.1186   & -526.8307  & -2.5883 & -0.0789 & 2 \\     
        1BM2 5 & yes  &  -535.3334 & -0.2396    & -0.3746 & -0.3746 & 3 \\    
        1CKA 1 & yes*  & -295.8571  & 0.         & 0.      & 0. & 0 \\                   
        1CKA 2 & yes  & -295.3571  & 0.         & 0.      & 0. & 0 \\                   
        1CKA 3 & yes  & -293.8687  & 0.         & 0.      & 0. & 0 \\                   
        1CKA 4 & no*   &  -4.3122   & -293.8687  & 0.      & 0. & 4 \\               
        1CKA 5 & no*   &  -4.2849   & -293.4203  & 0.      & 0. & 3 \\           
        1G9O 1 & no*   &  -2.0574   & -451.4604  & -1.2525 & -1.2525 & 5 \\ 
        1G9O 2 & no*   &  -3.2106   & -453.2474  & -0.2177 & -0.1915 & 1 \\ 
        1G9O 3 & no*   &  -1.9008   & -453.7856  & -0.4417 & -0.1019 & 1 \\ 
        1G9O 4 & no*   &  -0.5030   & -456.7331  & -0.3855 & -0.1455 & 5 \\ 
        1G9O 5 & no*   &  -0.4298   & -456.9981  & -0.1495 & -0.5114 & 5 \\ 
        1M61 1 & yes  & -528.0700 & 0.          & 0.      & 0. & 0 \\               
        1M61 2 & yes  & -528.7653 & 0.          & 0.      & 0. & 0 \\               
        1M61 3 & yes  & -530.0684 & 0.           & 0.      & 0. & 0 \\               
        1M61 4 & yes  & -534.5248 & 0.           & 0.     & 0.& 0 \\               
        1M61 5 & yes  & -548.0096 & 0.           & -0.2521 & -0.1345 & 0 \\     
        1O4C 1 & no*  &  -574.0047 & -0.3465     & -0.0690 & -0.0587 & 6 \\    
        1O4C 2 & no*   &  -6.4214   & -574.8584   & -0.1963 & -0.3175 & 4 \\         
        1O4C 3 & yes  & -573.6314 &  0.          & -0.3461 & -0.0997 & 0 \\             
        1O4C 4 & yes  & -575.8667 &  0.          & -0.3640 & -0.1382 & 0 \\             
        1O4C 5 & no*   & -573.3479 &  0.          & -0.1131 & -0.2206 & 0 \\      
        1R6J 1 & yes  & -440.7417 &  0.          & -0.2604 & -0.2002 & 0 \\        
        1R6J 2 & yes  & -437.2537 &  0.          & -0.0071 & -0.0183 & 0 \\        
        1R6J 3 & yes  & -439.4335 &  0.          & -0.0537 & -0.0732 & 0 \\       
        1R6J 4 & yes  & -439.5988 &  0.          & -0.0639 & -0.0601 & 0 \\        
        1R6J 5 & yes  & -438.0222 &  0.          & -0.0735 & -0.0244 & 0 \\        
        2BYG 1 & yes  & -496.2991 &  0.          & -3.1878 & -0.0257 & 0 \\        
        2BYG 2 & yes  & -494.8723 &  0.          & -0.0524 & -0.0831 & 0 \\        
        2BYG 3 & yes*  & -494.4390 &  0.          & -1.3564 & -0.0826 & 0 \\        
        2BYG 4 & yes  & -495.9213 &  0.          & -0.1968 & -0.6022 & 0 \\        
        2BYG 5 & no*   &  -1.8604   & -497.5123   & -0.0933 & -0.0386 & 2 \\   
       \bottomrule


 \end{tabular}
}   
\caption{Résultats  avec vingt positions actives }
\label{tab:result_20_actives}   
\end{table}



\bigskip

{\raggedright

$^*$ utilisation du second protocole toulbar2

}


   \paragraph{Résultats pour des protocoles plus longs}

    \begin{table}[h]
      \centering

      \begin{tabular}{ccccc}

        \toprule
        Proteins & GMEC & H & H+ & H++ \\
        \cmidrule{1-5}
        1ABO 1 & -309.1670 & -0.0675 & -0.0675 & 0. \\
        1CKA 5 & -299.2329 & -0.2859 & -0.0640 & 0. \\
        1G9O 3 & -477.2503 & -0.1366 & 0. & 0. \\
        1M61 2 & -538.6026 & -3.5105 & -2.1673 & -0.0188 \\
        \toprule


 \end{tabular}      
 \caption{Résultats pour 3 fois (resp 9 fois)plus de cycles heuristiques protocole H+ (resp H++)}
\label{tab:H+_H++}       
\end{table}


    \begin{table}[h]
      \centering

      \begin{tabular}{ccccc}

        \toprule
        Test (positions actives)& RE8b2 & RE8b2+ \\
        \cmidrule{1-3}
        1A81 2 (20) & -1.0069 & -1.0069  \\
        1G9O 1 (20) & -1.2525 & -1.2525 \\
        1A81 3 (30) & -1.2025 & -0.7606 \\
        1BM2 4 (30) & -2.3854 & -1.1104 \\
        1G9O 5 (30) & -2.3857 & -1.8477 \\
        \toprule


 \end{tabular}      
 \caption{Ecarts d'énergie avec des trajectoires 2 fois plus longues pour le protocole RE8b2}
\label{tab:RE8b2+}       
\end{table}




   \paragraph{Les énergies de références}


Les énergies de références ont été optimisées sur les fréquences en acides aminés de mes neufs proteines plus 2PTK (SH3).L'optimisation débute avec des énergies de références de Marcel.Elles ne sont pas dans les publications. 



    \begin{table}[h]
      \centering

      \begin{tabular}{ccccc}

        \toprule
        Acide Aminé & ref ener & initial ref ener\\
        \cmidrule{1-2}
        A & -8.519   & -10.89 \\
        C & -9.426   & -25    \\
        d & -17.824  & -18    \\
        D & -19.699  & -19.29 \\
        e & -17.824  & -18    \\
        E & -20.048  & -20.86 \\
        F & -18.017  & -16.51 \\
        h & -21.704  & -21.88 \\
        I & -10.530  & -12.6  \\
        K & -20.999  & -21.63 \\
        L & -12.227  & -12.26 \\
        M & -11.985  & -13.3  \\
        N & -16.715  & -16.4  \\
        Q & -17.718  & -16.7  \\
        R & -22.107  & -24.95 \\
        S & -11.881  & -13    \\
        T & -11.413  & -12.13 \\
        V & -9.5654  & -10.5  \\
        W & -19.267  & -20.91 \\
        Y & -20.893  & -18.62 \\

        \toprule


 \end{tabular}      
 \caption{}
\label{tab:H+_H++}       
\end{table}


\paragraph{Superfamily}

Nous travaillons avec la base de données SCOP à la version 1.75 et les outils SAM à la version 3.5.
Les scripts superfamily utilisent le programme hmmscan avec le paramètrage suivant:«-E 10 -Z 15438 » .(E-value seuil et nombre de comparasions a effectuées)
Le script ass3.pl est paramétré avec:«-t n -f 8 -e 10 »( exécution parallèle sur huit coeurs et E-value seuil). 



    \begin{table}[h]
           \raggedleft{}

      \begin{tabular}{ccccccc}

        \toprule
        Protein & seq  & Match/seq & Superfamily & superfamily & Family & family \\
        Protein & number & size & Evalue & success & Evalue &  success\\
        \cmidrule{1-7}
        1A81 & 236 & no & & & & \\
        1ABO & 203 & 51/58  & 4.4e-4 & 100\% & 2.8e-3 & 100\% \\
        1BM2 & 209 & 78/98  & 4.2e-5 & 100\% & 2.6e-3 & 100\% \\
        1CKA & 416 & 40/57  & 1.1e-5 & 100\% & 3.4e-3 & 100\% \\
        1G9O & 338 & 79/91  & 7.0e-7 & 100\% & 2.5e-3 & 100\%  \\
        1M61 & 405 & 97/109 & 7.2e-7 & 100\% & 2.6e-4 &  100\% \\
        1O4C & 274 & 95/104 & 2.1e-4 & 100\% & 4.5e-3 &  100\% \\
        1R6J & 270 & 74/82  & 9.8e-6 & 100\% & 4.6e-3 &  100\% \\
        2BYG & 426 & 59/97  & 1.4e-5 & 100\% & 7.1e-3 &  100\% \\
        \bottomrule        
      \end{tabular}   
     \caption{Résultats Superfamily pour le protocole RE8b2}   
\label{tab:superfamily_bestRE}       
\end{table}

\paragraph{Empreinte mémoire}

Proteus avec le protocole RE8b2 et la plus grosse protéine 1M61 utilise selon pmap, 2153312K de mémoire vive.



\paragraph{toulbar2}

La décomposition par paire de notre fonction d'énergie permet une représentation des énergies sous forme d' "un réseau de fonctions de coûts".Dans lequel chaque interaction entre acides aminés est représenté par une arête dans le réseau et ou l'énergie d'un rotamère est représentée par un noeud dans le réseau.
Un ensemble de transformations qui préservent l'équivalence des problèmes est appliqué sur le réseau. Elles agissent en déplaçant des coûts entre les fonctions avec pour objectif de transférer les coûts vers des fonctions d'arité un ou zéro. Cela permet une mise à jour d'un minorant m et d'un majorant M du GMEC. L'arbre de décision associé à l'algorithme de recherche (de type «depth-firstbranch-and-bound») est «élagué» à partir de m et M. Ce processus est appliqué itérativement.





%%% Local Variables:
%%% mode: latex
%%% TeX-master: "rapport"
%%% End:
